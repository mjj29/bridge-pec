\documentclass[a4paper,12pt]{article}

\usepackage{palatino}
\usepackage{fullpage}

\author{Matthew Johnson\\Henry Lockwood}
\title{System notes on Pascal's Encrypted Club}

\begin{document}

\maketitle

\section{A brief history}

NB: the system has changed sufficiently that this history is out-of-date.  Do
not rely on it to describe our bidding!

We wanted to play a system making use of encrypted bidding.  Our first attempt
used a fairly standard Acol structure, with a weak NT and three weak 2s.  It
was soon apparent that crypto raises in these circumstances (requiring 11+,
4-card support and either A or K but not both) never come up, at least to a
first approximation.

We also quite liked the idea of light and limited openings.  This was from a
couple of experiments with a (highly illegal) forcing pass system, where a pass
in first or second seat showed any 12+, with 0--11 HCP hands forced to open
somehow.  In particular, 2-suited bids at the 1-level have a lot of
attractions.

Our next attempt was to consider the idea of a 12+ 1C opener, with all other
openings 8--11.  This is fine in theory, but EBU regulations say that this 1C
opener cannot have a 5+ card major suit without a longer minor.  This meant
that we needed an opening-values 2C bid, and generally lost the point of the
system.  We fell back upon a strong club structure.

Given the strong C structure, we then looked at our other opening bids.  Our
2-level openers are pre-emptive, and show either single-suiters or two-suiters.
This is based upon the Suction convention: 2C shows D or both majors, 2D shows
H or S\&C, 2H shows S or C\&D, and 2S shows H and a minor.  We don't have a way
to show a S\&D two-suiter; this is a likely future use for 2NT.

Major suit openings are played as takeout bids.  This means that 1S promises 5+
H and 0--3 S, with 1H promising 5+ S and 0--3 H.  These were, at one stage,
4-card openings in the other major, but we found that this made bidding in
competition too difficult.  In uncontested auctions, this has the great
advantage that you gain a "free" cue-bid: after a 1S opening, 2S shows a good
raise of H while 2H, 3H etc. are pre-emptive.

The next implication is that we need a way to show hands with 4+ in both
majors.  This falls to 1NT; we have 2C/2D/2H/2S as weak bids "to play", and 2NT
as a strong enquiry.  Bids in the major suits are primarily pre-emptive; 2NT
asks for range and better major.  3m is strong and one-round forcing.

1C includes balanced hands 16--19 and 23+ (we're using 2NT as a 20--22 balanced
hand, for now).  The 1D opening, therefore, has to cover all opening hands that
do not fall into one of the above categories.  It therefore includes:

\begin{itemize}
\item 12--15 balanced.
\item Unbalanced hands, 8--15, with no 5-card major and at most one 4-card
      major.
\end{itemize}

Responses to 1D are selected to allow the 1NT rebid, showing a weak NT opener,
on all auctions.  Therefore: 1H = 0--7, 1S = 8+.  After interference, pass is
weak; any other action promises 8+.

So that's our one-level openings.  For two-level openers, we wanted them to be
the two-suited bids we'd originally considered for our one-level openings.  I
can't remember if it was Matt or me who first found out about Suction as a
defence to 1NT, or who suggested it as a structure for 2-level openings, but
the fact remains that we like it.  Essentially, each 2-level bid shows either a
weak 2 in the suit above OR the other two suits.  Of course, this leaves the
non-touching pairs uncovered, and leaves you with two ways to show a pre-empt
in clubs.  We soon thought of playing 2S as showing H and a minor (as opposed
to the original C or H\&D).  This still left S\&D uncovered, but we decided to
cope.

At this stage, our 2NT opening was fairly standard: 20--22 balanced or
semi-balanced.  After a few sessions playing, we modified this to include the
spade-diamond two-suiters.  Of course, that makes the bid forcing, so we
changed the NT ranges such that this included our game-forcing NT hands: any
23+.  Responses are suit preference, on the assumption that the weak hand is
more frequent, with transfer preference being used if responder has a strong
hand.  If opener has the strong option, he can rebid 3NT.

The next change (there's still more!) was to include GF two-suiters in our
2-level openings.  This narrows the specification of the 1C opening, and is
easy to specify: opener can rebid his opened suit (whichever it is) over any
response.  In the 2NT case, this opening now shows GF (semi-)balanced, GF with
S\&D, or weak with S\&D.  In the two-suited case, any rebid other than 3NT shows
the strong hand, and any transfer break does also.

Finally, our responses to 1C: initially, these were natural, promising a 4-card
suit and 8+ (with any weak hands going via 1D).  Boring!  We then tried
encrypted responses: 1H promised HA or HK, not both; 1S showed neither or both
in H, and exactly one in S, and so on.  1NT was a positive with no potential
key; 1D was the negative.  Opener would rebid NT if balanced, 2H to deny a
potential key, and anything else to show a potential place to play if holding
the key K or denying if holding the key A.  We ditched this structure not
because it doesn't work but because it frequently (read nearly always) ends up
at the 3-level before either hand has actually mentioned a suit, which could
still be only 4 cards.

We now play a far more elegant structure, similar to our responses to 1D: 1D is
negative (0--7), 1H is positive distributional (8+), and 1S is positive balanced
(8+, again).

As for the system name: suction is measured in Pascals, and the system is
encrypted.  It went from there.

\section{System description}

\subsection{Philosophy}

\begin{itemize}
\item establish strength first
\item establish shape (balanced/unbalanced) second
\item then find a contract
\item single-suited hands are all 6+ cards and deny a side 4-card suit.
\end{itemize}

\subsection{Openings}

\subsubsection{1 level}

\begin{tabular}{ll}
1C  & 12--15 balanced or 10--15 unbalanced, no five-card major and at most one \\
	 & four-card major.\\
1D  & 16+; excludes balanced hands 24+ and game-forcing two-suited hands.\\
1H  & 10--15 unbalanced, with 4+ spades and additional constraints: first or
      second seat\\
	 & non-vulnerable, possible canape with hearts, but 5+ spades if
      not holding 4+ hearts. \\
	 & Other positions, or vulnerable, 5+ spades and 0--3 hearts.\\
1S  & 10--15 unbalanced, 5+ hearts, and 0--3 spades.\\
1NT & 9--11 balanced in first or second non-vulnerable; otherwise 10--15
      unbalanced with \\
	 & 4+ in both majors.\\
\end{tabular}

\subsubsection{2 level}

\begin{tabular}{ll}
2C  & weak with D, single suited OR weak with H\&S, 4--4 or better OR GF with H\&S.\\
2D  & weak with H, single suited OR weak with S\&C, 5+/4+ or better OR GF with S\&C.\\
2H  & weak with S, single suited OR weak with C\&D, 4+/4+ OR GF with C\&D.\\
2S  & weak with H and a minor OR GF with H and a minor. 4+/4+.\\
2NT & weak with D\&S, 4+/4+, OR 24+ balanced OR GF with D\&S.\\
\end{tabular}

\subsubsection{Higher openings}

\begin{tabular}{ll}
3-level & natural in suits. \\
3NT & gambling. \\
Higher-level openings & natural. \\
4NT & specific ace-asking. \\
\end{tabular}

\newpage 

\subsection{Responses \& rebids}

\begin{itemize}
\item 1C:
	\begin{itemize}
   \item 1D = 0--7, no 6-card suit
   \item 1H = 8--15 any
   \item 1S = 15+ any, game-forcing.
		\begin{itemize}
      \item major suit bids promise exactly 4; either 4-4-4-1 with singleton other
         major or has a longer minor (as there's no other hand shape that fits the
         opening and has exactly one 4-card major)
      \item 1NT rebid = 12--15 balanced; see "Continuations after 1NT".
      \item minor suit bids deny a 4-card major.
		\end{itemize}
   \item 2x = 6+ card suit, 0--7. Non-forcing.
		\begin{itemize}
      \item Opener will pass unless holding something exciting in context.
         Principle: find a fit and find a level. No close games at matchpoints.
		\end{itemize}
	\end{itemize}

\item 1D:
	\begin{itemize}
   \item 1H = 0--7, no 6-card suit.
   \item 1S = 8+, game-forcing.
		\begin{itemize}
      \item 1NT rebid = 16--19 balanced; see "Continuations after 1NT".
      \item Simple suit rebids are suction-style, showing either the suit above
         (single-suited) or the other two (C$\rightarrow$ D or majors; D-> H or (S\&minor); H->
         S or minors; S$\rightarrow$C or (H\&minor). Could be 3-suited, short in the suit
         above. These are shown (2C-2D-3C: short D. 2D-2H-2S: short H. 2H-2S-3H:
         short S. 2S-3C-3S: short C.).
			\begin{itemize}
			\item Responder will bid the next suit (relay). Opener then rebids NT with the
				single-suited hand, or a suit with the two-suited option.
			\end{itemize}
      \item Jump suit rebids are GF, single-suited.
      \item 2NT rebid = 20--23 balanced; see "Continuations after 2NT".
		\end{itemize}
   \item 2x = 6+ card suit, 0--7. Non-forcing.
		\begin{itemize}
      \item Opener will pass unless holding something exciting in context.
		\end{itemize}
	\end{itemize}

\item 1H (if denying H):
	\begin{itemize}
   \item 1S/2S/3S/4S = to play/limit raises/to play.
   \item 1NT = 1-round force.
   \item 2C/D = natural, forcing.
   \item 2H = GF+ raise, neither or both of S AK.
   \item 2NT = good raise, with exactly one of S AK.
		\begin{itemize}
      \item Next bid shows 0/1 loser and the S K, or 2/3 losers and the SA.
         Rebidding S denies the other top honour.
		\end{itemize}
   \item 3C/3D = fit jump.
   \item 3H = GF raise, two of AKQ.
   \item 4C/D/H = splinter agreeing spades
	\end{itemize}

\item 1S:
	\begin{itemize}
   \item 2H/3H/4H = to play/limit raise/to play.
   \item 1NT = 1-round force.
   \item 2C/D = natural, forcing.
   \item 2S = GF+ raise, neither or both of H AK.
   \item 2NT = good raise, with exactly one of H AK.
		\begin{itemize}
      \item Next bid shows 0/1 loser and the H K, or 2/3 losers and the HA.
         Rebidding H denies the other top honour.
		\end{itemize}
   \item 3C/3D = fit jump.
   \item 3S = GF raise, two of AKQ.
   \item 4C/D/S = splinter agreeing hearts
	\end{itemize}

\item 1H if not denying H:
	\begin{itemize}
   \item 1S = forcing enquiry.
		\begin{itemize}
      \item 1NT = both majors. Continuations as after artificial 1NT opening.
		\end{itemize}
   \item 1NT = non-forcing.
   \item 2C/D = natural, forcing.
   \item 2H = GF+ raise, neither or both of S AK.
   \item 2S/3S/4S = limit raises/to play.
   \item 2NT = good raise, with exactly one of S AK.
		\begin{itemize}
      \item Next bid shows 0/1 loser and the S K, or 2/3 losers and the SA.
         Rebidding S denies the other top honour.
		\end{itemize}
   \item 3C/3D = fit jump.
   \item 3H = GF raise, two of AKQ.
   \item 4C/D/H = splinter agreeing spades
	\end{itemize}

\item 1NT if artificial:
	\begin{itemize}
   \item 2C/D: weak, to play.
   \item 2H/S: preference, to play.
   \item 2NT: forcing enquiry:
		\begin{itemize}
      \item 3C/D strong, 3H/S weak. 3C/H better H, 3D/S better S.
		\end{itemize}
   \item 3C/D: strong, good suit.
   \item 3H/S: good raise.
	\end{itemize}

\item 2C:
	\begin{itemize}
   \item 2D = pass/correct.
		\begin{itemize}
      \item 2NT = GF opening.
		\end{itemize}
   \item 2H = forcing; major-focused.
		\begin{itemize}
      \item 2S = better S
      \item 2NT = both equal
      \item 3D = weak, diamonds.
      \item 3C = GF opening
      \item 3H = better H
		\end{itemize}
   \item 2NT = forcing, how good are your diamonds?
		\begin{itemize}
      \item 3C = GF opening
      \item 3D = bad diamonds
      \item 3H = 2-suited
      \item 3S = decent diamonds
      \item 3NT = top-of-range diamonds
		\end{itemize}
	\end{itemize}

\item 2D:
	\begin{itemize}
   \item 2H = pass/correct.
		\begin{itemize}
      \item 2NT = GF opening.
		\end{itemize}
   \item 2S = forcing, C/S-focused.
		\begin{itemize}
      \item 2NT = both equal.
      \item 3C = better C.
      \item 3D = GF opening.
      \item 3H = better S, minimum.
      \item 3S = better S, maximum.
		\end{itemize}
   \item 2NT = forcing, how good are your hearts?
		\begin{itemize}
      \item 3C = two-suited.
      \item 3D = GF opening.
      \item 3H = dire, single-suited.
      \item 3S = decent, single-suited.
      \item 3NT = top-of-range hearts.
		\end{itemize}
	\end{itemize}

\item 2H:
	\begin{itemize}
   \item 2S = pass/correct.
		\begin{itemize}
      \item 2NT = GF opening.
		\end{itemize}
   \item 3C = forcing, C/D-focused.
		\begin{itemize}
      \item 3D = better D.
      \item 3H = GF opening.
      \item 3S = single-suited.
      \item 3NT = both equal.
      \item 4C = better C.
		\end{itemize}
   \item 2NT = forcing, how good are your spades?
		\begin{itemize}
      \item 3C = two-suited.
      \item 3D = dire spades.
      \item 3H = GF opening.
      \item 3S = decent, single-suited.
      \item 3NT = top-of-range spades.
		\end{itemize}
	\end{itemize}

\item 2S:
	\begin{itemize}
   \item 2NT = forcing, which minor?
		\begin{itemize}
      \item 3C/D: that minor
      \item 3H: GF, H\&C
      \item 3S: GF, H\&D
		\end{itemize}
   \item 3C/D/H, etc: to play (pass/correct if a minor).
   \item 3S = setting H, asking for information
	\end{itemize}

\item 2NT:
	\begin{itemize}
   \item 3D/3S: simple preference
		\begin{itemize}
      \item After weak preference, any rebid shows GF hand.
		\end{itemize}
   \item 3C/3H: transfer preference; stronger.
		\begin{itemize}
      \item Opener breaks transfer if GF two-suiter; suit has been set so bid controls.
      \item A 3NT rebid shows 24+ balanced
			\begin{itemize}
         \item 3C/D = 2 under transfers to H/S
         \item 3H/S = Swiss for no trumps
         \item 4N = Quantitative
         \item 5N = Quantitative (for 6 or 7)
			\end{itemize}
		\end{itemize}
	\end{itemize}

\item 3NT:
	\begin{itemize}
	\item pass: stops in 3 suits
	\item clubs at any level: pass or correct
	\item 4D: asks for singletons
		\begin{itemize}
		\item 4H/S: singleton in that suit
		\item 4N: no singletons
		\item 5C/D: singleton in the other suit
		\end{itemize}
	\end{itemize}

\newpage

\item 4NT:
	\begin{itemize}
	\item 5C: No aces
	\item 5D: Club ace or Heart, Diamond and Spade aces
	\item 5H: Diamond ace or Club, Heart and Spade aces
	\item 5S: Heart ace or Club, Diamond and Spade aces
	\item 5N: Two aces not including Spades
	\item 6C: Spade ace or Club, Diamond and Heart aces
	\item 6D: Spade and Club aces
	\item 6H: Spade and Diamond aces
	\item 6S: Spade and Heart aces
	\end{itemize}

\end{itemize}

\section{Continuations after 1NT}

This refers only to a natural 1NT, though it may be 9--11, 12--15 or 16--19
depending on route. Maximum is an 11-count, 14--15-count, or 18--19-count
depending on range.

\begin{itemize}

\item 2D and 2H are transfers; transfer breaks apply with any 4-card fit.  2NT
shows 4/5-card support, no side 4-card suit, and a maximum.  Other suits
show 4 cards, and 4-card support with a maximum.  A simple jump acceptance
shows 4-card support and a minimum.

\item 2S is a range enquiry/transfer to C.  2NT shows a minimum, after which 3C is to
play and other bids are GF, either cues or Swiss.  3C shows any maximum, and
can be passed if 2S was weak takeout.

\item 2NT is a transfer to diamonds; 3C is the only available super-accept.

\item 3C promises a long suit that may need some help to run.  Opener should
pass, or bid 3NT with (e.g.) Kxx in the suit.

\item 3D is 5--5 in the majors, at least invitational.

\item 4N is blackwood.

\item 2C is where it gets distinctive.  This is 5-card modified puppet Keri, which allows the system to be used with weak diamond hands with tolerance for at least one major (e.g. 4=0=6=3 pattern with very limited values).
\end{itemize}

\paragraph{Responses to 5-card puppet Keri}

	\begin{itemize}
	\item 2H/S with a 5-card major
		\begin{itemize}
		\item 3D to play
		\item other rebids are as after this sequence in 5 card puppet stayman.
		\end{itemize}
	\item 2D with all other hands. 
		\begin{itemize}
		\item Pass if weak with diamonds
		\item 2H = 4 spades; may have 4 hearts.
			\begin{itemize}
			\item 2S: forcing; promises 4 hearts.
				\begin{itemize}
				\item 2NT/3NT: inv/GF, not 4 hearts.
				\item 3C/3D: 4 hearts, feature in suit bid; GF
				\item 3H/4H: inv/GF, 4 hearts.
				\item 3S: GF, 4 hearts, auto-Swiss.
				\item 4C/4D: GF, 4 hearts, Swiss.
				\end{itemize}
			\item 2NT/3NT: min/max, denies a 4-card major.
			\item 3C/3D/3H: 4 spades, feature in suit bid; max.
			\item 3S/4S: min/max, 4 spades.
			\item 4C/4D: max, non-serious Swiss.
			\end{itemize}

		\item 2S = 4 hearts; denies 4 spades.
			\begin{itemize}
			\item 2NT/3NT: min/max, denies 4 H.
			\item 3C/3D: 4 hearts, feature in suit bid, max.
			\item 3H/4H: min/max, 4 hearts.
			\item 3S: 4 hearts, maximum, auto-Swiss.
			\item 4C/4D: GF, 4 hearts, Swiss.
			\end{itemize}

		\item 2NT = 3--3 majors; invitational strength.
		\item 3C/D = 3--3 majors, GF, values in the minor.
		\item 3NT=3--3 majors, GF.
		\end{itemize}
	\end{itemize}

\subsection{After 1NT is doubled for penalties}

We play a modified form of suction as the escape after 1NT-X.

\begin{itemize}
\item Pass forces XX, to play or weak with a two-suiter, not the reds or the majors
	\begin{itemize}
	\item XX forced
		\begin{itemize}
		\item Pass with a strong hand, to play
		\item bid 4 card suits up the line
		\end{itemize}
	\end{itemize}
\item XX forces 2C, weak with clubs or the reds
	\begin{itemize}
	\item 2C forced
		\begin{itemize}
		\item Pass with clubs
		\item 2D with D\&H
		\end{itemize}
	\end{itemize}
\item 2C forces 2D, weak with diamonds or the majors
	\begin{itemize}
	\item Pass with 5 clubs
	\item 2D with all other hands
		\begin{itemize}
		\item Pass with diamonds
		\item 2H with the majors
		\end{itemize}
	\end{itemize}
\item 2D forces 2H, weak with hearts
	\begin{itemize}
	\item Pass with 5 diamonds
	\item 2H all other hands
	\end{itemize}
\item 2H forces 2S, weak with spades
	\begin{itemize}
	\item Pass with 5 hearts
	\item 2S all other hands
	\end{itemize}
\end{itemize}

\subsection{After direct overcalls of 1NT}

Lebensohl:

\begin{itemize}
\item suits at the 2 level: to play
\item suits at the 3 level: GF, natural
\item direct cue: staymanic, denies a stop
\item 3N: natural, denies a stop
\item 2N: puppet to 3C
	\begin{itemize}
	\item 3C forced
		\begin{itemize}
		\item suits below the cue: to play
		\item suits above the cue: invitational, natural
		\item cue: staymanic, promises a stop
		\item 3N: natural, promises a stop
		\end{itemize}
	\end{itemize}
\end{itemize}

\section{Continuations after 2NT}

\begin{itemize}
\item 3C: advanced 5-card puppet Stayman
	\begin{itemize}
   \item 3H/S shows 5 cards.
   \item 3D promises 4 hearts or 3 spades
		\begin{itemize}
      \item 3H: 0--3 hearts, 4 spades
			\begin{itemize}
         \item 3NT: 4 hearts
         \item 4C/D: Swiss
         \item 4S: to play.
			\end{itemize}
      \item 3S: 4 hearts, 0--3 or 5 spades
			\begin{itemize}
         \item 3NT: 3/4 spades (Responder bids 4S if 5/4)
         \item 4C/D: Swiss
         \item 4H: to play.
			\end{itemize}
      \item 3N: 3--3 majors
      \item 4C: 4--4 majors, invitational
      \item 4D: 4--4 majors, GF
		\end{itemize}
   \item 3NT denies a 4-card major.
	\end{itemize}

\item 3D/3H: transfers.  Superaccept with any 4-card fit; note that 3NT is a
superaccept that still allows the use of Swiss.

\item 3S: minor-suit Stayman.  Asks for a 4- or 5-card minor.
	\begin{itemize}
   \item 3NT: no 4-card minor.
   \item 4C/4D: 4 cards.
   \item 4H/4S: 5 cards in the corresponding minor.
	\end{itemize}

\item 4C/D: Swiss
\item 4N: Quantitative
\end{itemize}

\section{Encrypted bidding}

Once responder makes an encrypted raise, opener can either accept or decline
the key.  To decline, simply rebid the suit (cheaply with a minimum; jump if
maximum.  Use common sense.).

If the key is accepted, subsequent bids are either two-way or three-way, depending on the key.

\subsection{Two-way encryption}

This is far more common, as it requires responder and opener to have one each
of the trump AK. Opener's rebid shows a useful side-suit if holding the trump
K (0--1 losers) or a potential weakness if holding the trump A (2--3 losers).
Responder can then either place the contract, cue-bid (either showing with the
K or denying with the A), or use Blackwood.

Blackwood here is Roman Keycard, but it's modified because there are only three keycards
still of interest.

\subsection{Three-way encryption}

Requires AKQ of trumps between the two hands, and enough values for a
game-forcing raise in responder's hand.  Rare!

Opener's rebid shows a first-round control; the location of this control is
encrypted.  We use the concept of the shift here: a shift of 1 indicates the
next suit up (C from NT or S, D from C, H from D, S from H).  A shift of 2
indicates the next suit after that, and a shift of 3 shows the final suit.
Each cue therefore says nothing about control in the suit bid.

The sum of the shift and the HCP of opener's honour is 5.  Opener, holding a D
control, therefore bids C with the trump A, S/NT with the trump K, and H with
the trump Q.

Of course, Blackwood goes funny here too.  You only need to show 0--3 keycards,
and the location of the Q is known, so 0--1--2--3 would seem a sensible set
of responses.

\section{Slam conventions}

\subsection{Roman key-card Blackwood (3014)}

\subsection{First-round-control showing cues}

Once a suit is agreed and a GF is established, new suits are cues.

\subsection{General Swiss}

4C/4D, once a game-force is established and suit agreed, are slam tries, based
on control points

Control points are 2 for an ace, 1 for a king/singleton, and 1 for the queen of
trumps.  There are 13 available from honours; it is possible to substitute
outside singletons for kings, but care must be taken to avoid double-counting.
11 CPs are necessary for a small slam; 13 for a grand.

\begin{itemize}
\item 4C: 4 or 6 (exceptionally 8) control points)
	\begin{itemize}
   \item trump suit: 0--4 CPs
   \item 4D: 5/6 CP, relay
		\begin{itemize}
      \item trump suit: 4 CPs
      \item suits: 6 CP, lowest king/singleton
		\end{itemize}
   \item suits: 7 or 9 CP, lowest king/singleton
   \item small slam: 8 CP
   \item grand slam: 10 CP
	\end{itemize}

\item 4D: 5 or 7 (exceptionally 9) control points)
	\begin{itemize}
   \item trump suit: 0--3 CPs
   \item 4H: 4/5 CP, relay
		\begin{itemize}
      \item trump suit: 5 CPs
      \item suits: 7 CP, lowest king/singleton
		\end{itemize}
   \item suits: 6 or 8 CP, lowest king/singleton
   \item small slam: 7 CP
   \item grand slam: 9 CP
	\end{itemize}
\end{itemize}

If you find duplication in kings/singletons subtract one CP and sign off at the
appropriate level. If your partner signs off and you have 2 unshown CPs so far,
raise one level.

If the hand bidding Swiss has shown a GF, add 2 to all numbers. If it has shown
a weak hand (opening at the 2 level or giving a 0--7 response to 1D), subtract 2
from all numbers.

\section{Competitive bidding}

Overcalls are constructive and for takeout, promising the two unbid suits (at least 5--4).
Where opponents have shown two suits, overcalls are natural.

When responding to an overcall, a bid of one of the suits is simply competitive
and showing preference.  A good raise of the cheaper suit can be shown with the
cheaper cuebid, and a good raise of the other suit can be shown with the other
cuebid.  

Similarly, 1NT and 2NT can be used as encrypted raises of the cheaper and
higher suits respectively.


This means that a double needs to include the possibility of a single-suited
hand, on which one might make a natural overcall.  Responder should make the
cheapest bid if holding a very weak hand.

All strong hands are shown via an immediate cuebid of opponent's suit.  Again,
a next-step response is negative.

Two-suited pre-emptive hands can be shown via 2NT.  It is worth noting that
this is the Tom Jones 2NT: It's Not Unusual.  This promises both majors over a
minor-suit opening, or the other major and an unspecified minor over a
minor-suit opening.  

Jump overcalls are weak; jump overcalls of a major suit over a minor-suit
opening may include the other minor (possible canape).  Similarly, jump
overcalls of 3C over a 1 of a major opening could include D.

Jump cuebids are stopper-asking for 3NT; they promise a long running suit.  An
overcall of 3NT promises a long running suit and a stopper.

\subsection{Defences}

These are mostly all based on suction:

\begin{itemize}
\item clubs: diamonds or the majors
\item diamonds: hearts or spades and clubs (or diamonds)
\item hearts: spades or the minors
\item spades: clubs or hearts and diamonds (or clubs)
\end{itemize}

\subsubsection{Natural 1NT}

Suction, constructive values. Double is for penalties and 2N shows a
non-touching two suiter.

\subsubsection{Artificial strong bids}

Suction, weak. Double shows the suction bid in their suit and 2N shows a
non-touching two suiter.

\subsubsection{Phoney/Prepared suits}

Suction, constructive values. Double shows the suction bid in their suit and 2N
is natural.

\subsubsection{Multi 2D}

Dixon:

\begin{itemize}
\item X: 16+
\item 2H/S: takeout of the other major
\item 2N: 16-18 bal, stops in both majors.
	\begin{itemize}
	\item Responses as "Continuations after 2NT"
	\end{itemize}
\item 3x: intermediate, natural
\end{itemize}

\subsubsection{Natural weak openings}

Double is for takeout up to 4H. After a weak two and a takeout double, there
are lebensohl responses. 4N is takeout over 4S and a big two suiter otherwise.

\section{Dealing with interference}

\subsection{Direct interference over 1C}

\begin{itemize}
\item 1C-(X) (penalty interest (showing 16+):
	\begin{itemize}
   \item Redouble creates a game-force, and indicates that the double may not have been 
      a winning action.
   \item Pass forces redouble; either to play or with a two-suiter not including clubs.
   \item 1D/1H are transfers to 1H/1S.  Note that opener may pass these if holding a 5-card 
      suit.
	\end{itemize}
\item 1C-(X) (showing C):
	\begin{itemize}
   \item Redouble creates a game-force; passes are now forcing and doubles are for penalty.
   \item Pass shows a weak hand, with nowhere in particular to play.
   \item Suit bids are natural, 8--14, and promise 4 cards.
   \item 1NT shows a balanced hand, 8--14, with a club stop.
	\end{itemize}
\item 1C-(X) (takeout):
	\begin{itemize}
   \item Redouble still creates a forcing pass situation.
   \item Ask opponents "takeout of what, exactly?".
   \item 1NT cannot show a stopper in any particular suit.
	\end{itemize}
\item 1C-(natural bid):
	\begin{itemize}
   \item Responder's double creates a game-force.
   \item 1NT promises a stop; 8--14 balanced.
   \item Suit bids are natural, 8--14.
   \item Pass is a weak hand with nowhere to go (potentially including an 8--11 balanced hand with no stop).
	\end{itemize}
\end{itemize}

\subsection{Direct interference over 1D}

\begin{itemize}
\item 1D-(X):
	\begin{itemize}
   \item Redouble with good 8+ hands.
   \item Pass with weak hands.
   \item Suit bids are natural, weak, 6-card suit.
	\end{itemize}
\item 1D-(bid):
	\begin{itemize}
   \item Pass with weak hands.
   \item Double to establish a game-force (and forcing passes, etc.).
		\begin{itemize}
      \item Suction rebids are on; immediate NT bid denies a stop.
      \item To show a stop, Suction-force into opponents' suit then bid NT.
		\end{itemize}
   \item Suit bids are natural, weak, 6-card suit.
	\end{itemize}
\end{itemize}

\subsection{Balancing-seat interference (e.g. 1D-(P)-1S-(2C))}

\begin{itemize}
\item 1C-(P)-1S-(anything):
\item 1D-(P)-1S-(anything):
\begin{itemize}
   \item Passes are forcing
   \item NT bids promise a stopper.
   \item Suction-style rebids are on if 3C is a sufficient bid (1D opener)
   \item Pass indicates a balanced hand without a stopper in opponents' suit.
	\begin{itemize}
      \item Responder's NT bids here are Lebensohl style
      \item There are no weak hands to show, so the only difference is in showing/denying a stopper.
		\end{itemize}
   \item Double is penalties.
	\end{itemize}

\item 1C-(P)-1D-(anything): it's a good idea to subside quietly.  

\item 1C-(P)-1H-(anything): passes are non-forcing. 
\end{itemize}

\section{Carding}

\subsection{Leads}

Leads are standard (2nd from bad suits, 4th from an honour, top of sequences),
except that they may vary based on signal requested (see below).

The exception is that we lead top of bad suits against no trumps.

After the initial lead we lead high from bad suits and low from good suits.

\subsection{Signals}

We play two forms of signalling. On partner's lead we play reverse attitude
(low encouraging, high discouraging), except if the initial lead was an odd
card. In that case we play Prism signals in that suit (see below). We also play
Prism signals when trumps are lead by either side, or if there is a suitable
obvious running suit in no trumps.

Leads to ruffs and other obvious situations may be McKenny.

\subsection{Discards}

Discards are Italian style, so an odd card is encouraging in the suit discarded
and an even card is McKenny. Thus, low even cards ask for the lower of the
outside suits and a high even card asks for the higher of the two outside
suits.

\subsection{Prism}

Prism is a signalling system based on the shape of the whole hand. An initial
hand of 13 cards will, perforce, have either one suit which is odd (and three
even) or one suit which is even (and three odd). This denotes the parity of the
hand.

The first card in a prism sequence will give the original parity of the hand.
Low for odd, high for even. Second and subsequent cards will indicate the suit
which is unique in parity. Take the spade suit 2 3 6, the following sequences indicate
the following parities and suits:

\begin{tabular}{llll}
\bf order & \bf parity & \bf outside suit & \bf relative to spades \\
\hline
2 3 6 & odd    & higher       & hearts \\
2 6 3 & odd    & middle       & diamonds \\
3 6 2 & odd    & lower        & clubs \\
3 2 6 & even   & higher       & hearts \\
6 2 3 & even   & middle       & diamonds \\
6 3 2 & even   & lower        & clubs \\
\end{tabular}

This information can be combined with the other defender's hand and dummy to
deduce the shape of declarer's hand as follows. Add together the length of your
suits and dummy's, then subtract each from 13. This will give you either four
even numbers, four odd numbers or two of each.

\begin{itemize}
\item If all four are even then declarer's unique suit is the same as partners and the same parity.
\item If all four are odd then declarer's unique suit is the same as partners and the other parity.
\item If partner's suit is in the even pair, then declarer's suit is the other one in that pair and the other parity.
\item If partner's suit is in the odd pair, then declarer's suit is the other one in that pair and the same parity.
\end{itemize}

\end{document}
