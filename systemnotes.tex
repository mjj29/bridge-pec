\documentclass[a4paper,14pt]{extarticle}

\usepackage{palatino}
\usepackage{fullpage}

\usepackage[
  bookmarks,
  bookmarksopen,
  bookmarksnumbered=false,
  colorlinks,
  pdfpagemode=None,
  urlcolor=black,
  citecolor=black,
  linkcolor=black,
  pagecolor=black,
  plainpages=false,
  pdfpagelabels,
  pdftitle={System notes on Pascal's Encrypted Club},
  pdfauthor={Matthew Johnson and Henry Lockwood}]{hyperref}

\author{Matthew Johnson\\Henry Lockwood}
\title{System notes on Pascal's Encrypted Club}

\usepackage{ifthen}
\usepackage{color}
 
\newcommand{\newevenside}{
	\ifthenelse{\isodd{\thepage}}{\newpage}{
	\newpage
	\textcolor{white}{placeholder} 
	\thispagestyle{empty}
	\newpage
	}
}

\begin{document}

\newcommand{\orange}[1]{[OB #1]}
\newcommand{\xref}[1]{[\ref{#1}]}

\maketitle
\tableofcontents

\section*{Notes from system card}
\begin{tabular*}{\textwidth}{ll@{\extracolsep{\fill}}r}
1. & 2NT encrypted raise & \pageref{note:1} \\
2. & Jump fit & \pageref{note:2} \\
4. & 1$\spadesuit$ negative & \pageref{note:4} \\
5. & 1N F1 response & \pageref{note:5} \\
6. & 3M/4m splinter & \pageref{note:6} \\
7. & 2N enquiry responses & \pageref{note:7} \\
8. & 3$\clubsuit$ 5-card puppet Stayman & \pageref{note:8} \\
8a. & 2$\clubsuit$ 5-card puppet Keri & \pageref{note:8a} \\
9. & Responses to a natural 1NT & \pageref{note:9} \\
10. & 2NT--3$\spadesuit$ & \pageref{note:8} \\
13. & Lebensohl & \pageref{note:13} \\
14. & Responses to 1$\clubsuit$ & \pageref{note:14} \\
15. & Dixon & \pageref{note:15} \\
18. & Responses to 2x & \pageref{note:18} \\
19. & General Swiss & \pageref{note:19} \\
20. & Suction wriggle & \pageref{note:20} \\
\end{tabular*}

\newpage

\section{Administriva}
\label{sec:admin}

\subsection{Notation}
\label{sec:notation}

References to the EBU Orange Book will be given as \orange{11C3}. References
elsewhere in this document will be given as \xref{sec:resp:2level}.

\subsection{System regulation}

Allowed systems are determined by the Regulating Authority for a given event.
This system is primarily played under the auspices of the English Bridge Union
and is allowed at Level 4 \orange{9B}, which is the standard tournament level
in England. Events not directly run by the EBU may have other restrictions.
References to the Orange Book sections allowing each convention at Level 4 are
given throughout the document.

\subsubsection{WBF classification}

Under the WBF classification system this is a Red system with Brown Sticker
Conventions. Thus, it is not legal in Category 3 WBF events and in Category 2
events the card with suggested defences and appropriate forms must be submitted
in advance. Suggested defences are given in Appendix \ref{appx:defences}.

\newpage

\section{System description}
\label{sec:system}

\subsection{Philosophy}
\label{sec:philosophy}

\begin{itemize}
\item lots of multi-way bids which show very distinct hands
\item establish strength first
\item establish shape (balanced/unbalanced) second
\item then find a contract
\item single-suited hands are all 6+ cards and deny a side 4-card suit, although
this may be bent with 4 small cards in a minor.
\end{itemize}

\subsection{Hand evaluation}

The descriptions below of particular bids will generally be in terms of point
ranges. These are not a hard and fast rule, however. We will try and make the
bid which describes the hand best. This may mean small deviations of strength, 
particularly with extra shape. No-trump ranges are the least likely to vary and
preemptive actions most likely.

Opening bids at the 1 level are 10--15. The lower limit is usually governed by
the rule of 19 as much as the actual point count and some good 15 point hands
may be upgraded to a 1$\diamondsuit$ opening. Losing trick count is also 
useful for deciding when to upgrade. Upgraded hands will, obviously, meet the
extended rule of 25 in all cases. \orange{10B5}

In order to determine whether a hand is game forcing (for example to open a
distributional hand 1$\diamondsuit$ or at the two level) we generally
rely on losing trick count. Three or fewer losing tricks is enough to be game
forcing, occasionally 4 if there are also 9 clear-cut tricks. This is just a
guide, however, so might not strictly apply to all hands. The hands will meet
the extended rule of 25 in all cases.

\newpage

\subsection{Openings}
\label{sec:openings}

\subsubsection{1 level}
\label{sec:open:1level}

Responses in Section \ref{sec:resp:1level}. When playing in an event with fewer
than seven boards in a round, strike out the differences based on position and
vulnerability; the vulnerable meanings apply at all positions and
vulnerabilities.

\paragraph{1$\clubsuit$}

One of the following \orange{11C11}:

\begin{itemize}
\item (not 3\textsuperscript{rd} non-vul, events with 7+ boards/round) 12--15 balanced (4333, 4432, 5332). May occasionally include some semi-balanced hands esp. 5422.
\item 10--15, not balanced, no 5-card major and at most one 4-card major
\end{itemize}

\paragraph{1$\diamondsuit$}

16+, not game forcing \orange{11C12}. If balanced or semi-balanced then 16--23. If a two suiter (5422 or longer) or three suiter (4441 / 5440) then 16--GF. If a single suiter (6332 or longer) then 16--GF. (16--GF here is 'less than a 3-loser hand').

\paragraph{1$\heartsuit$}

Varies depending upon position and vulnerability, but always 10--15 points, not
balanced and promises 4 spades \orange{11C14}.

First, second or third seat non-vulnerable (events with 7+ boards per round only):
\begin{itemize}
\item 4+ spades, may be longer hearts but will always have 5 spades or 4 hearts
\end{itemize}

All other positions and vulnerabilities:
\begin{itemize}
\item 5+ spades, 0-3 hearts
\end{itemize}

\newpage

\paragraph{1$\spadesuit$}

10--15 points, not balanced. 5+ hearts, 0-3 spades \orange{11C14}

\paragraph{1NT}

Varies depending upon position, vulnerability and number of boards per round

First or second seat non-vulnerable:
\begin{itemize}
\item 9--11 balanced \orange{11F3}.
\end{itemize}

Third seat non-vulnerable:
\begin{itemize}
\item 9--15 balanced \orange{11F3}.
\end{itemize}

First or second seat vulnerable (1--6 boards/round):
\begin{itemize}
\item 14--16 balanced \orange{11F3}.
\end{itemize}

All other positions and vulnerabilities (1--6 boards/round):
\begin{itemize}
\item 12--15 balanced \orange{11F3}.
\end{itemize}

All other positions and vulnerabilities (7+ boards/round):
\begin{itemize}
\item 10--15 points, not balanced, 4+ hearts and 4+ spades \orange{11F5}.
\end{itemize}

For responses see Sections~\xref{sec:resp:1nnat} and ~\xref{sec:resp:1nart}.

\newpage 

\subsubsection{2 level}
\label{sec:open:2level}

Responses in Section \ref{sec:resp:2level}.

\paragraph{2$\clubsuit$}
One of the following \orange{11G10}:
\begin{itemize}
\item Weak (5-9) with $\diamondsuit$ (usually 6 cards)
\item Weak (5-9) with $\heartsuit$ \& $\spadesuit$, 4--4 or better (usually 5--4)
\item Game forcing (3 losers) with $\heartsuit$ \& $\spadesuit$, 4--4 or better (usually 5--4)
\end{itemize}

\paragraph{2$\diamondsuit$}
One of the following \orange{11G10}:
\begin{itemize}
\item Weak (5-9) with $\heartsuit$ (usually 6 cards)
\item Weak (5-9) with $\spadesuit$ \& $\clubsuit$, 5--4 or better 
\item Game forcing (3 losers) with $\spadesuit$ \& $\clubsuit$, 4--4 or better  (usually 5--4)
\end{itemize}

\paragraph{2$\heartsuit$}
One of the following \orange{11G10}:
\begin{itemize}
\item Weak (5-9) with $\spadesuit$ (usually 6 cards)
\item Weak (5-9) with $\clubsuit$ \& $\diamondsuit$, 4--4 or better (usually 5--4)
\item Game forcing (3 losers) with $\clubsuit$ \& $\diamondsuit$, 4--4 or better (usually 5--4)
\end{itemize}

\paragraph{2$\spadesuit$}
One of the following \orange{11G10}:
\begin{itemize}
\item Weak (5-9) with $\clubsuit$ (usually 7 cards)
\item Weak (5-9) with $\heartsuit$ \& one of the minors, 4--4 or better (usually 5--4)
\item Game forcing (3 losers) with $\heartsuit$ \& one of the minors, 4--4 or better (usually 5--4)
\end{itemize}

\paragraph{2NT}
One of the following \orange{11H8}:
\begin{itemize}
\item Weak (5-9) with $\spadesuit$ \& $\diamondsuit$, 4--4 or better (usually 5--4)
\item Game forcing (3 losers) with $\spadesuit$ \& $\diamondsuit$, 4--4 or better (usually 5--4)
\item 24+ (Game forcing) balanced hands
\end{itemize}

\subsubsection{Higher openings}
\label{sec:open:higher}

Responses in Section \ref{sec:resp:higher}.

\paragraph{3 level}

Three level bids are transfers. Either weak preempts or a 3-loser hand.
Typically seven cards, they may be occasionally a good six (particularly at
favourable vulnerability) or a bad eight.

\begin{itemize}
\item 3$\clubsuit$ - 7 card preempt or GF in diamonds \orange{11J8(a)}
\item 3$\diamondsuit$ - 7 card preempt or GF in hearts \orange{11J8(a)}
\item 3$\heartsuit$ - 7 card preempt or GF in spades \orange{11J8(a)}
\end{itemize}

\paragraph{3$\spadesuit$}

Gambling 3NT~\orange{11J8(d)}. 

Shows a solid seven or eight card minor suit and nothing outside better than a
queen.  Solid in this case is defined as 60\% likely to run given all
distributions of cards in the other hands. The following suits are all
considered to be solid:

\begin{itemize}
\item AKQJxxx
\item AKQxxxx
\item AKxxxxxx
\end{itemize}

\paragraph{3NT}

Shows a 4-level preempt in either minor or a 3 loser hand with
clubs~\orange{11K3(d)}. Usually eight cards. 

\paragraph{4-level suit openings}

We play Namyats, so:

\begin{itemize}
\item 4$\clubsuit$ - Strong 4-level bid in hearts~\orange{11L3}
\item 4$\diamondsuit$ - Strong 4-level bid in spades~\orange{11L3}
\item 4$\heartsuit$ - Purely preemptive 4-level bid in hearts
\item 4$\spadesuit$ - Purely preemptive 4-level bid in spades
\end{itemize}

Strong is defined as an 8--9 playing trick hand.

\paragraph{4NT}

Asks for specific aces~\orange{11L1}.

Will be bid with a hand missing at most two aces, or with a void. It must be happy playing at the six level opposite some responses and the seven level opposite some others.

\paragraph{5 of a major}

Shows an 11 trick hand missing two of the top three trump honours. Raise one level per trump honour.

\subsection{Responses \& rebids}
\label{sec:responses}

\subsubsection{1 level}
\label{sec:resp:1level}

\begin{itemize}
\label{note:14}
\item 1$\clubsuit$ (3\textsuperscript{rd}/4\textsuperscript{th} Non-Vulnerable):
	\begin{itemize}
   \item 1$\diamondsuit$ = 0--5, no 6-card suit
		\begin{itemize}
      \item suits = unbalanced, possible canape
      \item 1NT (3\textsuperscript{rd}) = 10--15, 5--4 minors
      \item 1NT (4\textsuperscript{th}) = 12--15 balanced; responses weak takeout.
		\end{itemize}
   \item 1$\heartsuit$ = 6--8, no 6-card suit
		\begin{itemize}
      \item suits = unbalanced, possible canape
      \item 1NT (3\textsuperscript{rd}) = 10--15, 5--4 minors
      \item 1NT (4\textsuperscript{th}) = 12--15 balanced; responses weak takeout.
		\end{itemize}
   \item 1$\spadesuit$/1N/2x = two-under transfer weak jump shift 6+ card suit two-above
		\begin{itemize}
		\item Complete the transfer with nothing more to say
		\item intervening bid = extras, 4+ support
		\end{itemize}
	\end{itemize}

\newpage

\item 1$\clubsuit$ (3\textsuperscript{rd}/4\textsuperscript{th} Vulnerable):
	\begin{itemize}
   \item 1$\diamondsuit$ = 0--8, no 6-card suit
		\begin{itemize}
      \item suits = unbalanced, possible canape
      \item 1NT = 12--15 balanced; responses weak takeout.
		\end{itemize}
   \item 1$\heartsuit$ = 9-11
		\begin{itemize}
      \item 1$\spadesuit$ = any 10-11
			\begin{itemize}
			\item 1NT = balanced, no interest in game. Responses are weak takeout.
			\item suits = natural, F1
			\item 2N = balanced, 13-15. Responses see ``Continuations after 2NT" \xref{sec:resp:2n}
			\end{itemize}
      \item 1NT = 12--15 balanced; see ``Continuations after 1NT" \xref{sec:resp:1n}.
      \item major suit bids 12-15 unbal and promise exactly 4; either 4-4-4-1
            with singleton other major or has a longer minor (as there's no other
            hand shape that fits the opening and has exactly one 4-card major)
      \item minor suit bids deny a 4-card major and are 12-15 unbal.
		\end{itemize}
   \item 1$\spadesuit$/1N/2x = two-under transfer weak jump shift 6+ card suit two-above
		\begin{itemize}
		\item Complete the transfer with nothing more to say
		\item intervening bid = extras, 4+ support
		\end{itemize}
	\end{itemize}

\newpage

\item 1$\clubsuit$ (1\textsuperscript{st}/2\textsuperscript{nd} Any vulnerability):
	\begin{itemize}
   \item 1$\diamondsuit$ = 0--9, no 6-card suit
		\begin{itemize}
      \item suits = unbalanced, possible canape
      \item 1NT = 12--15 balanced; responses weak takeout.
		\end{itemize}
   \item 1$\heartsuit$ = 10--15 any
		\begin{itemize}
      \item 1$\spadesuit$ = any 10-11
			\begin{itemize}
			\item 1NT = balanced, no interest in game. Responses are weak takeout.
			\item suits = natural, F1
			\item 2N = balanced, 13-15. Responses see ``Continuations after 2NT" \xref{sec:resp:2n}
			\end{itemize}
      \item 1NT = 12--15 balanced; see ``Continuations after 1NT" \xref{sec:resp:1n}.
      \item major suit bids 12-15 unbal and promise exactly 4; either 4-4-4-1
            with singleton other major or has a longer minor (as there's no other
            hand shape that fits the opening and has exactly one 4-card major)
      \item minor suit bids deny a 4-card major and are 12-15 unbal.
		\end{itemize}
   \item 1$\spadesuit$ = 15+ any, game-forcing.
		\begin{itemize}
      \item major suit bids promise exactly 4; either 4-4-4-1 with singleton other
         major or has a longer minor (as there's no other hand shape that fits the
         opening and has exactly one 4-card major)
      \item 1NT = 12--15 balanced; see ``Continuations after 1NT" \xref{sec:resp:1n}.
      \item minor suit bids deny a 4-card major.
		\end{itemize}
   \item 1N/2x = transfer weak jump shift 6+ card suit above, 0--9.
		\begin{itemize}
		\item Complete the transfer with nothing more to say
		\item 2N = extra values, nothing in the transfer suit
		\item Jump-completion = good support, invitational
		\item new suits = extras, tolerance, side suit
		\end{itemize}
	\end{itemize}

\newpage

\item 1$\diamondsuit$:
	\begin{itemize}
   \item 1$\heartsuit$ = 0--7, no 6-card suit.
		\begin{itemize}
		\item Suit bids are natural
      \item 1NT rebid = 16--19 balanced; see ``Continuations after 1NT" \xref{sec:resp:1n}.
      \item Jump suit rebids are almost-GF, semi-solid single-suited.
      \item 2NT rebid = 20--23 balanced; see ``Continuations after 2NT" \xref{sec:resp:2n}.
		\end{itemize}
   \item 1$\spadesuit$ = 8+, game-forcing.
		\begin{itemize}
      \item 1NT rebid = 16--19 balanced; see ``Continuations after 1NT" \xref{sec:resp:1n}.
      \item Simple suit rebids are suction-style, showing either the suit above
         (single-suited) or the other two, 16--21 HCP.
			\begin{itemize}
			\item $\clubsuit$ = diamonds or the majors
			\item $\diamondsuit$ = hearts or spades and a minor
			\item $\heartsuit$ = spades or the minors
			\item $\spadesuit$ = clubs or hearts and a minor
			\end{itemize}
		\item Could be 3-suited, short in the suit above. These are shown: 
			\begin{itemize}
         \item 2$\clubsuit$-2$\diamondsuit$-3$\clubsuit$: short diamonds.
			\item 2$\diamondsuit$-2$\heartsuit$-2$\spadesuit$: short hearts. 
			\item 2$\heartsuit$-2$\spadesuit$-3$\heartsuit$: short spades.
         \item 2$\spadesuit$-3$\clubsuit$-3$\spadesuit$: short clubs.
			\end{itemize}
		\item Continuations:
			\begin{itemize}
			\item Responder will bid the next suit (relay). Opener then rebids 2NT with the
				single-suited hand, a suit with the two/three-suited option or 3NT with
				a 4/5 loser hand with a solid single suiter (top-range sub-game-force).
			\item After GF+ suit agreement with a 3-suiter, 3N by opener shows no wish to make
				a slam try. Bidding the short suit shows a void and bidding a long suit shows
				a significantly better suit.
			\item After 1$\diamondsuit$-1x and showing suit(s) by opener, bidding 4$\clubsuit$/$\diamondsuit$
				is Swiss \xref{sec:swiss} agreeing opener's most recently bid suit, bidding one of the other suits below 
				game is slam inviting asking opener to bid swiss for that suit. Other suits below 3NT are an enquiry
				either wanting a stop or to find out openers' longer suit.
			\end{itemize}
      \item Jump suit rebids are almost-GF, semi-solid single-suited.
      \item 2NT rebid = 20--23 balanced; see ``Continuations after 2NT" \xref{sec:resp:2n}.
		\end{itemize}
   \item 1N/2x = transfer weak jump shift 6+ card suit above, 0--7.
		\begin{itemize}
		\item Complete the transfer with nothing more to say
		\item 2N = extra values, nothing in the transfer suit
		\item Jump-completion = good support, invitational
		\item new suits = extras, tolerance, side suit
		\item jump new suit = GF single suiter
		\end{itemize}
	\end{itemize}

\item 1$\heartsuit$ (if denying $\heartsuit$):
	\begin{itemize}
\label{note:4}
   \item 1$\spadesuit$/2$\spadesuit$/3$\spadesuit$/4$\spadesuit$ = to play/limit raises/to play.
\label{note:5}
   \item 1NT = 1-round force.
   \item 2$\clubsuit$/$\diamondsuit$ = natural, forcing.
   \item 2$\heartsuit$ = good raise, neither or both of $\spadesuit$ AK.
		\begin{itemize}
		\item Opener bids long suit trials below the trump suit
		\item Cues or Swiss~\xref{sec:swiss} above the trump suit
		\end{itemize}
   \item 2NT = good raise, with exactly one of $\spadesuit$ AK \xref{sec:encryption}.
		\begin{itemize}
      \item Next bid shows 0/1 loser and the $\spadesuit$ K, or 2/3 losers and the SA.
         Rebidding $\spadesuit$ denies the other top honour.
		\end{itemize}
   \item 3$\clubsuit$/3$\diamondsuit$ = fit jump.
   \item 3$\heartsuit$/4$\clubsuit$/$\diamondsuit$ = splinter agreeing spades
	\item 4$\heartsuit$ = to play
	\end{itemize}

\newpage

\item 1$\heartsuit$ if not denying $\heartsuit$:
	\begin{itemize}
   \item 1$\spadesuit$ = forcing enquiry.
		\begin{itemize}
      \item 1NT = both majors, minimum. Continuations as after artificial 1NT opening \xref{sec:resp:1nart}.
      \item 2$\clubsuit$ = 5+ spades and 4+ clubs, some extras
      \item 2$\diamondsuit$ = 5+ spades and 4+ diamonds, some extras
      \item 2$\heartsuit$ = both majors, maximum
		\item 2$\spadesuit$ = 5+ spades, minimum
		\item 2NT = 6+ spades, no outside suit, maximum
		\end{itemize}
   \item 1NT = non-forcing, any negative without spades
   \item 2$\clubsuit$/$\diamondsuit$ = natural, forcing.
   \item 2$\heartsuit$ = good raise, neither or both of $\spadesuit$ AK.
		\begin{itemize}
		\item Opener bids long suit trials below the trump suit
		\item Cues or Swiss~\xref{sec:swiss} above the trump suit
		\end{itemize}
   \item 2$\spadesuit$/3$\spadesuit$/4$\spadesuit$ = to play/limit raise/to play.
\label{note:1}
   \item 2NT = good raise, with exactly one of $\spadesuit$ AK \xref{sec:encryption}.
		\begin{itemize}
      \item Next bid shows 0/1 loser and the $\spadesuit$ K, or 2/3 losers and the SA.
         Rebidding $\spadesuit$ denies the other top honour.
		\end{itemize}
\label{note:2}
   \item 3$\clubsuit$/3$\diamondsuit$ = fit jump.
\label{note:6}
   \item 3$\heartsuit$/4$\clubsuit$/$\diamondsuit$ = splinter agreeing spades
	\item 4$\heartsuit$ = to play
	\end{itemize}

\newpage

\item 1$\spadesuit$:
	\begin{itemize}
   \item 2$\heartsuit$/3$\heartsuit$/4$\heartsuit$ = to play/limit raise/to play.
   \item 1NT = 1-round force.
   \item 2$\clubsuit$/$\diamondsuit$ = natural, forcing.
   \item 2$\spadesuit$ = good raise, neither or both of $\heartsuit$ AK.
		\begin{itemize}
		\item Opener bids long suit trials below the trump suit
		\item Cues or Swiss~\xref{sec:swiss} above the trump suit
		\end{itemize}
   \item 2NT = good raise, with exactly one of $\heartsuit$ AK \xref{sec:encryption}.
		\begin{itemize}
      \item Next bid shows 0/1 loser and the $\heartsuit$ K, or 2/3 losers and the HA.
         Rebidding $\heartsuit$ denies the other top honour.
		\end{itemize}
   \item 3$\clubsuit$/3$\diamondsuit$ = fit jump.
   \item 3$\spadesuit$/4$\clubsuit$/$\diamondsuit$ = splinter agreeing hearts
   \item 4$\spadesuit$ = to play
	\end{itemize}

\subsubsection{Natural 1NT}
\label{sec:resp:1nnat}

If 1NT is natural not in third-seat non-vulnerable then the responses are
detailed in the section ``Continuations after 1NT" \xref{sec:resp:1n}.

In the case of the third- or fourth-seat non-vulnerable 1NT the responses are
weak takeout.

\subsubsection{Artificial 1NT}
\label{sec:resp:1nart}

\item 1NT if artificial:
	\begin{itemize}
   \item 2$\clubsuit$/$\diamondsuit$: weak, to play.
   \item 2$\heartsuit$/$\spadesuit$: preference, to play.
   \item 2NT: forcing enquiry:
\label{note:7}
		\begin{itemize}
      \item 3$\clubsuit$/$\diamondsuit$ weak, 3$\heartsuit$/$\spadesuit$ strong. 3$\clubsuit$/$\heartsuit$ better $\heartsuit$, 3$\diamondsuit$/$\spadesuit$ better $\spadesuit$.
		\end{itemize}
   \item 3$\clubsuit$/$\diamondsuit$: strong, good suit.
   \item 3$\heartsuit$/$\spadesuit$: natural slam try.
	\end{itemize}

\newpage

\subsubsection{2 level}
\label{sec:resp:2level}
\label{note:18}

\item 2$\clubsuit$:
	\begin{itemize}
   \item 2$\diamondsuit$ = pass/correct.
		\begin{itemize}
      \item 2NT = GF opening.
		\end{itemize}
   \item 2$\heartsuit$/$\spadesuit$ = non-forcing; major-focused.
		\begin{itemize}
		\item Pass = minimum weak with both majors
		\item 3$\heartsuit$ = maximum weak with better $\heartsuit$
		\item 2/3$\spadesuit$ = maximum weak with better $\spadesuit$
		\item 2NT = maximum weak both equal
		\item 3$\diamondsuit$ = weak, diamonds.
		\item 3$\clubsuit$ = GF opening
		\end{itemize}
	\item 2NT = F1, how good are your diamonds?
		\begin{itemize}
		\item 3$\clubsuit$ = GF opening
 		\item 3$\diamondsuit$ = dire, single-suited
		\item 3$\heartsuit$ = 2-suited
		\item 3$\spadesuit$ = decent diamonds
		\item 3NT = top-of-range diamonds
		\end{itemize}
	\item 3$\clubsuit$ = GF enquiry
		\begin{itemize}
		\item 3$\diamondsuit$ = weak, diamonds
		\item 3$\heartsuit$ = weak, majors, better hearts
		\item 3$\spadesuit$ = weak, majors, better spades
		\end{itemize}

	\end{itemize}

\newpage

\item 2$\diamondsuit$:
	\begin{itemize}
   \item 2$\heartsuit$ = pass/correct.
		\begin{itemize}
      \item 2NT = GF opening.
		\end{itemize}
   \item 2$\spadesuit$/3$\clubsuit$ = non-forcing, $\clubsuit$/$\spadesuit$-focused.
		\begin{itemize}
		\item Pass = minimum weak $\clubsuit$/$\spadesuit$
      \item 2/3NT = maximum weak, both equal.
      \item 3/4$\clubsuit$ = maximum weak, better $\clubsuit$.
      \item 3$\diamondsuit$ = GF opening.
      \item 3$\heartsuit$ = weak with $\heartsuit$
      \item 3$\spadesuit$ = maximum weak, better $\spadesuit$.
		\end{itemize}
   \item 2NT = F1, how good are your hearts?
		\begin{itemize}
      \item 3$\clubsuit$ = two-suited.
      \item 3$\diamondsuit$ = GF opening.
      \item 3$\heartsuit$ = dire, single-suited.
      \item 3$\spadesuit$ = decent, single-suited.
      \item 3NT = top-of-range hearts.
		\end{itemize}
	\item 3$\diamondsuit$ = GF enquiry
	\end{itemize}

\newpage

\item 2$\heartsuit$:
	\begin{itemize}
   \item 2$\spadesuit$ = pass/correct.
		\begin{itemize}
      \item 2NT = GF opening.
		\end{itemize}
   \item 3$\clubsuit$/3$\diamondsuit$ = non-forcing, $\clubsuit$/$\diamondsuit$-focused.
		\begin{itemize}
		\item Pass = minimum weak minors
      \item 3/4$\diamondsuit$ = maximum weak, better $\diamondsuit$.
      \item 3$\heartsuit$ = GF opening.
      \item 3$\spadesuit$ = single-suited.
      \item 3NT = maximum weak minors, both equal.
      \item 4$\clubsuit$ = maximum weak, better $\clubsuit$.
		\end{itemize}
   \item 2NT = F1, how good are your spades?
		\begin{itemize}
      \item 3$\clubsuit$ = two-suited.
      \item 3$\diamondsuit$ = dire, single-suited.
      \item 3$\heartsuit$ = GF opening.
      \item 3$\spadesuit$ = decent, single-suited.
      \item 3NT = top-of-range spades.
		\end{itemize}
	\item 3$\heartsuit$ = GF enquiry
	\end{itemize}

\newpage

\item 2$\spadesuit$:
	\begin{itemize}
   \item 2NT = Enquiry
		\begin{itemize}
      \item 3$\clubsuit$ = weak with clubs
      \item 3$\diamondsuit$ = weak with $\heartsuit$\&$\diamondsuit$
      \item 3$\heartsuit$ = weak with $\heartsuit$\&$\clubsuit$
      \item 3$\spadesuit$ = GF, $\heartsuit$\&$\clubsuit$
      \item 3NT = GF, $\heartsuit$\&$\diamondsuit$
		\end{itemize}
   \item 3$\clubsuit$/$\diamondsuit$/$\heartsuit$, etc = pass or correct
   \item 3$\spadesuit$ = constructive but non-forcing with 5/6 spades
	\end{itemize}

\item 2NT:
	\begin{itemize}
   \item 3$\diamondsuit$/3$\spadesuit$ = simple preference
   \item 4+$\diamondsuit$/4+$\spadesuit$ = preemptive raise
		\begin{itemize}
      \item After weak preference/raise, any rebid shows GF hand.
		\end{itemize}
   \item 3$\clubsuit$/3$\heartsuit$ = transfer preference; stronger.
		\begin{itemize}
      \item Opener breaks transfer if GF two-suiter; suit has been set so bid controls.
      \item A 3NT rebid shows 24+ balanced
			\begin{itemize}
         \item 4$\clubsuit$/$\diamondsuit$ = 2 under transfers to $\heartsuit$/$\spadesuit$
         \item 4$\heartsuit$/$\spadesuit$ = Swiss for no trumps \xref{sec:swiss}
         \item 4N = Quantitative
         \item 5N = Quantitative (for 6 or 7)
			\end{itemize}
		\end{itemize}
	\end{itemize}

\subsubsection{Responses to higher opening bids}
\label{sec:resp:higher}

\item 3$\clubsuit$/$\diamondsuit$/$\heartsuit$:
	\begin{itemize}
	\item complete transfer (at any level) = to play
		\begin{itemize}
		\item 3N (if available) = GF option
		\item Other slam tries (if available) = GF option
		\end{itemize}
	\item 3x = shortage ask
		\begin{itemize}
		\item 3N = no shortage, weak option
		\item 4 level-completion = GF option
		\item other suits = 0 or 1, weak option
		\end{itemize}
	\item 3N = to play
		\begin{itemize}
		\item slam tries = GF option
		\end{itemize}
	\item 4$\clubsuit$/$\diamondsuit$ = General Swiss~\xref{sec:swiss}
	\end{itemize}

\newpage

\item 3$\spadesuit$:
	\begin{itemize}
	\item 3N = stops in 3 suits
	\item clubs at any level = pass or correct
	\item 4$\diamondsuit$ = asks for singletons
		\begin{itemize}
		\item 4$\heartsuit$/$\spadesuit$ = singleton in that suit
		\item 4N = no singletons
		\item 5$\clubsuit$/$\diamondsuit$ = singleton in the other minor
		\end{itemize}
	\end{itemize}

\item 3NT:
	\begin{itemize}
	\item clubs at any level = pass or correct
		\begin{itemize}
		\item Pass = weak clubs
		\item Diamonds = weak diamonds
		\item Others = GF clubs
		\end{itemize}
	\item 4$\diamondsuit$ = enquiry
		\begin{itemize}
		\item 4$\heartsuit$ = clubs
		\item 4$\spadesuit$ = diamonds
		\item 4NT = GF clubs
		\end{itemize}
	\end{itemize}

\item 4$\clubsuit$/$\diamondsuit$:
	\begin{itemize}
	\item next-step = slam try with support
		\begin{itemize}
		\item complete transfer = decline
		\item new suits = cues
		\end{itemize}
	\item completion = to play
	\end{itemize}

\item 4NT:
	\begin{itemize}
	\item 5$\clubsuit$ = No aces
	\item 5$\diamondsuit$ = Club ace or Heart, Diamond and Spade aces
	\item 5$\heartsuit$ = Diamond ace or Club, Heart and Spade aces
	\item 5$\spadesuit$ = Heart ace or Club, Diamond and Spade aces
	\item 5N = Two aces not including Spades
	\item 6$\clubsuit$ = Spade ace or Club, Diamond and Heart aces
	\item 6$\diamondsuit$ = Spade and Club aces
	\item 6$\heartsuit$ = Spade and Diamond aces
	\item 6$\spadesuit$ = Spade and Heart aces
	\end{itemize}

\end{itemize}

\newpage

\subsection{Continuations after 1NT}
\label{sec:resp:1n}


This refers only to a natural 1NT, though it may be 9--11, 12--15 or 16--19
depending on route. Maximum is an 11-count, 14--15-count, or 18--19-count
depending on range.

\begin{itemize}

\item 2$\clubsuit$ is 5-card puppet Keri (see below), which allows the system to be used with weak diamond hands with either tolerance for the majors or playing in 3$\diamondsuit$. (e.g. 2=3=6=2 pattern with very limited values).

\label{note:9}

After 2$\clubsuit$ is doubled:

	\begin{itemize}
	\item Pass = no 4 or 5 card major, no diamond fit
	\item XX = a 4 card major
	\item 2$\diamondsuit$ = diamond fit, no 4 or 5 card major
	\item 2$\heartsuit$/$\spadesuit$ = 5 card suit
	\end{itemize}

\item 2$\diamondsuit$ and 2$\heartsuit$ are transfers; transfer breaks apply with any 4-card fit.  2NT
shows 4/5-card support, no side 4-card suit, and a maximum.  Other suits
show 4 cards, and 4-card support with a maximum.  A simple jump acceptance
shows 4-card support and a minimum.

After the transfer is doubled:

	\begin{itemize}
	\item Pass = 2 cards in support, minimum
	\item Redouble = 2-3 cards in support, maximum
	\item Complete = 3 cards in support, minimum
	\item Normal transfer breaks = 4+ cards in support
	\end{itemize}

\item 2$\spadesuit$ is a range enquiry/transfer to $\clubsuit$.  2NT shows a
minimum, after which 3$\clubsuit$ is to play and other bids are GF, first round
cues.  3$\clubsuit$ shows any maximum, and can be passed if 2$\spadesuit$ was
weak takeout.

\item 2NT is a transfer to diamonds; 3$\clubsuit$ is the only available super-accept which 
should be used with any 4 or honour-third.

\item 3$\clubsuit$ promises a long suit that may need some help to run.  Opener should
pass, or bid 3NT with (e.g.) Kxx in the suit.

\item 3$\diamondsuit$ is 5--5 in the majors, at least invitational.

\item 3$\heartsuit$/3$\spadesuit$ are slam tries in clubs and diamonds.

\item 4$\clubsuit$/4$\diamondsuit$ are General Swiss \xref{sec:swiss}.

\item 4N is Blackwood \xref{sec:blackwood}.

\end{itemize}

\paragraph{Responses to 5-card puppet Keri}
\label{note:8a}

Because we may have already established a game-force before bidding 1NT and we
use the principle of fast-arrival some of the ranges vary. In the responses
below, min/max and invitational / GF are inverted if a game force has already
been established, so:

\begin{itemize}
\item min $\rightarrow$ max
\item max $\rightarrow$ min
\item GF $\rightarrow$ denies slam interest
\item invitational $\rightarrow$ slam try
\end{itemize}

Accepting slam-tries should generally be done by bidding
4$\clubsuit$/4$\diamondsuit$ which are General Swiss~\xref{sec:swiss} where
available. Either in a suit if one has been agreed or for no-trumps.


	\begin{itemize}
	\item 2$\heartsuit$/$\spadesuit$ with a 5-card major
		\begin{itemize}
		\item 3$\clubsuit$ to play
		\item 3$\diamondsuit$ to play
		\item other rebids are as after this sequence in 5 card puppet stayman.
		\end{itemize}
\newpage
	\item 2$\diamondsuit$ with all other hands. 
		\begin{itemize}
		\item Pass if weak with diamonds
		\item 2$\heartsuit$ = 4 spades; may have 4 hearts.
			\begin{itemize}
			\item 2$\spadesuit$ = forcing; promises 4 hearts.
				\begin{itemize}
				\item 2NT/3NT = inv/GF, not 4 hearts.
				\item 3$\clubsuit$/3$\diamondsuit$ = 4 hearts, feature in suit bid; GF
				\item 3$\heartsuit$/4$\heartsuit$ = inv/GF, 4 hearts.
				\item 3$\spadesuit$ = GF, 4 hearts, auto-Swiss \xref{sec:swiss}.
				\item 4$\clubsuit$/4$\diamondsuit$ = GF, 4 hearts, Swiss \xref{sec:swiss}.
				\end{itemize}
			\item 2NT/3NT = min/max, denies a 4-card major.
			\item 3$\clubsuit$/3$\diamondsuit$/3$\heartsuit$ = 4 spades, feature in suit bid; max.
			\item 3$\spadesuit$/4$\spadesuit$ = min/max, 4 spades.
			\item 4$\clubsuit$/4$\diamondsuit$ = max, non-serious Swiss \xref{sec:swiss}.
			\end{itemize}

		\item 2$\spadesuit$ = 4 hearts; denies 4 spades.
			\begin{itemize}
			\item 2NT/3NT = min/max, denies 4 $\heartsuit$.
			\item 3$\clubsuit$/3$\diamondsuit$ = 4 hearts, feature in suit bid, max.
			\item 3$\heartsuit$/4$\heartsuit$ = min/max, 4 hearts.
			\item 3$\spadesuit$ = 4 hearts, maximum, auto-Swiss \xref{sec:swiss}.
			\item 4$\clubsuit$/4$\diamondsuit$ = GF, 4 hearts, Swiss \xref{sec:swiss}.
			\end{itemize}

		\item 2NT = 3--3 majors; invitational strength.
		\item 3$\clubsuit$ = to play
		\item 3NT=3--3 majors, GF.
		\end{itemize}
	\end{itemize}

\newpage

\subsubsection{After 1NT is doubled for penalties}
\label{sec:resp:1nx}
\label{note:20}

We play a modified form of suction as the escape after 1NT-X.

\begin{itemize}
\item Pass forces XX, to play or weak with a two-suiter, not the reds or the majors
	\begin{itemize}
	\item XX forced
		\begin{itemize}
		\item Pass with a strong hand, to play
		\item bid 4 card suits up the line
		\end{itemize}
	\end{itemize}
\item XX forces 2$\clubsuit$, weak with clubs or the reds
	\begin{itemize}
	\item 2$\clubsuit$ forced
		\begin{itemize}
		\item Pass with clubs
		\item 2$\diamondsuit$ with $\diamondsuit$\&$\heartsuit$
		\end{itemize}
	\end{itemize}
\item 2$\clubsuit$ puppets 2$\diamondsuit$, weak with diamonds or the majors
	\begin{itemize}
	\item Pass with 5 clubs
	\item 2$\diamondsuit$ with all other hands
		\begin{itemize}
		\item Pass with diamonds
		\item 2$\heartsuit$ with the majors
		\end{itemize}
	\end{itemize}
\item 2$\diamondsuit$ puppets 2$\heartsuit$, weak with hearts
	\begin{itemize}
	\item Pass with 5 diamonds
	\item 2$\heartsuit$ all other hands
	\end{itemize}
\item 2$\heartsuit$ puppets 2$\spadesuit$, weak with spades
	\begin{itemize}
	\item Pass with 5 hearts
	\item 2$\spadesuit$ all other hands
	\end{itemize}
\end{itemize}

This defence is played a level higher in the rare case of 2NT being doubled for
penalties and applies in all cases that a natural notrump bid below game is doubled for
penalties.

\subsubsection{After direct overcalls of 1NT}
\label{sec:resp:lebensohl}
\label{note:13}

Lebensohl:

\begin{itemize}
\item suits at the 2 level = to play
\item suits at the 3 level = GF, natural
\item direct cue = staymanic, denies a stop
\item 3N = natural, denies a stop
\item 2N = puppet to 3$\clubsuit$
	\begin{itemize}
	\item 3$\clubsuit$ forced
		\begin{itemize}
		\item suits below the cue = to play
		\item suits above the cue = invitational, natural
		\item cue = staymanic, promises a stop
		\item 3N = natural, promises a stop
		\end{itemize}
	\end{itemize}
\item Double is for penalties
\item Double of a natural 2$\clubsuit$ overcall is stayman showing a club stop (optional)
\end{itemize}

\newpage

\subsection{Continuations after 2NT}
\label{sec:resp:2n}

\begin{itemize}
\label{note:8}
\item 3$\clubsuit$: advanced 5-card puppet Stayman
	\begin{itemize}
   \item 3$\heartsuit$/$\spadesuit$ shows 5 cards.
   \item 3$\diamondsuit$ promises 4 hearts or 3--4 spades
		\begin{itemize}
      \item 3$\heartsuit$ = 0--3 hearts, 0--4 spades
			\begin{itemize}
         \item 3$\spadesuit$ = 4 spades (Responder bids 4$\spadesuit$ or 3N)
         \item 3NT = 3 spades or 4 hearts
			\end{itemize}
      \item 3$\spadesuit$ = 4 hearts, 0--3 or 5 spades
			\begin{itemize}
         \item 3NT = 3/4 spades (Responder bids 4$\spadesuit$ if 5/4)
         \item 4$\clubsuit$/$\diamondsuit$ = 4 hearts, Swiss \xref{sec:swiss}
         \item 4$\heartsuit$ = to play.
			\end{itemize}
      \item 3N = 4--4 majors
			\begin{itemize}
			\item Pass = with 3 spades
         \item 4$\heartsuit$ = with 4 hearts.
         \item 4$\spadesuit$ = with 4 spades.
			\end{itemize}
		\item 4$\clubsuit$/$\diamondsuit$ = Swiss \xref{sec:swiss} for no-trumps
		\end{itemize}
   \item 3NT denies 4 hearts or 3 spades
	\end{itemize}

\item 3$\diamondsuit$/3$\heartsuit$: transfers.  Superaccept with any 4-card fit; note that 3NT is a
superaccept that still allows the use of Swiss.

\label{note:10}
\item 3$\spadesuit$: minor-suit Stayman.  Asks for a 4- or 5-card minor.
	\begin{itemize}
   \item 3NT = no 4-card minor.
   \item 4$\clubsuit$/4$\diamondsuit$ = 4 cards.
   \item 4$\heartsuit$/4$\spadesuit$ = 5 cards in the corresponding minor.
	\end{itemize}

\item 4$\clubsuit$/$\diamondsuit$: Swiss \xref{sec:swiss}
\item 4N: Quantitative
\end{itemize}

\newpage

\section{Encrypted bidding}
\label{sec:encryption}

Once responder makes an encrypted raise, opener can either accept or decline
the key.  To decline, simply rebid the suit (cheaply with a minimum; jump if
maximum.  Use common sense.).

If the key is accepted, subsequent bids are two-way, depending on the key.

Opener's rebid shows a useful side-suit if holding the trump
K (0--1 losers) or a potential weakness if holding the trump A (2--3 losers).
Responder can then either place the contract, cue-bid (either showing with the
K or denying with the A), or use Blackwood.

Blackwood here is Roman Keycard, but it's modified because there are only three keycards
still of interest.

\section{Slam conventions}
\label{sec:slam}

\subsection{Roman key-card Blackwood (3014)}
\label{sec:rkcb}

4NT with an agreed suit is Roman key-card Blackwood. The key-cards are the four aces
and the king of the agreed suit. Responses are:

\begin{itemize}
\item 5$\clubsuit$: 0 or 3 key cards
\item 5$\diamondsuit$: 1 or 4 key cards
\item 5$\heartsuit$: 2 or 5 key cards without the queen of trumps
\item 5$\spadesuit$: 2 or 5 key cards with the queen of trumps
\end{itemize}

\subsection{Blackwood}
\label{sec:blackwood}

When no suit has/can be agreed, we play a modified form of RKCB which only asks
for the four aces and has the following responses:

\begin{itemize}
\item 5$\clubsuit$: 0 or 3 aces
\item 5$\diamondsuit$: 1 or 4 aces
\item 5$\heartsuit$: 2 aces
\end{itemize}

5$\spadesuit$ is always available as a sign-off in 5N.

\newpage

\subsection{Exclusion Blackwood}
\label{sec:exclusion}

Unwarranted bids of an outside suit at the 5 level are Exclusion Blackwood.
Responses are as whichever form of Blackwood would apply, but ignoring the ace
in the suit bid. The next available suit shows 0 outside aces / keycards and
bids proceed from there.


\subsection{King you have or king you don't}
\label{sec:kyhokyd}

After RKCB, the next available suit which is not the trump suit asks for the
queen of trumps and outside kings. 5N just asks for outside kings. Without the
queen of trumps the respons to the queen-ask is always the trump suit at the
lowest level. Responses to 5N, or if you have the queen of trumps are; with one
outside king, bid that suit. With two outside kings, bid the suit of the king
you are missing. With no kings bid 6 of the trump suit (or 5N if available).

\subsection{First-round-control showing cues}
\label{sec:cues}

Once a suit is agreed and a GF is established, new suits are cues showing an
ace or a void in that suit. Unless otherwise agreed to be something else the
bid of a suit a level above when it would be forcing for a round is a first round
control showing cue agreeing the most recently shown suit.

\newpage

\subsection{General Swiss}
\label{sec:swiss}
\label{note:19}

4$\clubsuit$/4$\diamondsuit$, once a game-force is established and suit agreed, are slam tries, based
on control points.

Control points are 2 for an ace, 1 for a king/singleton, and 1 for the queen of
trumps.  There are 13 available from honours; it is possible to substitute
outside singletons for kings, but care must be taken to avoid double-counting.
11 CPs are necessary for a small slam; 13 for a grand.

\begin{itemize}
\item 4$\clubsuit$: 4 or 6 (exceptionally 8) control points
	\begin{itemize}
   \item trump suit = 0--4 CPs
   \item 4$\diamondsuit$ = 5/6 CP, relay
		\begin{itemize}
      \item trump suit = 4 CPs
      \item suits = 6 CP, lowest king/singleton
		\end{itemize}
   \item suits = 7 or 9 CP, lowest king/singleton
   \item small slam = 8 CP
   \item grand slam = 10 CP
	\end{itemize}

\item 4$\diamondsuit$: 5 or 7 (exceptionally 9) control points
	\begin{itemize}
   \item trump suit = 0--3 CPs
   \item 4$\heartsuit$ = 4/5 CP, relay
		\begin{itemize}
      \item trump suit = 5 CPs
      \item suits = 7 CP, lowest king/singleton
		\end{itemize}
   \item suits = 6 or 8 CP, lowest king/singleton
   \item small slam = 7 CP
   \item grand slam = 9 CP
	\end{itemize}
\end{itemize}

While cueing kings/singletons, 4NT shows a singleton or king which can't be
shown below 5 of the trump suit (usually diamonds over clubs) and asks partner
to signoff appropriately whether they have duplication.

If you find duplication in kings/singletons subtract one CP and sign off at the
appropriate level. If your partner signs off and you have 2 as-yet unshown CPs,
raise one level.

If the hand bidding Swiss has shown a GF (opening a strong 2 level option,
1$\diamondsuit$ and a jump rebid, or 1$\diamondsuit$ and a 2NT rebid), add 2 to
all numbers. If it has shown a weak hand (opening a weak 2 or 3 level option or
giving a 0--7 response to 1$\diamondsuit$ or opening 9-11 1NT), subtract 2 from all numbers.

There is a small problem if you have 7 CPs and are agreeing hearts, since in
that case, there is no available next step. In this case you may wish to show
6 CPs and ignore one point of duplication.

\subsubsection{Interference}

After interference direct over the start of General Swiss:

\begin{itemize}
	\item Pass = would sign-off in game (forcing)
	\item X or XX = next-step
	\item Suits = Accepting the slam try and checking for duplication as normal
\end{itemize}

After interference direct after a non-signoff response to Swiss:

\begin{itemize}
	\item Pass = would sign-off in game (forcing)
	\item X or XX = a second-round control in that suit
	\item Suits = a second-round control in that suit
\end{itemize}

\newpage

\section{Competitive bidding}
\label{sec:competitive}

We tried an artificial overcall scheme but found it couldn't cope with a number of situations.
Therefore, we have a fairly natural overcall style.

\subsection{Natural suits}
\label{sec:def:1x}

Simple overcalls are constructive and natural 5 cards, starting at
10HCP~\orange{11N2}. In response, if 1N is available, then 2N is a crypto
raise~\xref{sec:encryption}. In all cases a responsive cue is a good raise and
direct raises are preemptive.

Double is standard takeout (or a strong hand)~\orange{11N3}.

Direct 1NT is 15-17 balanced with a stop or semi-balanced and may contain a
singleton ace~\orange{11N8}, protective 1NT is 11-14 with a stop.  After these
``Continuations after 1NT''~\xref{sec:resp:1n} apply, with the exception that
after the auction (1x)-1N-(2y), 2x is stayman without a stop if available.
Other bids are lebensohl.

Jump overcalls are weak, showing a 6 card suit and between 4HCP (favourable)
and enough to make a simple overcall. In protective they are intermediate
(11-15, reasonable 6+ card suit). This also applies after the sequence (Pass)-Pass-(1N).

Jump cuebids are stopper-asking for 3NT \orange{11N17}; they promise a long
running suit.  An overcall of 3NT promises a long running suit and a stopper.
Responses are the same as a gambling 3NT opening~\xref{sec:resp:higher}.


Simple cuebids are Michaels showing at least 5--5 in the majors (over a minor)
or the other major and either minor (over a major) \orange{11N7}. In all cases they are
either weak or strong. Similarly, 2NT over a 1 level bid is unusual, showing
5--5 or better in the minors (over a major) or the other minor and either major
(over a minor) \orange{11N9}. Again, it's weak or strong.

After Michaels or Unusual, bidding any of the suits which could have been shown
is pass or correct at that level. Overcaller then corrects to the next highest
suit (or passes or raises) with a weak hand. With a strong hand he breaks to
another suit. Of the two suits which are not 'pass or correct', re-cueing is
strong and agreeing advancer's preferred suit and the other suit is strong with
the remaining option.

\newpage

For example:
\begin{itemize}
\item (1$\heartsuit$)-2$\heartsuit$-(P)
	\begin{itemize}
	\item 3$\clubsuit$-3$\diamondsuit$ is weak with spades and diamonds
	\item 3$\clubsuit$-3$\heartsuit$ is strong with spades and clubs
	\item 3$\clubsuit$-3$\spadesuit$ is strong with spades and diamonds
	\item 3$\diamondsuit$-3$\heartsuit$ is strong with spades and diamonds
	\item 3$\diamondsuit$-3$\spadesuit$ is weak with spades and clubs
	\item 3$\diamondsuit$-4$\clubsuit$ is strong with spades and clubs
	\item 2$\spadesuit$-3$\clubsuit$ is strong with spades and clubs
	\item 2$\spadesuit$-3$\diamondsuit$ is strong with spades and diamonds
	\end{itemize}
\end{itemize}

If advancer has a strong hand then the following response structures apply:

Any suit which overcaller could have at any level is to play opposite a weak
hand and pass or correct. If responder has passed and overcalled has two known
suits, then other two suits at the 3 level are strong raises in one of
overcaller's suits; the cheaper outside suit for the cheaper known suit. These
are invitational-plus if it is possible to stop below game in that suit and
game forcing otherwise. 2NT and 3NT are both natural. The special case of
(1$\clubsuit$)-2$\clubsuit$-(P)-2$\diamondsuit$ asks for the better of
overcall's majors.

After an invitational raise, overcaller bids the agreed suit at the 3 level
with a minimum weak hand and bids game with a good weak hand. With a strong
hand he bids swiss in that suit or does something else.

\newpage

For example:

\begin{itemize}
\item (1$\clubsuit$)-2$\clubsuit$-(P)
	\begin{itemize}
	\item 2$\diamondsuit$ - which is your better major
	\item 2$\heartsuit$ - to play
	\item 2$\spadesuit$ - to play
	\item 2NT - natural, invitational
	\item 3$\clubsuit$ - invitational with hearts
	\item 3$\diamondsuit$ - invitational with spades
	\item 3$\heartsuit$ - to play
	\item 3$\spadesuit$ - to play
	\item 3NT - to play
	\end{itemize}
\item (1$\heartsuit$)-2NT-(P)
	\begin{itemize}
	\item 3$\clubsuit$ - to play
	\item 3$\diamondsuit$ - to play
	\item 3$\heartsuit$ - invitational with clubs
	\item 3$\spadesuit$ - invitational with diamonds
	\item 3NT - to play
	\end{itemize}
\end{itemize}

In the case where the two suits are not specified (responder still having
		passed), then any suit which overcaller could have is pass or correct.
2NT, if available (i.e. after Michaels), is an invitational-plus enquiry as to
the other suit and re-cueing is a strong raise in the suit which has been
promised. A direct 4$\clubsuit$/4$\diamondsuit$ if the promised suit is a major
is Swiss~\xref{sec:swiss} for that suit and 3NT is to play. If 2NT is not
available (i.e. after Unusual) then the re-cue is any stronger hand and
responses are which other outside suit overcaller has.

\newpage

For example:

\begin{itemize}
\item (1$\heartsuit$)-2$\heartsuit$-(P)
	\begin{itemize}
	\item 2$\spadesuit$ - to play
	\item 2NT - enquiry, invitational
		\begin{itemize}
		\item 3$\clubsuit$ - weak clubs and spades (responses = Swiss for clubs)
		\item 3$\diamondsuit$ - weak diamonds and spades (responses = Swiss for diamonds)
		\item 3$\heartsuit$ - strong clubs and spades
		\item 3$\spadesuit$ - strong diamonds and spades
		\end{itemize}
	\item 3$\clubsuit$ - pass or correct
	\item 3$\diamondsuit$ - pass or correct
	\item 3$\heartsuit$ - invitational in spades
	\item 3$\spadesuit$ - to play
	\item 4$\clubsuit$ - Swiss for Spades
	\item 4$\diamondsuit$ - Swiss for Spades
	\item 3NT - to play
	\end{itemize}
\item (1$\clubsuit$)-2NT-(P)
	\begin{itemize}
	\item 3$\clubsuit$ - any invitational
		\begin{itemize}
		\item 3$\heartsuit$ - hearts and diamonds
		\item 3$\spadesuit$ - spades and diamonds
		\end{itemize}
	\item 3$\diamondsuit$ - to play
	\item 3$\heartsuit$ - pass or correct
	\item 3$\spadesuit$ - pass or correct
	\item 3NT - to play
	\end{itemize}
\end{itemize}

If responder makes a raise of opener's suit (so, the sequences (1$\clubsuit$)-2$\clubsuit$-(3$\clubsuit$)
or (1$\heartsuit$)-2NT-(3$\clubsuit$)) the structure is as follows: Double shows a strong raise of
overcaller's known suit (if only one) or cheaper known suit (if two). If an outside suit other than the re-cue
is available then that is a strong raise in the remaining suit(s) and the re-cue is a strong raise with no 
preference, otherwise the re-cue is a strong raise in the remaining suit(s).

\newpage

For example:

\begin{itemize}
\item (1$\clubsuit$)-2$\clubsuit$-(3$\clubsuit$)
	\begin{itemize}
	\item X - invitational hearts
	\item 3$\diamondsuit$ - invitational spades
	\item 3NT - to play
	\item 4$\clubsuit$ - GF, no preference
	\end{itemize}
\item (1$\clubsuit$)-2NT-(3$\clubsuit$)
	\begin{itemize}
	\item X - invitational diamonds
	\item 3$\diamondsuit$ - to play
	\item 3$\heartsuit$ - pass or correct
	\item 3$\spadesuit$ - pass or correct
	\item 3NT - to play
	\item 4$\clubsuit$ - GF in either major
	\end{itemize}
\end{itemize}

\newpage

\subsection{Defences}
\label{sec:defences}

Aside from defences to natural suit openings, these are mostly all based on suction:

\begin{itemize}
\item clubs: diamonds or the majors
\item diamonds: hearts or spades and clubs (or diamonds)
\item hearts: spades or the minors
\item spades: clubs or hearts and diamonds (or clubs)
\end{itemize}

The responses to these are not the same as the responses to a 2-level opening,
but are more standard `pass or correct' responses. For example, after
(1N)-2$\diamondsuit$-(P), 2$\spadesuit$ says pass with the blacks, invitational
in hearts.

\subsubsection{Natural 1NT}
\label{sec:def:1n}

Suction, constructive values. Double is for penalties and 2N shows a
non-touching two suiter. \orange{11P7}

\subsubsection{Artificial strong bids}
\label{sec:def:strong}

Suction, weak. Double shows the suction bid in their suit and 2N shows a
non-touching two suiter. \orange{11M2}

\subsubsection{Short suits}
\label{sec:def:short}

If it `could be as short as 2', then we treat it as a natural opening and the
defences in Section~\ref{sec:def:1x} apply.

\subsubsection{Phoney/Prepared suits}
\label{sec:def:phoney}

Suction, constructive values. Double shows the suction bid in their suit and 2N
is natural. \orange{11M2}

\newpage 

\subsubsection{Multi 2$\diamondsuit$}
\label{sec:def:multi}

\label{note:15}
Dixon \orange{11Q2}:

\begin{itemize}
\item X: 16+
\item 2$\heartsuit$/$\spadesuit$: takeout of the other major
\item 2N: 16-18 bal, stops in both majors.
	\begin{itemize}
	\item Responses as ``Continuations after 2NT" \xref{sec:resp:2n}
	\end{itemize}
\item 3x: intermediate, natural
\end{itemize}

\subsubsection{Natural weak openings}
\label{sec:def:weak}

Double is for takeout up to 4$\heartsuit$. After a weak two and a takeout double, there
are Lebensohl responses \xref{sec:resp:lebensohl}. 4N is takeout over 4$\spadesuit$ and a big two suiter otherwise.

Over a weak two,  2NT is natural showing 16-18 and ``Continuations after 2NT'' \xref{sec:resp:2n}
apply.

\subsubsection{Natural stronger jump openings}
\label{sec:def:inter-strong}

For intermediate natural 2-level openings the overcalls are as natural 1-level
openings~\xref{sec:def:1x}, with 2N as modified unusual. All actions need to be
sounder than over a 1-level opening.

Strong (forcing or not) natural 2-level openings all overcalls are weak and natural.

\subsubsection{Transfer openings}
\label{sec:def:transfer}

Including 2$\clubsuit$ which is a weak 2 in diamonds or a strong hand.

\begin{itemize}
\item Double shows the suit bid
\item Bidding the transferred to suit is takeout of that suit
\item Jump-bidding the transferred to suit is Michaels
\item Other bids and continuations are as normal for that level and strength of bid
\end{itemize}

\subsubsection{Shortage preempts}

If the suit bid is short then double shows an overcall in that suit or any
large hand.  Other suits are natural but reasonably robust and 2NT shows stops
in all the other suits.

If the suit bid is one of the three promised then double is takeout or any large hand.
Suit bids are natural but reasonably robust. 2NT shows a balanced hand with at 
least a stop in the suit bid and some cards in the others.

\subsubsection{Other multi-way preempts}

Multi-way preempts will deny the suit bid and so double shows an overcall in
that suit. Big hands either pass then act or start with a double.

Other overcalls are natural.

\subsubsection{Definite two-suited preempts}

Such as 2NT showing both minors. Double shows the other two suits and a direct
cue bid asks for a stop in that suit, promises stops for any lower suits, but
does not show or deny stops in any higher suits. With a stop partner bids 3N if
he has a stop in all remaining higher suits, or the next suit he does not have
a stop in.

\subsubsection{Doubles}
\label{sec:def:doubles}

Doubles of freely bid slams are Lightner, asking for the lead of dummy's first shown suit.

Doubles of other artificial bids are lead directing. If the doubler has already shown that
suit then it asks for the lead of a different suit.

\newpage

\subsection{Dealing with interference}
\label{sec:interference}

\subsubsection{Direct interference over 1$\clubsuit$}
\label{sec:intf:1c}

\begin{itemize}
\item 1$\clubsuit$-(X) (penalty interest (showing 16+):
	\begin{itemize}
   \item Redouble creates a game-force, and indicates that the double may not have been 
      a winning action.
   \item Pass forces redouble; either to play or with a two-suiter not including clubs.
   \item 1$\diamondsuit$/1$\heartsuit$ are transfers to 1$\heartsuit$/1$\spadesuit$.  Note that opener may pass these if holding a 5-card 
      suit.
	\end{itemize}
\item 1$\clubsuit$-(X) (showing $\clubsuit$):
	\begin{itemize}
   \item Redouble creates a game-force; passes are now forcing and doubles are for penalty.
   \item Pass shows a weak hand, with nowhere in particular to play.
   \item Suit bids are natural, 10--15, and promise 4 cards.
   \item 1NT shows a balanced hand, 10--15, with a club stop.
	\end{itemize}
\item 1$\clubsuit$-(X) (takeout):
	\begin{itemize}
   \item Redouble still creates a forcing pass situation.
   \item Ask opponents ``takeout of what, exactly?".
   \item 1NT cannot show a stopper in any particular suit.
	\end{itemize}
\item 1$\clubsuit$-(natural bid):
	\begin{itemize}
   \item Responder's double creates a game-force.
   \item 1NT promises a stop; 10--15 balanced. See ``Responses after 1NT'' \xref{sec:resp:1n}
   \item Suit bids are natural, 10--15. After a NT rebid by opener, 3$\clubsuit$ is checkback and normal system is off. 
   \item Exception: cheapest bid (or 2$\spadesuit$ over 1$\spadesuit$) shows a 12-15 balanced hand with no stop.  System is on, including Keri, transfers etc.  After 1$\clubsuit$-(1$\diamondsuit$)-1$\heartsuit$, for example, 1$\spadesuit$ is a range enquiry or transfer to $\clubsuit$. See \xref{sec:resp:1n} or \xref{sec:resp:2n}.
   \item Pass is a weak hand with nowhere to go (potentially including an 8--11 balanced hand with no stop).
	\end{itemize}
\end{itemize}

\subsubsection{Direct interference over 1$\diamondsuit$}
\label{sec:intf:1d}

\begin{itemize}
\item 1$\diamondsuit$-(X):
	\begin{itemize}
   \item Redouble with good 8+ hands.
   \item Pass with weak hands.
   \item Suit bids are natural, weak, 6-card suit.
	\end{itemize}
\item 1$\diamondsuit$-(bid):
	\begin{itemize}
   \item Pass with weak hands.
   \item Double to establish a game-force (and forcing passes, etc.).
		\begin{itemize}
      \item Suction rebids are on; immediate NT bid denies a stop.
      \item To show a stop, Suction-force into opponents' suit then bid NT.
		\end{itemize}
   \item Suit bids are natural, weak, 6-card suit.
	\end{itemize}
\item 1$\diamondsuit$-(spade for a laugh):
	\begin{itemize}
   \item Double is for penalties
   \item Passes are forcing
	\end{itemize}
\end{itemize}

\subsubsection{Balancing-seat interference (e.g. 1$\diamondsuit$-(P)-1$\spadesuit$-(2$\clubsuit$))}
\label{sec:intf:bal}

\begin{itemize}
\item 1$\clubsuit$-(P)-1$\spadesuit$-(anything):
\item 1$\diamondsuit$-(P)-1$\spadesuit$-(anything):
\begin{itemize}
   \item Passes are forcing
   \item NT bids promise a stopper.
   \item Suction-style rebids are on if 3$\clubsuit$ is a sufficient bid (1$\diamondsuit$ opener)
   \item Pass indicates a balanced hand without a stopper in opponents' suit.
	\begin{itemize}
      \item Responder's NT bids here are Lebensohl style \xref{sec:resp:lebensohl}
      \item There are no weak hands to show, so the only difference is in showing/denying a stopper.
	\end{itemize}
   \item Double is penalties.
\end{itemize}

\item 1$\clubsuit$-(P)-1$\diamondsuit$-(anything): it's a good idea to subside quietly, however, double is takeout below 2NT, penalties above.

\item 1$\clubsuit$/$\diamondsuit$-(P)-1$\heartsuit$-(anything): passes are non-forcing, X is takeout / suction

\item 1$\clubsuit$-(P)-1$\heartsuit$-(1$\spadesuit$)-P: 12-15 balanced without a spade stop.

Responses:
	\begin{itemize}
	\item 1NT = Lebensohl
		\begin{itemize}
		\item 2x = invitational
		\item 2$\spadesuit$ = stayman with a stop
		\end{itemize}
	\item 2x = weak
	\item 2$\spadesuit$ = stayman without a stop
	\item 2NT = invitational with a stop
	\item 3x = forcing
	\item 3NT = to play, shows a stop
	\end{itemize}

\item 1NT (majors)-(anything): double is for penalties, cue bid is a forcing enquiry

\end{itemize}

\newpage
\subsubsection{Interference over a 2-level opening}
\label{sec:intf:2level}

A few general principles here:

\begin{itemize}
\item 2x-(X):
   \begin{itemize}
      \item XX = forcing enquiry (even over further interference, starts a penalty auction)
      \item Pass = no desire to play opposite a weak hand
      \item Relay = some tolerence for the weak options
      \item Enquiries = forcing unless oppo bid
   \end{itemize}
\item 2x-(2y):
   \begin{itemize}
      \item X = optional (pass if it's your suit, pull otherwise, starts a penalty auction)
      \item \ldots
   \end{itemize}
\item Opener's rebids facing passed responder:
   \begin{itemize}
   \item Suit from a weak option at the lowest level = decent weak option
   \item X/denied suit/NT = strong option (X is optional)
   \end{itemize}
\item 2x-(p)-relay-(X):
   \begin{itemize}
	\item Pass = weak single suiter
	\item XX = weak 2 suiter
	\item 2N = GF two suiter
   \end{itemize}
\item 2N-(x):
   \begin{itemize}
	\item Pass = no preference, weak
	\item XX = no preference, strong
	\item others = as normal
   \end{itemize}
\end{itemize}

\subsection{Forcing passes}
\label{sec:forcepass}

There are a number of situations where a pass can be forcing in the sense of
``either we bid at least once more or defend doubled". The following situations
are definitely forcing-pass situations:

\begin{itemize}
\item After a freely bid game
\item After 1NT-X and they escape at the two level
\item ...
\end{itemize}

\newpage

\section{Carding}
\label{sec:carding}

\subsection{Leads}
\label{sec:card:leads}

Leads are standard (2nd from bad suits, 4th from an honour, top of sequences),
except that they may vary based on signal requested (see below).

The exception is that we lead top of bad suits against no trumps.

After the initial lead we lead high from bad suits and low from good suits.

\subsection{Signals}
\label{sec:card:signals}

We play two forms of signalling. On partner's lead we play reverse attitude
(low encouraging, high discouraging), except if the initial lead was an odd
card. In that case we play Prism signals in that suit (see below). We also play
Prism signals when trumps are led by either side, or if there is a suitable
obvious running suit in no trumps.

Leads to ruffs and other obvious situations may be McKenny.

\subsection{Discards}
\label{sec:card:discards}

Discards are Italian style, so an odd card is encouraging in the suit discarded
and an even card is McKenny. Thus, low even cards ask for the lower of the
outside suits and high even cards ask for the higher of the two outside
suits.

\newpage
\subsection{Prism}
\label{sec:card:prism}

Prism is a signalling system based on the shape of the whole hand. An initial
hand of 13 cards will, perforce, have either one suit which is odd (and three
even) or one suit which is even (and three odd). This denotes the parity of the
hand.

The first card in a prism sequence will give the original parity of the hand.
Low for odd, high for even. Second and subsequent cards will indicate the suit
which is unique in parity. Take the spade suit 2 3 6, the following sequences indicate
the following parities and suits:

\begin{tabular}{llll}
\bf order & \bf parity & \bf outside suit & \bf relative to spades \\
\hline
2 3 6 & odd    & higher       & hearts \\
2 6 3 & odd    & middle       & diamonds \\
3 6 2 & odd    & lower        & clubs \\
3 2 6 & even   & higher       & hearts \\
6 2 3 & even   & middle       & diamonds \\
6 3 2 & even   & lower        & clubs \\
\end{tabular}

This information can be combined with the other defender's hand and dummy to
deduce the shape of declarer's hand as follows. Add together the length of your
suits and dummy's, then subtract each from 13. This will give you either four
even numbers, four odd numbers or two of each.

\begin{itemize}
\item If all four are even then declarer's unique suit is the same as partners and the same parity.
\item If all four are odd then declarer's unique suit is the same as partners and the other parity.
\item If partner's suit is in the even pair, then declarer's suit is the other one in that pair and the other parity.
\item If partner's suit is in the odd pair, then declarer's suit is the other one in that pair and the same parity.
\end{itemize}

\newpage

\section{A brief history}
\label{sec:history}

{\bf NB: the system has changed sufficiently that this history is out-of-date.  Do
not rely on it to describe our bidding! }

We wanted to play a system making use of encrypted bidding.  Our first attempt
used a fairly standard Acol structure, with a weak NT and three weak 2s.  It
was soon apparent that crypto raises in these circumstances (requiring 11+,
4-card support and either A or K but not both) never come up, at least to a
first approximation.

We also quite liked the idea of light and limited openings.  This was from a
couple of experiments with a (highly illegal) forcing pass system, where a pass
in first or second seat showed any 12+, with 0--11 HCP hands forced to open
somehow.  In particular, 2-suited bids at the 1-level have a lot of
attractions.

Our next attempt was to consider the idea of a 12+ 1$\clubsuit$ opener, with all other
openings 8--11.  This is fine in theory, but EBU regulations say that this 1$\clubsuit$
opener cannot have a 5+ card major suit without a longer minor.  This meant
that we needed an opening-values 2$\clubsuit$ bid, and generally lost the point of the
system.  We fell back upon a strong club structure.

Given the strong $\clubsuit$ structure, we then looked at our other opening bids.  Our
2-level openers are pre-emptive, and show either single-suiters or two-suiters.
This is based upon the Suction convention: 2$\clubsuit$ shows $\diamondsuit$ or both majors, 2$\diamondsuit$ shows
$\heartsuit$ or $\spadesuit$\&$\clubsuit$, 2$\heartsuit$ shows $\spadesuit$ or $\clubsuit$\&$\diamondsuit$, and 2$\spadesuit$ shows $\heartsuit$ and a minor.  We don't have a way
to show a $\spadesuit$\&$\diamondsuit$ two-suiter; this is a likely future use for 2NT.

Major suit openings are played as takeout bids.  This means that 1$\spadesuit$ promises 5+
H and 0--3 $\spadesuit$, with 1$\heartsuit$ promising 5+ $\spadesuit$ and 0--3 $\heartsuit$.  These were, at one stage,
4-card openings in the other major, but we found that this made bidding in
competition too difficult.  In uncontested auctions, this has the great
advantage that you gain a ``free" cue-bid: after a 1$\spadesuit$ opening, 2$\spadesuit$ shows a good
raise of $\heartsuit$ while 2$\heartsuit$, 3$\heartsuit$ etc. are pre-emptive.

The next implication is that we need a way to show hands with 4+ in both
majors.  This falls to 1NT; we have 2$\clubsuit$/2$\diamondsuit$/2$\heartsuit$/2$\spadesuit$ as weak bids ``to play", and 2NT
as a strong enquiry.  Bids in the major suits are primarily pre-emptive; 2NT
asks for range and better major.  3m is strong and one-round forcing.

1$\clubsuit$ includes balanced hands 16--19 and 23+ (we're using 2NT as a 20--22 balanced
hand, for now).  The 1$\diamondsuit$ opening, therefore, has to cover all opening hands that
do not fall into one of the above categories.  It therefore includes:

\begin{itemize}
\item 12--15 balanced.
\item Unbalanced hands, 8--15, with no 5-card major and at most one 4-card
      major.
\end{itemize}

Responses to 1$\diamondsuit$ are selected to allow the 1NT rebid, showing a weak NT opener,
on all auctions.  Therefore: 1$\heartsuit$ = 0--7, 1$\spadesuit$ = 8+.  After interference, pass is
weak; any other action promises 8+.

So that's our one-level openings.  For two-level openers, we wanted them to be
the two-suited bids we'd originally considered for our one-level openings.  I
can't remember if it was Matt or me who first found out about Suction as a
defence to 1NT, or who suggested it as a structure for 2-level openings, but
the fact remains that we like it.  Essentially, each 2-level bid shows either a
weak 2 in the suit above OR the other two suits.  Of course, this leaves the
non-touching pairs uncovered, and leaves you with two ways to show a pre-empt
in clubs.  We soon thought of playing 2$\spadesuit$ as showing $\heartsuit$ and
a minor (as opposed to the original $\clubsuit$ or
		$\heartsuit$\&$\diamondsuit$).  This still left
$\spadesuit$\&$\diamondsuit$ uncovered, but we decided to cope.

At this stage, our 2NT opening was fairly standard: 20--22 balanced or
semi-balanced.  After a few sessions playing, we modified this to include the
spade-diamond two-suiters.  Of course, that makes the bid forcing, so we
changed the NT ranges such that this included our game-forcing NT hands: any
23+.  Responses are suit preference, on the assumption that the weak hand is
more frequent, with transfer preference being used if responder has a strong
hand.  If opener has the strong option, he can rebid 3NT.

The next change (there's still more!) was to include GF two-suiters in our
2-level openings.  This narrows the specification of the 1$\clubsuit$ opening, and is
easy to specify: opener can rebid his opened suit (whichever it is) over any
response.  In the 2NT case, this opening now shows GF (semi-)balanced, GF with
S\&$\diamondsuit$, or weak with $\spadesuit$\&$\diamondsuit$.  In the two-suited case, any rebid other than 3NT shows
the strong hand, and any transfer break does also.

Finally, our responses to 1$\clubsuit$: initially, these were natural, promising a 4-card
suit and 8+ (with any weak hands going via 1$\diamondsuit$).  Boring!  We then tried
encrypted responses: 1$\heartsuit$ promised HA or HK, not both; 1$\spadesuit$ showed neither or both
in $\heartsuit$, and exactly one in $\spadesuit$, and so on.  1NT was a positive with no potential
key; 1$\diamondsuit$ was the negative.  Opener would rebid NT if balanced, 2$\heartsuit$ to deny a
potential key, and anything else to show a potential place to play if holding
the key K or denying if holding the key A.  We ditched this structure not
because it doesn't work but because it frequently (read nearly always) ends up
at the 3-level before either hand has actually mentioned a suit, which could
still be only 4 cards.

We now play a far more elegant structure, similar to our responses to 1$\diamondsuit$: 1$\diamondsuit$ is
negative (0--7), 1$\heartsuit$ is positive distributional (8+), and 1$\spadesuit$ is positive balanced
(8+, again).

As for the system name: suction is measured in Pascals, and the system is
encrypted.  It went from there.

\newevenside

% appendices below here
\def\thesection{\Alph{section}}
\setcounter{section}{0}

\small
\section{Prepared Defences}
\label{appx:defences}

Please note that the EBU does not recommend providing prepared defences to
systems.  Nonetheless, if you would find these useful here are some {\em simple}
defences to our system, but there are probably several better ones.

\subsection{1 level bids}

\begin{itemize}
\item 1\clubs
	One of the following:
	\begin{itemize}
	\item Your defence to a possibly short club or diamond
	\item Your defence to a weak no-trump
	\item Your defence to a natural club opening 
	\item Your defence to a polish club
	\end{itemize}
\item 1\diamonds
	\begin{itemize}
	\item Defend the same as you would a strong (e.g. precision) club
	\end{itemize}
\item 1\hearts (Level 5 only)
	\begin{itemize}
	\item X = a heart overcall
	\item 1\spades = takeout of spades
	\item 1N = normal 1N overcall range showing a spade stop
	\item suits = natural overcalls
	\item 2\spades = whatever a cue of a natural 1 spade would mean 
	\item Jump bids = normal jump-overcall range
	\end{itemize}
\item 1\spades (Level 5 only)
	\begin{itemize}
	\item X = takeout of hearts
	\item 1N = normal 1N overcall range showing a heart stop
	\item suits = natural overcalls
	\item 2\hearts = whatever a cue of a natural 1 heart would mean 
	\item Jump bids = normal jump-overcall range
	\end{itemize}
\end{itemize}

\subsection{2 level bids}

\begin{itemize}
\item 2\clubs/\diamonds/\hearts
	\begin{itemize}
	\item X = an overcall in the suit bid
	\item suits = natural overcalls
	\item 2N = normal 2N over a weak-two range with some values in the other three suits
	\end{itemize}
\item 2\spades
	\begin{itemize}
	\item X = takeout of hearts
	\item 2N = normal 2N over a weak-two range with a decent heart stop
	\item suits = natural overcalls
	\end{itemize}
\item 2NT
	\begin{itemize}
	\item X = takeout (showing hearts and clubs)
	\item 3\clubs/\hearts = natural overcalls
	\item 3\diamonds/\spades = asking for a stop for 3NT
	\end{itemize}
\end{itemize}

\subsection{3/4 level bids}

\begin{itemize}
\item 3\clubs/\diamonds/\hearts
	\begin{itemize}
	\item X = an overcall in the suit bid
	\item next suit = takeout of that suit
	\item 3N = normal 3N over a 3-level preempt with a stop in the suit shown (not bid)
	\end{itemize}
\item 3\spades
	\begin{itemize}
	\item X = spade overcall (as of a 3-minor preempt or gambling 3N opening)
	\item suits = natural overcalls (as of a 4-minor preempt or gambling 3N opening)
	\end{itemize}
\item 3NT
	\begin{itemize}
	\item X = both majors (takeoutish)
	\item suits = natural overcalls (as of a 4-minor preempt)
	\end{itemize}
\item 4\clubs/\diamonds
	\begin{itemize}
	\item X = an overcall in the suit bid (as of a 3-major preempt)
	\item suits = natural overcalls (as of a 3-major preempt)
	\item our major = takeout
	\end{itemize}
\end{itemize}




\end{document}
