System notes on Pascal's Encrypted Club

A brief history:
================

NB: the system has changed sufficiently that this history is out-of-date.  Do
not rely on it to describe our bidding!

We wanted to play a system making use of encrypted bidding.  Our first attempt
used a fairly standard Acol structure, with a weak NT and three weak 2s.  It
was soon apparent that crypto raises in these circumstances (requiring 11+,
4-card support and either A or K but not both) never come up, at least to a
first approximation.

We also quite liked the idea of light and limited openings.  This was from a
couple of experiments with a (highly illegal) forcing pass system, where a pass
in first or second seat showed any 12+, with 0-11 HCP hands forced to open
somehow.  In particular, 2-suited bids at the 1-level have a lot of
attractions.

Our next attempt was to consider the idea of a 12+ 1C opener, with all other
openings 8-11.  This is fine in theory, but EBU regulations say that this 1C
opener cannot have a 5+ card major suit without a longer minor.  This meant
that we needed an opening-values 2C bid, and generally lost the point of the
system.  We fell back upon a strong club structure.

Given the strong C structure, we then looked at our other opening bids.  Our
2-level openers are pre-emptive, and show either single-suiters or two-suiters.
This is based upon the Suction convention: 2C shows D or both majors, 2D shows
H or S&C, 2H shows S or C&D, and 2S shows H and a minor.  We don't have a way
to show a S&D two-suiter; this is a likely future use for 2NT.

Major suit openings are played as takeout bids.  This means that 1S promises 5+
H and 0-3 S, with 1H promising 5+ S and 0-3 H.  These were, at one stage,
4-card openings in the other major, but we found that this made bidding in
competition too difficult.  In uncontested auctions, this has the great
advantage that you gain a "free" cue-bid: after a 1S opening, 2S shows a good
raise of H while 2H, 3H etc. are pre-emptive.

The next implication is that we need a way to show hands with 4+ in both
majors.  This falls to 1NT; we have 2C/2D/2H/2S as weak bids "to play", and 2NT
as a strong enquiry.  Bids in the major suits are primarily pre-emptive; 2NT
asks for range and better major.  3m is strong and one-round forcing.

1C includes balanced hands 16-19 and 23+ (we're using 2NT as a 20-22 balanced
hand, for now).  The 1D opening, therefore, has to cover all opening hands that
do not fall into one of the above categories.  It therefore includes:
    - 12-15 balanced.
    - Unbalanced hands, 8-15, with no 5-card major and at most one 4-card
      major.

Responses to 1D are selected to allow the 1NT rebid, showing a weak NT opener,
on all auctions.  Therefore: 1H = 0-7, 1S = 8+.  After interference, pass is
weak; any other action promises 8+.

So that's our one-level openings.  For two-level openers, we wanted them to be
the two-suited bids we'd originally considered for our one-level openings.  I
can't remember if it was Matt or me who first found out about Suction as a
defence to 1NT, or who suggested it as a structure for 2-level openings, but
the fact remains that we like it.  Essentially, each 2-level bid shows either a
weak 2 in the suit above OR the other two suits.  Of course, this leaves the
non-touching pairs uncovered, and leaves you with two ways to show a pre-empt
in clubs.  We soon thought of playing 2S as showing H and a minor (as opposed
to the original C or H&D).  This still left S&D uncovered, but we decided to
cope.

At this stage, our 2NT opening was fairly standard: 20-22 balanced or
semi-balanced.  After a few sessions playing, we modified this to include the
spade-diamond two-suiters.  Of course, that makes the bid forcing, so we
changed the NT ranges such that this included our game-forcing NT hands: any
23+.  Responses are suit preference, on the assumption that the weak hand is
more frequent, with transfer preference being used if responder has a strong
hand.  If opener has the strong option, he can rebid 3NT.

The next change (there's still more!) was to include GF two-suiters in our
2-level openings.  This narrows the specification of the 1C opening, and is
easy to specify: opener can rebid his opened suit (whichever it is) over any
response.  In the 2NT case, this opening now shows GF (semi-)balanced, GF with
S&D, or weak with S&D.  In the two-suited case, any rebid other than 3NT shows
the strong hand, and any transfer break does also.

Finally, our responses to 1C: initially, these were natural, promising a 4-card
suit and 8+ (with any weak hands going via 1D).  Boring!  We then tried
encrypted responses: 1H promised HA or HK, not both; 1S showed neither or both
in H, and exactly one in S, and so on.  1NT was a positive with no potential
key; 1D was the negative.  Opener would rebid NT if balanced, 2H to deny a
potential key, and anything else to show a potential place to play if holding
the key K or denying if holding the key A.  We ditched this structure not
because it doesn't work but because it frequently (read nearly always) ends up
at the 3-level before either hand has actually mentioned a suit, which could
still be only 4 cards.

We now play a far more elegant structure, similar to our responses to 1D: 1D is
negative (0-7), 1H is positive distributional (8+), and 1S is positive balanced
(8+, again).

As for the system name: suction is measured in Pascals, and the system is
encrypted.  It went from there.

System description
==================

Philosophy:
-----------

   - establish strength first
   - establish shape (balanced/unbalanced) second
   - then find a contract
   - single-suited hands are all 6+ cards and deny a side 4-card suit.

Openings:
---------

1C  - 12-15 balanced or 10-15 unbalanced, no five-card major and at most one 
      four-card major.
1D  - 16+; excludes balanced hands 24+ and game-forcing two-suited hands.
1H  - 10-15 unbalanced, with 4+ spades and additional constraints: first or
      second seat non-vulnerable, possible canape with hearts, but 5+ spades if
      not holding 4+ hearts. Other positions, or vulnerable, 5+ spades and 0-3
      hearts.
1S  - 10-15 unbalanced, 5+ hearts, and 0-3 spades.
1NT - 9-11 balanced in first or second non-vulnerable; otherwise 10-15
      unbalanced with 4+ in both majors.

2C  - weak with D, single suited OR weak with H&S, 4-4 or better OR GF with H&S.
2D  - weak with H, single suited OR weak with S&C, 5+/4+ or better OR GF with S&C.
2H  - weak with S, single suited OR weak with C&D, 4+/4+ OR GF with C&D.
2S  - weak with H and a minor OR GF with H and a minor. 4+/4+.
2NT - weak with D&S, 4+/4+, OR 24+ balanced OR GF with D&S.

3-level: natural in suits; 3NT gambling.
Higher-level openings: mostly natural excepting 4NT: specific ace-asking.

Responses & rebids
------------------

1C:
   1D = 0-7, no 6-card suit
   1H = 8-15 any
   1S = 15+ any, game-forcing.
      - major suit bids promise exactly 4; either 4-4-4-1 with singleton other
         major or has a longer minor (as there's no other hand shape that fits the
         opening and has exactly one 4-card major)
      - 1NT rebid = 12-15 balanced; see "Continuations after 1NT".
      - minor suit bids deny a 4-card major.
   2x = 6+ card suit, 0-7. Non-forcing.
      - Opener will pass unless holding something exciting in context.
         Principle: find a fit and find a level. No close games at matchpoints.

1D:
   1H = 0-7, no 6-card suit.
   1S = 8+, game-forcing.
      - 1NT rebid = 16-19 balanced; see "Continuations after 1NT".
      - Simple suit rebids are suction-style, showing either the suit above
         (single-suited) or the other two (C-> D or majors; D-> H or (S&minor); H->
         S or minors; S->C or (H&minor). Could be 3-suited, short in the suit
         above. These are shown (2C-2D-3C: short D. 2D-2H-2S: short H. 2H-2S-3H:
         short S. 2S-3C-3S: short C.).
            o Responder will bid the next suit (relay). Opener then rebids NT with the
               single-suited hand, or a suit with the two-suited option.
      - Jump suit rebids are GF, single-suited.
      - 2NT rebid = 20-23 balanced; see "Continuations after 2NT".
   2x = 6+ card suit, 0-7. Non-forcing.
      - Opener will pass unless holding something exciting in context.

1H (if denying H):
   1S/2S/3S/4S = to play/limit raises/to play.
   1NT = 1-round force.
   2C/D = natural, forcing.
   2H = GF+ raise, neither or both of S AK.
   2NT = good raise, with exactly one of S AK.
      - Next bid shows 0/1 loser and the S K, or 2/3 losers and the SA.
         Rebidding S denies the other top honour.
   3C/3D = fit jump.
   3H = GF raise, two of AKQ.
   4C/D/H = splinter agreeing spades

1S:
   2H/3H/4H = to play/limit raise/to play.
   1NT = 1-round force.
   2C/D = natural, forcing.
   2S = GF+ raise, neither or both of H AK.
   2NT = good raise, with exactly one of H AK.
      - Next bid shows 0/1 loser and the H K, or 2/3 losers and the HA.
         Rebidding H denies the other top honour.
   3C/3D = fit jump.
   3S = GF raise, two of AKQ.
   4C/D/S = splinter agreeing hearts

1H if not denying H:
   1S = forcing enquiry.
      - 1NT = both majors. Continuations as after artificial 1NT opening.
   1NT = non-forcing.
   2C/D = natural, forcing.
   2H = GF+ raise, neither or both of S AK.
   2S/3S/4S = limit raises/to play.
   2NT = good raise, with exactly one of S AK.
      - Next bid shows 0/1 loser and the S K, or 2/3 losers and the SA.
         Rebidding S denies the other top honour.
   3C/3D = fit jump.
   3H = GF raise, two of AKQ.
   4C/D/H = splinter agreeing spades

1NT if artificial:
   2C/D: weak, to play.
   2H/S: preference, to play.
   2NT: forcing enquiry:
      - 3C/D strong, 3H/S weak. 3C/H better H, 3D/S better S.
   3C/D: strong, good suit.
   3H/S: good raise.

2C:
   2D = pass/correct.
      - 2NT = GF opening.
   2H = forcing; major-focused.
      - 2S = better S
      - 2NT = both equal
      - 3D = weak, diamonds.
      - 3C = GF opening
      - 3H = better H
   2NT = forcing, how good are your diamonds?
      - 3C = GF opening
      - 3D = bad diamonds
      - 3H = 2-suited
      - 3S = decent diamonds
      - 3NT = top-of-range diamonds
2D:
   2H = pass/correct.
      - 2NT = GF opening.
   2S = forcing, C/S-focused.
      - 2NT = both equal.
      - 3C = better C.
      - 3D = GF opening.
      - 3H = better S, minimum.
      - 3S = better S, maximum.
   2NT = forcing, how good are your hearts?
      - 3C = two-suited.
      - 3D = GF opening.
      - 3H = dire, single-suited.
      - 3S = decent, single-suited.
      - 3NT = top-of-range hearts.
2H:
   2S = pass/correct.
      - 2NT = GF opening.
   3C = forcing, C/D-focused.
      - 3D = better D.
      - 3H = GF opening.
      - 3S = single-suited.
      - 3NT = both equal.
      - 4C = better C.
   2NT = forcing, how good are your spades?
      - 3C = two-suited.
      - 3D = dire spades.
      - 3H = GF opening.
      - 3S = decent, single-suited.
      - 3NT = top-of-range spades.
2S:
   2NT = forcing, which minor?
      - 3C/D: that minor
      - 3H: GF, H&C
      - 3S: GF, H&D
   3C/D/H, etc: to play (pass/correct if a minor).
   3S = setting H, asking for information

2NT:
   3D/3S: simple preference
      - After weak preference, any rebid shows GF hand.
   3C/3H: transfer preference; stronger.
      - Opener breaks transfer if GF two-suiter; suit has been set so bid controls.
      - A 3NT rebid shows 24+ balanced
         o 3C/D = 2 under transfers to H/S
         o 3H/S = Swiss for no trumps


Continuations after 1NT:
------------------------

This refers only to a natural 1NT, though it may be 9-11, 12-15 or 16-19
depending on route. Maximum is an 11-count, 14-15-count, or 18-19-count
depending on range.

2D and 2H are transfers; transfer breaks apply with any 4-card fit.  2NT
shows 4/5-card support, no side 4-card suit, and a maximum.  Other suits
show 4 cards, and 4-card support with a maximum.  A simple jump acceptance
shows 4-card support and a minimum.

2S is a range enquiry/transfer to C.  2NT shows a minimum, after which 3C
is to play and other bids are GF.  3C shows any maximum, and can be passed
if 2S was weak takeout.

2NT is a transfer to diamonds; 3C is the only available super-accept.

3C promises a long suit that may need some help to run.  Opener should
pass, or bid 3NT with (e.g.) Kxx in the suit.

3D is 5-5 in the majors, at least invitational.

2C is where it gets distinctive.  This is 5-card modified puppet Keri.
Opener responds 2H/S with a 5-card major, and 2D with all other hands. 
This allows the system to be used with weak diamond hands with tolerance
for at least one major (e.g. 4=0=6=3 pattern with very limited values).
After this response, 3D is to play; other rebids by responder are as after
a 5-card Stayman sequence.

After 1NT-2C-2D, responder has some options.  All of these show
invitational+ values.

2H = 4 spades; may have 4 hearts.
   - 2S: forcing; promises 4 hearts.
      - 2NT/3NT: inv/GF, not 4 hearts.
      - 3C/3D: 4 hearts, feature in suit bid; GF
      - 3H/4H: inv/GF, 4 hearts.
      - 3S: GF, 4 hearts, auto-Swiss.
      - 4C/4D: GF, 4 hearts, Swiss.
   - 2NT/3NT: min/max, denies a 4-card major.
   - 3C/3D/3H: 4 spades, feature in suit bid; max.
   - 3S/4S: min/max, 4 spades.
   - 4C/4D: max, non-serious Swiss.

2S = 4 hearts; denies 4 spades.
   - 2NT/3NT: min/max, denies 4 H.
   - 3C/3D: 4 hearts, feature in suit bid, max.
   - 3H/4H: min/max, 4 hearts.
   - 3S: 4 hearts, maximum, auto-Swiss.
   - 4C/4D: GF, 4 hearts, Swiss.

2NT = 3-3 majors; invitational strength.
3C/D = 3-3 majors, GF, values in the minor.
3NT=3-3 majors, GF.

Continuations after 2NT:
------------------------

3C: advanced 5-card puppet Stayman
   - 3H/S shows 5 cards.
   - 3D promises at least one 4-card major
      o 3H: 0-3 hearts, 4 spades
      o 3S: 4 hearts, 0-3 or 5 spades
         - 3NT: 4 spades
            o 4S: 5-4 spades and hearts
         - 4C/D: Swiss
         - 4H: to play.
      o 3N: 3-3 majors
      o 4C: 4-4 majors, invitational
      o 4D: 4-4 majors, GF
   - 3NT denies a 4-card major.

3D/3H: transfers.  Superaccept with any 4-card fit; note that 3NT is a
superaccept that still allows the use of Swiss.

3S: minor-suit Stayman.  Asks for a 4- or 5-card minor.
   - 3NT: no 4-card minor.
   - 4C/4D: 4 cards.
   - 4H/4S: 5 cards in the corresponding minor.

Encrypted bidding:
------------------

Once responder makes an encrypted raise, opener can either accept or decline
the key.  To decline, simply rebid the suit (cheaply with a minimum; jump if
maximum.  Use common sense.).

If the key is accepted, subsequent bids are either two-way or three-way, depending on the key.

Two-way encryption:

This is far more common, as it requires responder and opener to have one each
of the trump AK.  Opener's rebid shows a useful side-suit if holding the trump
K (0-1 losers) or a potential weakness if holding the trump A (2-3 losers).
Responder can then either place the contract, cue-bid (either showing with the
K or denying with the A), or use Blackwood.

Blackwood here is Roman Keycard, but it's modified: 5C is 0/3, 5D is 1, 5H is 2
without the Q, and 5S is 2 with.  Note that there are only 3 keycards still of
interest!

Three-way encryption:

Requires AKQ of trumps between the two hands, and enough values for a game-forcing raise in responder's hand.  Rare!

Opener's rebid shows a first-round control; the location of this control is
encrypted.  We use the concept of the shift here: a shift of 1 indicates the
next suit up (C from NT or S, D from C, H from D, S from H).  A shift of 2
indicates the next suit after that, and a shift of 3 shows the final suit.
Each cue therefore says nothing about control in the suit bid.

The sum of the shift and the HCP of opener's honour is 5.  Opener, holding a D
control, therefore bids C with the trump A, S/NT with the trump K, and H with
the trump Q.

Of course, Blackwood goes funny here too.  You only need to show 0-3 keycards,
and the location of the Q is known, so 0 - 1 - 2 - 3 would seem a sensible set
of responses.



Slam conventions:
-----------------

RKCB (3014)

Cues - once a suit is agreed and a GF is established, new suits are cues.

General Swiss:

4C/4D, once a game-force is established and suit agreed, are slam tries, based
on control points

Control points are 2 for an ace, 1 for a king/singleton, and 1 for the queen of
trumps.  There are 13 available from honours; it is possible to substitute
outside singletons for kings, but care must be taken to avoid double-counting.
11 CPs are necessary for a small slam; 13 for a grand.

4C: 4 or 6 (exceptionally 8) control points)
   - trump suit: 0-4 CPs
   - 4D: 5/6 CP, relay
      - trump suit: 4 CPs
      - suits: 6 CP, lowest king/singleton
   - suits: 7 or 9 CP, lowest king/singleton
   - small slam: 8 CP
   - grand slam: 10 CP

4D: 5 or 7 (exceptionally 9) control points)
   - trump suit: 0-3 CPs
   - 4H: 4/5 CP, relay
      - trump suit: 5 CPs
      - suits: 7 CP, lowest king/singleton
   - suits: 6 or 8 CP, lowest king/singleton
   - small slam: 7 CP
   - grand slam: 9 CP

If you find duplication in kings/singletons subtract one CP and sign off at the
appropriate level. If your partner signs off and you have 2 unshown CPs so far,
raise one level.

If the hand bidding Swiss has shown a GF, add 2 to all numbers. If it has shown
a weak hand (opening at the 2 level or giving a 0-7 response to 1D), subtract 2
from all numbers.

Competitive bidding:
--------------------

Overcalls are constructive and for takeout, promising the two unbid suits (at least 5-4).
Where opponents have shown two suits, overcalls are natural.

When responding to an overcall, a bid of one of the suits is simply competitive
and showing preference.  A good raise of the cheaper suit can be shown with the
cheaper cuebid, and a good raise of the other suit can be shown with the other
cuebid.  

Similarly, 1NT and 2NT can be used as encrypted raises of the cheaper and
higher suits respectively.


This means that a double needs to include the possibility of a single-suited
hand, on which one might make a natural overcall.  Responder should make the
cheapest bid if holding a very weak hand.

All strong hands are shown via an immediate cuebid of opponent's suit.  Again,
a next-step response is negative.

Two-suited pre-emptive hands can be shown via 2NT.  It is worth noting that
this is the Tom Jones 2NT: It's Not Unusual.  This promises both majors over a
minor-suit opening, or the other major and an unspecified minor over a
minor-suit opening.  

Jump overcalls are weak; jump overcalls of a major suit over a minor-suit
opening may include the other minor (possible canape).  Similarly, jump
overcalls of 3C over a 1 of a major opening could include D.

Jump cuebids are stopper-asking for 3NT; they promise a long running suit.  An
overcall of 3NT promises a long running suit and a stopper.

Dealing with interference
-------------------------

1C-(X) (penalty interest (showing 16+):
   - Redouble creates a game-force, and indicates that the double may not have been 
      a winning action.
   - Pass forces redouble; either to play or with a two-suiter not including clubs.
   - 1D/1H are transfers to 1H/1S.  Note that opener may pass these if holding a 5-card 
      suit.
1C-(X) (showing C):
   - Redouble creates a game-force; passes are now forcing and doubles are for penalty.
   - Pass shows a weak hand, with nowhere in particular to play.
   - Suit bids are natural, 8-14, and promise 4 cards.
   - 1NT shows a balanced hand, 8-14, with a club stop.
1C-(X) (takeout):
   - Redouble still creates a forcing pass situation.
   - Ask opponents "takeout of what, exactly?".
   - 1NT cannot show a stopper in any particular suit.
1C-(natural bid):
   - Responder's double creates a game-force.
   - 1NT promises a stop; 8-14 balanced.
   - Suit bids are natural, 8-14.
   - Pass is a weak hand with nowhere to go (potentially including an 8-11 balanced hand with no stop).

1D-(X):
   - Redouble with good 8+ hands.
   - Pass with weak hands.
   - Suit bids are natural, weak, 6-card suit.
1D-(bid):
   - Pass with weak hands.
   - Double to establish a game-force (and forcing passes, etc.).
      - Suction rebids are on; immediate NT bid denies a stop.
      - To show a stop, Suction-force into opponents' suit then bid NT.
   - Suit bids are natural, weak, 6-card suit.

Dealing with balancing-seat interference (e.g. 1D-(P)-1S-(2C)):

1C-(P)-1S-(anything):
1D-(P)-1S-(anything):
   - Passes are forcing
   - NT bids promise a stopper.
   - Suction-style rebids are on if 3C is a sufficient bid (1D opener)
   - Pass indicates a balanced hand without a stopper in opponents' suit.
      - Responder's NT bids here are Lebensohl style
      - There are no weak hands to show, so the only difference is in showing/denying a stopper.
   - Double is penalties.

1C-(P)-1D-(anything): it's a good idea to subside quietly.  

1C-(P)-1H-(anything): passes are non-forcing. 

