\documentclass[a4paper,12pt]{article}

\usepackage{palatino}
\usepackage{fullpage}

\author{Matthew Johnson\\Henry Lockwood}
\title{System notes on Pascal's Encrypted Club}

\begin{document}

\maketitle
\tableofcontents

\section{A brief history}

NB: the system has changed sufficiently that this history is out-of-date.  Do
not rely on it to describe our bidding!

We wanted to play a system making use of encrypted bidding.  Our first attempt
used a fairly standard Acol structure, with a weak NT and three weak 2s.  It
was soon apparent that crypto raises in these circumstances (requiring 11+,
4-card support and either A or K but not both) never come up, at least to a
first approximation.

We also quite liked the idea of light and limited openings.  This was from a
couple of experiments with a (highly illegal) forcing pass system, where a pass
in first or second seat showed any 12+, with 0--11 HCP hands forced to open
somehow.  In particular, 2-suited bids at the 1-level have a lot of
attractions.

Our next attempt was to consider the idea of a 12+ 1$\clubsuit$ opener, with all other
openings 8--11.  This is fine in theory, but EBU regulations say that this 1$\clubsuit$
opener cannot have a 5+ card major suit without a longer minor.  This meant
that we needed an opening-values 2$\clubsuit$ bid, and generally lost the point of the
system.  We fell back upon a strong club structure.

Given the strong $\clubsuit$ structure, we then looked at our other opening bids.  Our
2-level openers are pre-emptive, and show either single-suiters or two-suiters.
This is based upon the Suction convention: 2$\clubsuit$ shows $\diamondsuit$ or both majors, 2$\diamondsuit$ shows
$\heartsuit$ or $\spadesuit$\&$\clubsuit$, 2$\heartsuit$ shows $\spadesuit$ or $\clubsuit$\&$\diamondsuit$, and 2$\spadesuit$ shows $\heartsuit$ and a minor.  We don't have a way
to show a $\spadesuit$\&$\diamondsuit$ two-suiter; this is a likely future use for 2NT.

Major suit openings are played as takeout bids.  This means that 1$\spadesuit$ promises 5+
H and 0--3 $\spadesuit$, with 1$\heartsuit$ promising 5+ $\spadesuit$ and 0--3 $\heartsuit$.  These were, at one stage,
4-card openings in the other major, but we found that this made bidding in
competition too difficult.  In uncontested auctions, this has the great
advantage that you gain a ``free" cue-bid: after a 1$\spadesuit$ opening, 2$\spadesuit$ shows a good
raise of $\heartsuit$ while 2$\heartsuit$, 3$\heartsuit$ etc. are pre-emptive.

The next implication is that we need a way to show hands with 4+ in both
majors.  This falls to 1NT; we have 2$\clubsuit$/2$\diamondsuit$/2$\heartsuit$/2$\spadesuit$ as weak bids ``to play", and 2NT
as a strong enquiry.  Bids in the major suits are primarily pre-emptive; 2NT
asks for range and better major.  3m is strong and one-round forcing.

1$\clubsuit$ includes balanced hands 16--19 and 23+ (we're using 2NT as a 20--22 balanced
hand, for now).  The 1$\diamondsuit$ opening, therefore, has to cover all opening hands that
do not fall into one of the above categories.  It therefore includes:

\begin{itemize}
\item 12--15 balanced.
\item Unbalanced hands, 8--15, with no 5-card major and at most one 4-card
      major.
\end{itemize}

Responses to 1$\diamondsuit$ are selected to allow the 1NT rebid, showing a weak NT opener,
on all auctions.  Therefore: 1$\heartsuit$ = 0--7, 1$\spadesuit$ = 8+.  After interference, pass is
weak; any other action promises 8+.

So that's our one-level openings.  For two-level openers, we wanted them to be
the two-suited bids we'd originally considered for our one-level openings.  I
can't remember if it was Matt or me who first found out about Suction as a
defence to 1NT, or who suggested it as a structure for 2-level openings, but
the fact remains that we like it.  Essentially, each 2-level bid shows either a
weak 2 in the suit above OR the other two suits.  Of course, this leaves the
non-touching pairs uncovered, and leaves you with two ways to show a pre-empt
in clubs.  We soon thought of playing 2$\spadesuit$ as showing $\heartsuit$ and a minor (as opposed
to the original $\clubsuit$ or $\heartsuit$\&$\diamondsuit$).  This still left $\spadesuit$\&$\diamondsuit$ uncovered, but we decided to
cope.

At this stage, our 2NT opening was fairly standard: 20--22 balanced or
semi-balanced.  After a few sessions playing, we modified this to include the
spade-diamond two-suiters.  Of course, that makes the bid forcing, so we
changed the NT ranges such that this included our game-forcing NT hands: any
23+.  Responses are suit preference, on the assumption that the weak hand is
more frequent, with transfer preference being used if responder has a strong
hand.  If opener has the strong option, he can rebid 3NT.

The next change (there's still more!) was to include GF two-suiters in our
2-level openings.  This narrows the specification of the 1$\clubsuit$ opening, and is
easy to specify: opener can rebid his opened suit (whichever it is) over any
response.  In the 2NT case, this opening now shows GF (semi-)balanced, GF with
S\&$\diamondsuit$, or weak with $\spadesuit$\&$\diamondsuit$.  In the two-suited case, any rebid other than 3NT shows
the strong hand, and any transfer break does also.

Finally, our responses to 1$\clubsuit$: initially, these were natural, promising a 4-card
suit and 8+ (with any weak hands going via 1$\diamondsuit$).  Boring!  We then tried
encrypted responses: 1$\heartsuit$ promised HA or HK, not both; 1$\spadesuit$ showed neither or both
in $\heartsuit$, and exactly one in $\spadesuit$, and so on.  1NT was a positive with no potential
key; 1$\diamondsuit$ was the negative.  Opener would rebid NT if balanced, 2$\heartsuit$ to deny a
potential key, and anything else to show a potential place to play if holding
the key K or denying if holding the key A.  We ditched this structure not
because it doesn't work but because it frequently (read nearly always) ends up
at the 3-level before either hand has actually mentioned a suit, which could
still be only 4 cards.

We now play a far more elegant structure, similar to our responses to 1$\diamondsuit$: 1$\diamondsuit$ is
negative (0--7), 1$\heartsuit$ is positive distributional (8+), and 1$\spadesuit$ is positive balanced
(8+, again).

As for the system name: suction is measured in Pascals, and the system is
encrypted.  It went from there.

\section{System description}

\subsection{Philosophy}

\begin{itemize}
\item establish strength first
\item establish shape (balanced/unbalanced) second
\item then find a contract
\item single-suited hands are all 6+ cards and deny a side 4-card suit.
\end{itemize}

\subsection{Openings}

\subsubsection{1 level}

\begin{tabular}{ll}
1$\clubsuit$  & 12--15 balanced or 10--15 unbalanced, no five-card major and at most one \\
	 & four-card major.\\
1$\diamondsuit$  & 16+; excludes balanced hands 24+ and game-forcing two-suited hands.\\
1$\heartsuit$  & 10--15 unbalanced, with 4+ spades and additional constraints: first or
      second seat\\
	 & non-vulnerable, possible canape with hearts, but 5+ spades if
      not holding 4+ hearts. \\
	 & Other positions, or vulnerable, 5+ spades and 0--3 hearts.\\
1$\spadesuit$  & 10--15 unbalanced, 5+ hearts, and 0--3 spades.\\
1NT & 9--11 balanced in first or second non-vulnerable; otherwise 10--15
      unbalanced with \\
	 & 4+ in both majors.\\
\end{tabular}

\subsubsection{2 level}

\begin{tabular}{ll}
2$\clubsuit$  & weak with $\diamondsuit$, single suited OR weak with $\heartsuit$\&$\spadesuit$, 4--4 or better OR GF with $\heartsuit$\&$\spadesuit$.\\
2$\diamondsuit$  & weak with $\heartsuit$, single suited OR weak with $\spadesuit$\&$\clubsuit$, 5+/4+ or better OR GF with $\spadesuit$\&$\clubsuit$.\\
2$\heartsuit$  & weak with $\spadesuit$, single suited OR weak with $\clubsuit$\&$\diamondsuit$, 4+/4+ OR GF with $\clubsuit$\&$\diamondsuit$.\\
2$\spadesuit$  & weak with $\heartsuit$ and a minor OR GF with $\heartsuit$ and a minor. 4+/4+.\\
2NT & weak with $\diamondsuit$\&$\spadesuit$, 4+/4+, OR 24+ balanced OR GF with $\diamondsuit$\&$\spadesuit$.\\
\end{tabular}

\subsubsection{Higher openings}

\begin{tabular}{ll}
3-level & natural in suits. \\
3NT & gambling. \\
Higher-level openings & natural. \\
4NT & specific ace-asking. \\
\end{tabular}

\newpage 

\subsection{Responses \& rebids}

\begin{itemize}
\item 1$\clubsuit$:
	\begin{itemize}
   \item 1$\diamondsuit$ = 0--9, no 6-card suit
		\begin{itemize}
      \item suits = unbalanced, possible canape
      \item 1NT = 12--15 balanced; responses weak takeout.
		\end{itemize}
   \item 1$\heartsuit$ = 10--15 any
		\begin{itemize}
      \item 1$\spadesuit$ = any 10-11
      \item 1NT = 12--15 balanced; see ``Continuations after 1NT".
      \item 2 major suit bids 12-15 unbal and promise exactly 4; either 4-4-4-1
            with singleton other major or has a longer minor (as there's no other
            hand shape that fits the opening and has exactly one 4-card major)
      \item 2 minor suit bids deny a 4-card major and are 12-15 unbal.
		\end{itemize}
   \item 1$\spadesuit$ = 15+ any, game-forcing.
		\begin{itemize}
      \item major suit bids promise exactly 4; either 4-4-4-1 with singleton other
         major or has a longer minor (as there's no other hand shape that fits the
         opening and has exactly one 4-card major)
      \item 1NT = 12--15 balanced; see ``Continuations after 1NT".
      \item minor suit bids deny a 4-card major.
		\end{itemize}
   \item 1N/2x = transfer weak jump shift 6+ card suit above, 0--9.
		\begin{itemize}
		\item Complete the transfer with nothing more to say
		\item 2N = extra values, nothing in the transfer suit
		\item Jump-completion = good support, invitational
		\item new suits = extras, tolerance, side suit
		\end{itemize}
	\end{itemize}

\item 1$\diamondsuit$:
	\begin{itemize}
   \item 1$\heartsuit$ = 0--7, no 6-card suit.
   \item 1$\spadesuit$ = 8+, game-forcing.
		\begin{itemize}
      \item 1NT rebid = 16--19 balanced; see ``Continuations after 1NT".
      \item Simple suit rebids are suction-style, showing either the suit above
         (single-suited) or the other two ($\clubsuit$$\rightarrow$ $\diamondsuit$ or majors; $\diamondsuit$$\rightarrow$ $\heartsuit$ or ($\spadesuit$\&minor); $\heartsuit$$\rightarrow$
         $\spadesuit$ or minors; $\spadesuit$$\rightarrow$$\clubsuit$ or ($\heartsuit$\&minor). Could be 3-suited, short in the suit
         above. These are shown (2$\clubsuit$-2$\diamondsuit$-3$\clubsuit$: short $\diamondsuit$. 2$\diamondsuit$-2$\heartsuit$-2$\spadesuit$: short $\heartsuit$. 2$\heartsuit$-2$\spadesuit$-3$\heartsuit$:
         short $\spadesuit$. 2$\spadesuit$-3$\clubsuit$-3$\spadesuit$: short $\clubsuit$.).
			\begin{itemize}
			\item Responder will bid the next suit (relay). Opener then rebids NT with the
				single-suited hand, or a suit with the two-suited option.
			\end{itemize}
      \item Jump suit rebids are GF, single-suited.
      \item 2NT rebid = 20--23 balanced; see ``Continuations after 2NT".
		\end{itemize}
   \item 1N/2x = transfer weak jump shift 6+ card suit above, 0--7.
		\begin{itemize}
		\item Complete the transfer with nothing more to say
		\item 2N = extra values, nothing in the transfer suit
		\item Jump-completion = good support, invitational
		\item new suits = extras, tolerance, side suit
		\item jump new suit = GF single suiter
		\end{itemize}
	\end{itemize}

\item 1$\heartsuit$ (if denying $\heartsuit$):
	\begin{itemize}
   \item 1$\spadesuit$/2$\spadesuit$/3$\spadesuit$/4$\spadesuit$ = to play/limit raises/to play.
   \item 1NT = 1-round force.
   \item 2$\clubsuit$/$\diamondsuit$ = natural, forcing.
   \item 2$\heartsuit$ = GF+ raise, neither or both of $\spadesuit$ AK.
   \item 2NT = good raise, with exactly one of $\spadesuit$ AK.
		\begin{itemize}
      \item Next bid shows 0/1 loser and the $\spadesuit$ K, or 2/3 losers and the SA.
         Rebidding $\spadesuit$ denies the other top honour.
		\end{itemize}
   \item 3$\clubsuit$/3$\diamondsuit$ = fit jump.
   \item 3$\heartsuit$ = GF raise, two of AKQ. (see ``Three-way encryption")
   \item 4$\clubsuit$/$\diamondsuit$/$\heartsuit$ = splinter agreeing spades
	\end{itemize}

\item 1$\spadesuit$:
	\begin{itemize}
   \item 2$\heartsuit$/3$\heartsuit$/4$\heartsuit$ = to play/limit raise/to play.
   \item 1NT = 1-round force.
   \item 2$\clubsuit$/$\diamondsuit$ = natural, forcing.
   \item 2$\spadesuit$ = GF+ raise, neither or both of $\heartsuit$ AK.
   \item 2NT = good raise, with exactly one of $\heartsuit$ AK.
		\begin{itemize}
      \item Next bid shows 0/1 loser and the $\heartsuit$ K, or 2/3 losers and the HA.
         Rebidding $\heartsuit$ denies the other top honour.
		\end{itemize}
   \item 3$\clubsuit$/3$\diamondsuit$ = fit jump.
   \item 3$\spadesuit$ = GF raise, two of AKQ. (see ``Three-way encryption")
   \item 4$\clubsuit$/$\diamondsuit$/$\spadesuit$ = splinter agreeing hearts
	\end{itemize}

\item 1$\heartsuit$ if not denying $\heartsuit$:
	\begin{itemize}
   \item 1$\spadesuit$ = forcing enquiry.
		\begin{itemize}
      \item 1NT = both majors. Continuations as after artificial 1NT opening.
		\end{itemize}
   \item 1NT = non-forcing.
   \item 2$\clubsuit$/$\diamondsuit$ = natural, forcing.
   \item 2$\heartsuit$ = GF+ raise, neither or both of $\spadesuit$ AK.
   \item 2$\spadesuit$/3$\spadesuit$/4$\spadesuit$ = limit raises/to play.
   \item 2NT = good raise, with exactly one of $\spadesuit$ AK.
		\begin{itemize}
      \item Next bid shows 0/1 loser and the $\spadesuit$ K, or 2/3 losers and the SA.
         Rebidding $\spadesuit$ denies the other top honour.
		\end{itemize}
   \item 3$\clubsuit$/3$\diamondsuit$ = fit jump.
   \item 3$\heartsuit$ = GF raise, two of AKQ. (see ``Three-way encryption")
   \item 4$\clubsuit$/$\diamondsuit$/$\heartsuit$ = splinter agreeing spades
	\end{itemize}

\item 1NT if artificial:
	\begin{itemize}
   \item 2$\clubsuit$/$\diamondsuit$: weak, to play.
   \item 2$\heartsuit$/$\spadesuit$: preference, to play.
   \item 2NT: forcing enquiry:
		\begin{itemize}
      \item 3$\clubsuit$/$\diamondsuit$ strong, 3$\heartsuit$/$\spadesuit$ weak. 3$\clubsuit$/$\heartsuit$ better $\heartsuit$, 3$\diamondsuit$/$\spadesuit$ better $\spadesuit$.
		\end{itemize}
   \item 3$\clubsuit$/$\diamondsuit$: strong, good suit.
   \item 3$\heartsuit$/$\spadesuit$: good raise.
	\end{itemize}

\item 2$\clubsuit$:
	\begin{itemize}
   \item 2$\diamondsuit$ = pass/correct.
		\begin{itemize}
      \item 2NT = GF opening.
		\end{itemize}
   \item 2$\heartsuit$ = forcing; major-focused.
		\begin{itemize}
      \item 2$\spadesuit$ = better $\spadesuit$
      \item 2NT = both equal
      \item 3$\diamondsuit$ = weak, diamonds.
      \item 3$\clubsuit$ = GF opening
      \item 3$\heartsuit$ = better $\heartsuit$
		\end{itemize}
   \item 2NT = forcing, how good are your diamonds?
		\begin{itemize}
      \item 3$\clubsuit$ = GF opening
      \item 3$\diamondsuit$ = bad diamonds
      \item 3$\heartsuit$ = 2-suited
      \item 3$\spadesuit$ = decent diamonds
      \item 3NT = top-of-range diamonds
		\end{itemize}
	\end{itemize}

\item 2$\diamondsuit$:
	\begin{itemize}
   \item 2$\heartsuit$ = pass/correct.
		\begin{itemize}
      \item 2NT = GF opening.
		\end{itemize}
   \item 2$\spadesuit$ = forcing, $\clubsuit$/$\spadesuit$-focused.
		\begin{itemize}
      \item 2NT = both equal.
      \item 3$\clubsuit$ = better $\clubsuit$.
      \item 3$\diamondsuit$ = GF opening.
      \item 3$\heartsuit$ = better $\spadesuit$, minimum.
      \item 3$\spadesuit$ = better $\spadesuit$, maximum.
		\end{itemize}
   \item 2NT = forcing, how good are your hearts?
		\begin{itemize}
      \item 3$\clubsuit$ = two-suited.
      \item 3$\diamondsuit$ = GF opening.
      \item 3$\heartsuit$ = dire, single-suited.
      \item 3$\spadesuit$ = decent, single-suited.
      \item 3NT = top-of-range hearts.
		\end{itemize}
	\end{itemize}

\item 2$\heartsuit$:
	\begin{itemize}
   \item 2$\spadesuit$ = pass/correct.
		\begin{itemize}
      \item 2NT = GF opening.
		\end{itemize}
   \item 3$\clubsuit$ = forcing, $\clubsuit$/$\diamondsuit$-focused.
		\begin{itemize}
      \item 3$\diamondsuit$ = better $\diamondsuit$.
      \item 3$\heartsuit$ = GF opening.
      \item 3$\spadesuit$ = single-suited.
      \item 3NT = both equal.
      \item 4$\clubsuit$ = better $\clubsuit$.
		\end{itemize}
   \item 2NT = forcing, how good are your spades?
		\begin{itemize}
      \item 3$\clubsuit$ = two-suited.
      \item 3$\diamondsuit$ = dire spades.
      \item 3$\heartsuit$ = GF opening.
      \item 3$\spadesuit$ = decent, single-suited.
      \item 3NT = top-of-range spades.
		\end{itemize}
	\end{itemize}

\item 2$\spadesuit$:
	\begin{itemize}
   \item 2NT = forcing, which minor?
		\begin{itemize}
      \item 3$\clubsuit$/$\diamondsuit$: that minor
      \item 3$\heartsuit$: GF, $\heartsuit$\&$\clubsuit$
      \item 3$\spadesuit$: GF, $\heartsuit$\&$\diamondsuit$
		\end{itemize}
   \item 3$\clubsuit$/$\diamondsuit$/$\heartsuit$, etc: to play (pass/correct if a minor).
   \item 3$\spadesuit$ = setting $\heartsuit$, asking for information
	\end{itemize}

\item 2NT:
	\begin{itemize}
   \item 3$\diamondsuit$/3$\spadesuit$: simple preference
		\begin{itemize}
      \item After weak preference, any rebid shows GF hand.
		\end{itemize}
   \item 3$\clubsuit$/3$\heartsuit$: transfer preference; stronger.
		\begin{itemize}
      \item Opener breaks transfer if GF two-suiter; suit has been set so bid controls.
      \item A 3NT rebid shows 24+ balanced
			\begin{itemize}
         \item 3$\clubsuit$/$\diamondsuit$ = 2 under transfers to $\heartsuit$/$\spadesuit$
         \item 3$\heartsuit$/$\spadesuit$ = Swiss for no trumps
         \item 4N = Quantitative
         \item 5N = Quantitative (for 6 or 7)
			\end{itemize}
		\end{itemize}
	\end{itemize}

\item 3NT:
	\begin{itemize}
	\item pass: stops in 3 suits
	\item clubs at any level: pass or correct
	\item 4$\diamondsuit$: asks for singletons
		\begin{itemize}
		\item 4$\heartsuit$/$\spadesuit$: singleton in that suit
		\item 4N: no singletons
		\item 5$\clubsuit$/$\diamondsuit$: singleton in the other suit
		\end{itemize}
	\end{itemize}

\item 4NT:
	\begin{itemize}
	\item 5$\clubsuit$: No aces
	\item 5$\diamondsuit$: Club ace or Heart, Diamond and Spade aces
	\item 5$\heartsuit$: Diamond ace or Club, Heart and Spade aces
	\item 5$\spadesuit$: Heart ace or Club, Diamond and Spade aces
	\item 5N: Two aces not including Spades
	\item 6$\clubsuit$: Spade ace or Club, Diamond and Heart aces
	\item 6$\diamondsuit$: Spade and Club aces
	\item 6$\heartsuit$: Spade and Diamond aces
	\item 6$\spadesuit$: Spade and Heart aces
	\end{itemize}

\end{itemize}

\section{Continuations after 1NT}

This refers only to a natural 1NT, though it may be 9--11, 12--15 or 16--19
depending on route. Maximum is an 11-count, 14--15-count, or 18--19-count
depending on range.

\begin{itemize}

\item 2$\diamondsuit$ and 2$\heartsuit$ are transfers; transfer breaks apply with any 4-card fit.  2NT
shows 4/5-card support, no side 4-card suit, and a maximum.  Other suits
show 4 cards, and 4-card support with a maximum.  A simple jump acceptance
shows 4-card support and a minimum.

\item 2$\spadesuit$ is a range enquiry/transfer to $\clubsuit$.  2NT shows a minimum, after which 3$\clubsuit$ is to
play and other bids are GF, either cues or Swiss.  3$\clubsuit$ shows any maximum, and
can be passed if 2$\spadesuit$ was weak takeout.

\item 2NT is a transfer to diamonds; 3$\clubsuit$ is the only available super-accept.

\item 3$\clubsuit$ promises a long suit that may need some help to run.  Opener should
pass, or bid 3NT with (e.g.) Kxx in the suit.

\item 3$\diamondsuit$ is 5--5 in the majors, at least invitational.

\item 3$\heartsuit$/3$\spadesuit$ are slam tries in clubs and diamonds.

\item 4N is blackwood.

\item 2$\clubsuit$ is where it gets distinctive.  This is 5-card modified puppet Keri, which allows the system to be used with weak diamond hands with tolerance for at least one major (e.g. 4=0=6=3 pattern with very limited values).
\end{itemize}

\paragraph{Responses to 5-card puppet Keri}

	\begin{itemize}
	\item 2$\heartsuit$/$\spadesuit$ with a 5-card major
		\begin{itemize}
		\item 3$\diamondsuit$ to play
		\item other rebids are as after this sequence in 5 card puppet stayman.
		\end{itemize}
	\item 2$\diamondsuit$ with all other hands. 
		\begin{itemize}
		\item Pass if weak with diamonds
		\item 2$\heartsuit$ = 4 spades; may have 4 hearts.
			\begin{itemize}
			\item 2$\spadesuit$: forcing; promises 4 hearts.
				\begin{itemize}
				\item 2NT/3NT: inv/GF, not 4 hearts.
				\item 3$\clubsuit$/3$\diamondsuit$: 4 hearts, feature in suit bid; GF
				\item 3$\heartsuit$/4$\heartsuit$: inv/GF, 4 hearts.
				\item 3$\spadesuit$: GF, 4 hearts, auto-Swiss.
				\item 4$\clubsuit$/4$\diamondsuit$: GF, 4 hearts, Swiss.
				\end{itemize}
			\item 2NT/3NT: min/max, denies a 4-card major.
			\item 3$\clubsuit$/3$\diamondsuit$/3$\heartsuit$: 4 spades, feature in suit bid; max.
			\item 3$\spadesuit$/4$\spadesuit$: min/max, 4 spades.
			\item 4$\clubsuit$/4$\diamondsuit$: max, non-serious Swiss.
			\end{itemize}

		\item 2$\spadesuit$ = 4 hearts; denies 4 spades.
			\begin{itemize}
			\item 2NT/3NT: min/max, denies 4 $\heartsuit$.
			\item 3$\clubsuit$/3$\diamondsuit$: 4 hearts, feature in suit bid, max.
			\item 3$\heartsuit$/4$\heartsuit$: min/max, 4 hearts.
			\item 3$\spadesuit$: 4 hearts, maximum, auto-Swiss.
			\item 4$\clubsuit$/4$\diamondsuit$: GF, 4 hearts, Swiss.
			\end{itemize}

		\item 2NT = 3--3 majors; invitational strength.
		\item 3$\clubsuit$/$\diamondsuit$ = 3--3 majors, GF, values in the minor.
		\item 3NT=3--3 majors, GF.
		\end{itemize}
	\end{itemize}

\subsection{After 1NT is doubled for penalties}

We play a modified form of suction as the escape after 1NT-X.

\begin{itemize}
\item Pass forces XX, to play or weak with a two-suiter, not the reds or the majors
	\begin{itemize}
	\item XX forced
		\begin{itemize}
		\item Pass with a strong hand, to play
		\item bid 4 card suits up the line
		\end{itemize}
	\end{itemize}
\item XX forces 2$\clubsuit$, weak with clubs or the reds
	\begin{itemize}
	\item 2$\clubsuit$ forced
		\begin{itemize}
		\item Pass with clubs
		\item 2$\diamondsuit$ with $\diamondsuit$\&$\heartsuit$
		\end{itemize}
	\end{itemize}
\item 2$\clubsuit$ forces 2$\diamondsuit$, weak with diamonds or the majors
	\begin{itemize}
	\item Pass with 5 clubs
	\item 2$\diamondsuit$ with all other hands
		\begin{itemize}
		\item Pass with diamonds
		\item 2$\heartsuit$ with the majors
		\end{itemize}
	\end{itemize}
\item 2$\diamondsuit$ forces 2$\heartsuit$, weak with hearts
	\begin{itemize}
	\item Pass with 5 diamonds
	\item 2$\heartsuit$ all other hands
	\end{itemize}
\item 2$\heartsuit$ forces 2$\spadesuit$, weak with spades
	\begin{itemize}
	\item Pass with 5 hearts
	\item 2$\spadesuit$ all other hands
	\end{itemize}
\end{itemize}

\subsection{After direct overcalls of 1NT}

Lebensohl:

\begin{itemize}
\item suits at the 2 level: to play
\item suits at the 3 level: GF, natural
\item direct cue: staymanic, denies a stop
\item 3N: natural, denies a stop
\item 2N: puppet to 3$\clubsuit$
	\begin{itemize}
	\item 3$\clubsuit$ forced
		\begin{itemize}
		\item suits below the cue: to play
		\item suits above the cue: invitational, natural
		\item cue: staymanic, promises a stop
		\item 3N: natural, promises a stop
		\end{itemize}
	\end{itemize}
\item Double is for penalties
\item Double of a natural 2$\clubsuit$ overcall is stayman showing a club stop (optional)
\end{itemize}

\section{Continuations after 2NT}

\begin{itemize}
\item 3$\clubsuit$: advanced 5-card puppet Stayman
	\begin{itemize}
   \item 3$\heartsuit$/$\spadesuit$ shows 5 cards.
   \item 3$\diamondsuit$ promises 4 hearts or 3--4 spades
		\begin{itemize}
      \item 3$\heartsuit$: 0--3 hearts, 0--4 spades
			\begin{itemize}
         \item 3$\spadesuit$: 4 spades (Responder bids 4$\spadesuit$ or 3N)
         \item 3NT: 3 spades or 4 hearts
			\end{itemize}
      \item 3$\spadesuit$: 4 hearts, 0--3 or 5 spades
			\begin{itemize}
         \item 3NT: 3/4 spades (Responder bids 4$\spadesuit$ if 5/4)
         \item 4$\clubsuit$/$\diamondsuit$: 4 hearts, Swiss
         \item 4$\heartsuit$: to play.
			\end{itemize}
      \item 3N: 4--4 majors
			\begin{itemize}
			\item Pass: with 3 spades
         \item 4$\heartsuit$: with 4 hearts.
         \item 4$\spadesuit$: with 4 spades.
			\end{itemize}
		\end{itemize}
   \item 3NT denies 4 hearts or 3 spades
	\end{itemize}

\item 3$\diamondsuit$/3$\heartsuit$: transfers.  Superaccept with any 4-card fit; note that 3NT is a
superaccept that still allows the use of Swiss.

\item 3$\spadesuit$: minor-suit Stayman.  Asks for a 4- or 5-card minor.
	\begin{itemize}
   \item 3NT: no 4-card minor.
   \item 4$\clubsuit$/4$\diamondsuit$: 4 cards.
   \item 4$\heartsuit$/4$\spadesuit$: 5 cards in the corresponding minor.
	\end{itemize}

\item 4$\clubsuit$/$\diamondsuit$: Swiss
\item 4N: Quantitative
\end{itemize}

\section{Encrypted bidding}

Once responder makes an encrypted raise, opener can either accept or decline
the key.  To decline, simply rebid the suit (cheaply with a minimum; jump if
maximum.  Use common sense.).

If the key is accepted, subsequent bids are either two-way or three-way, depending on the key.

\subsection{Two-way encryption}

This is far more common, as it requires responder and opener to have one each
of the trump AK. Opener's rebid shows a useful side-suit if holding the trump
K (0--1 losers) or a potential weakness if holding the trump A (2--3 losers).
Responder can then either place the contract, cue-bid (either showing with the
K or denying with the A), or use Blackwood.

Blackwood here is Roman Keycard, but it's modified because there are only three keycards
still of interest.

\subsection{Three-way encryption}

Requires AKQ of trumps between the two hands, and enough values for a
game-forcing raise in responder's hand.  Rare!

Opener's rebid shows a first-round control; the location of this control is
encrypted.  We use the concept of the shift here: a shift of 1 indicates the
next suit up ($\clubsuit$ from NT or $\spadesuit$, $\diamondsuit$ from $\clubsuit$, $\heartsuit$ from $\diamondsuit$, $\spadesuit$ from $\heartsuit$).  A shift of 2
indicates the next suit after that, and a shift of 3 shows the final suit.
Each cue therefore says nothing about control in the suit bid.

The sum of the shift and the HCP of opener's honour is 5.  Opener, holding a $\diamondsuit$
control, therefore bids $\clubsuit$ with the trump A, $\spadesuit$/NT with the trump K, and $\heartsuit$ with
the trump Q.

Of course, Blackwood goes funny here too.  You only need to show 0--3 keycards,
and the location of the Q is known, so 0--1--2--3 would seem a sensible set
of responses.

\section{Slam conventions}

\subsection{Roman key-card Blackwood (3014)}

Bids at the 5 level are exclusion blackwood. Responses are for keycards outside
the suit bid, starting at the next bid.

\subsection{First-round-control showing cues}

Once a suit is agreed and a GF is established, new suits are cues.

\subsection{General Swiss}

4$\clubsuit$/4$\diamondsuit$, once a game-force is established and suit agreed, are slam tries, based
on control points

Control points are 2 for an ace, 1 for a king/singleton, and 1 for the queen of
trumps.  There are 13 available from honours; it is possible to substitute
outside singletons for kings, but care must be taken to avoid double-counting.
11 CPs are necessary for a small slam; 13 for a grand.

\begin{itemize}
\item 4$\clubsuit$: 4 or 6 (exceptionally 8) control points)
	\begin{itemize}
   \item trump suit: 0--4 CPs
   \item 4$\diamondsuit$: 5/6 CP, relay
		\begin{itemize}
      \item trump suit: 4 CPs
      \item suits: 6 CP, lowest king/singleton
		\end{itemize}
   \item suits: 7 or 9 CP, lowest king/singleton
   \item small slam: 8 CP
   \item grand slam: 10 CP
	\end{itemize}

\item 4$\diamondsuit$: 5 or 7 (exceptionally 9) control points)
	\begin{itemize}
   \item trump suit: 0--3 CPs
   \item 4$\heartsuit$: 4/5 CP, relay
		\begin{itemize}
      \item trump suit: 5 CPs
      \item suits: 7 CP, lowest king/singleton
		\end{itemize}
   \item suits: 6 or 8 CP, lowest king/singleton
   \item small slam: 7 CP
   \item grand slam: 9 CP
	\end{itemize}
\end{itemize}

If you find duplication in kings/singletons subtract one CP and sign off at the
appropriate level. If your partner signs off and you have 2 unshown CPs so far,
raise one level.

If the hand bidding Swiss has shown a GF, add 2 to all numbers. If it has shown
a weak hand (opening at the 2 level or giving a 0--7 response to 1$\diamondsuit$), subtract 2
from all numbers.

\section{Competitive bidding}

We tried an artificial overcall scheme but found it couldn't cope with a number of situations.
Therefore, we have a fairly natural overcall style.

Simple overcalls are constructive and natural. In response, if 1N is available,
then 2N is a crypto raise. In all cases a responsive cue is a good raise and direct
raises are preemptive.

Double is standard takeout (or a strong hand).

Direct 1NT is 15-17 balanced with a stop, protective 1NT is 11-14 with a stop. 
After these ``Continuations after 1NT'' apply. 

Jump overcalls are weak; jump overcalls of a major suit over a minor-suit
opening may include the other minor (possible canape).  Similarly, jump
overcalls of 3$\clubsuit$ over a 1 of a major opening could include $\diamondsuit$.

In pass-out seat jump overcalls are intermediate. Also, after the sequence
(Pass)-Pass-(1N).

Jump cuebids are stopper-asking for 3NT; they promise a long running suit.  An
overcall of 3NT promises a long running suit and a stopper.

Simple cuebids are Michaels showing at least 5--5 in the majors (over a minor)
or the other major and either minor (over a major). In all cases they are
either weak or strong. Similarly, 2NT over a 1 level bid is unusual, showing
5--5 or better in the minors (over a major) or the other minor and either major
(over a minor). Again, it's weak or strong.

After Michaels or Unusual, bidding any of the suits which could have been shown
is pass or correct at that level. Overcaller then corrects to the next highest
suit (or passes or raises) with a weak hand. With a strong hand he breaks to
another suit. Of the two suits which are not 'pass or correct', re-cueing is
strong and agreeing advancer's preferred suit and the other suit is strong with
the remaining option.

For example:
\begin{itemize}
\item (1$\heartsuit$)-2$\heartsuit$-(P)
	\begin{itemize}
	\item 3$\clubsuit$-3$\diamondsuit$ is weak with spades and diamonds
	\item 3$\clubsuit$-3$\heartsuit$ is strong with spades and clubs
	\item 3$\clubsuit$-3$\spadesuit$ is strong with spades and diamonds
	\item 3$\diamondsuit$-3$\heartsuit$ is strong with spades and diamonds
	\item 3$\diamondsuit$-3$\spadesuit$ is weak with spades and clubs
	\item 3$\diamondsuit$-4$\clubsuit$ is strong with spades and clubs
	\item 2$\spadesuit$-3$\clubsuit$ is strong with spades and clubs
	\item 2$\spadesuit$-3$\diamondsuit$ is strong with spades and diamonds
	\end{itemize}
\end{itemize}



\subsection{Defences}

These are mostly all based on suction:

\begin{itemize}
\item clubs: diamonds or the majors
\item diamonds: hearts or spades and clubs (or diamonds)
\item hearts: spades or the minors
\item spades: clubs or hearts and diamonds (or clubs)
\end{itemize}

\subsubsection{Natural 1NT}

Suction, constructive values. Double is for penalties and 2N shows a
non-touching two suiter.

\subsubsection{Artificial strong bids}

Suction, weak. Double shows the suction bid in their suit and 2N shows a
non-touching two suiter.

\subsubsection{Phoney/Prepared suits}

Suction, constructive values. Double shows the suction bid in their suit and 2N
is natural.

\subsubsection{Multi 2$\diamondsuit$}

Dixon:

\begin{itemize}
\item X: 16+
\item 2$\heartsuit$/$\spadesuit$: takeout of the other major
\item 2N: 16-18 bal, stops in both majors.
	\begin{itemize}
	\item Responses as ``Continuations after 2NT"
	\end{itemize}
\item 3x: intermediate, natural
\end{itemize}

\subsubsection{Natural weak openings}

Double is for takeout up to 4$\heartsuit$. After a weak two and a takeout double, there
are Lebensohl responses. 4N is takeout over 4$\spadesuit$ and a big two suiter otherwise.

Over a weak two,  2NT is natural showing 16-18 and ``Continuations after 2NT''
apply.

\section{Lightner Doubles}

Doubles of freely bid slams are Lightner, asking for the lead of dummy's first shown suit.


\section{Dealing with interference}

\subsection{Direct interference over 1$\clubsuit$}

\begin{itemize}
\item 1$\clubsuit$-(X) (penalty interest (showing 16+):
	\begin{itemize}
   \item Redouble creates a game-force, and indicates that the double may not have been 
      a winning action.
   \item Pass forces redouble; either to play or with a two-suiter not including clubs.
   \item 1$\diamondsuit$/1$\heartsuit$ are transfers to 1$\heartsuit$/1$\spadesuit$.  Note that opener may pass these if holding a 5-card 
      suit.
	\end{itemize}
\item 1$\clubsuit$-(X) (showing $\clubsuit$):
	\begin{itemize}
   \item Redouble creates a game-force; passes are now forcing and doubles are for penalty.
   \item Pass shows a weak hand, with nowhere in particular to play.
   \item Suit bids are natural, 10--15, and promise 4 cards.
   \item 1NT shows a balanced hand, 10--15, with a club stop.
	\end{itemize}
\item 1$\clubsuit$-(X) (takeout):
	\begin{itemize}
   \item Redouble still creates a forcing pass situation.
   \item Ask opponents ``takeout of what, exactly?".
   \item 1NT cannot show a stopper in any particular suit.
	\end{itemize}
\item 1$\clubsuit$-(natural bid):
	\begin{itemize}
   \item Responder's double creates a game-force.
   \item 1NT promises a stop; 10--15 balanced.
   \item Suit bids are natural, 10--15.
   \item Exception: cheapest bid (or 2$\spadesuit$ over 1$\spadesuit$) shows a 12-15 balanced hand with no stop.  System is on, including Keri, transfers etc.  After 1$\clubsuit$-(1$\diamondsuit$)-1$\heartsuit$, for example, 1$\spadesuit$ is a range enquiry or transfer to $\clubsuit$.
   \item Pass is a weak hand with nowhere to go (potentially including an 8--11 balanced hand with no stop).
	\end{itemize}
\end{itemize}

\subsection{Direct interference over 1$\diamondsuit$}

\begin{itemize}
\item 1$\diamondsuit$-(X):
	\begin{itemize}
   \item Redouble with good 8+ hands.
   \item Pass with weak hands.
   \item Suit bids are natural, weak, 6-card suit.
	\end{itemize}
\item 1$\diamondsuit$-(bid):
	\begin{itemize}
   \item Pass with weak hands.
   \item Double to establish a game-force (and forcing passes, etc.).
		\begin{itemize}
      \item Suction rebids are on; immediate NT bid denies a stop.
      \item To show a stop, Suction-force into opponents' suit then bid NT.
		\end{itemize}
   \item Suit bids are natural, weak, 6-card suit.
	\end{itemize}
\item 1$\diamondsuit$-(spade for a laugh):
	\begin{itemize}
   \item Double is for penalties
   \item Passes are forcing
	\end{itemize}
\end{itemize}

\subsection{Balancing-seat interference (e.g. 1$\diamondsuit$-(P)-1$\spadesuit$-(2$\clubsuit$))}

\begin{itemize}
\item 1$\clubsuit$-(P)-1$\spadesuit$-(anything):
\item 1$\diamondsuit$-(P)-1$\spadesuit$-(anything):
\begin{itemize}
   \item Passes are forcing
   \item NT bids promise a stopper.
   \item Suction-style rebids are on if 3$\clubsuit$ is a sufficient bid (1$\diamondsuit$ opener)
   \item Pass indicates a balanced hand without a stopper in opponents' suit.
	\begin{itemize}
      \item Responder's NT bids here are Lebensohl style
      \item There are no weak hands to show, so the only difference is in showing/denying a stopper.
	\end{itemize}
   \item Double is penalties.
\end{itemize}

\item 1$\clubsuit$-(P)-1$\diamondsuit$-(anything): it's a good idea to subside quietly.  

\item 1$\clubsuit$-(P)-1$\heartsuit$-(anything): passes are non-forcing. 
\end{itemize}

\subsection{Interference over a 2-level opening}

A few general principles here:

\begin{itemize}
\item 2x-(X):
   \begin{itemize}
      \item XX: forcing enquiry (even over further interference)
      \item Pass: no desire to play opposite a weak hand
      \item Relay: some tolerence for the weak options
      \item Enquiries: forcing unless oppo bid
   \end{itemize}
\item 2x-(2y):
   \begin{itemize}
      \item X: optional (pass if it's your suit, pull otherwise)
      \item \ldots
   \end{itemize}
\item Opener's rebids facing passed responder:
   \begin{itemize}
   \item Suit from a weak option at the lowest level: decent weak option
   \item X/denied suit/NT: strong option (X is optional)
   \end{itemize}
\end{itemize}

\section{Carding}

\subsection{Leads}

Leads are standard (2nd from bad suits, 4th from an honour, top of sequences),
except that they may vary based on signal requested (see below).

The exception is that we lead top of bad suits against no trumps.

After the initial lead we lead high from bad suits and low from good suits.

\subsection{Signals}

We play two forms of signalling. On partner's lead we play reverse attitude
(low encouraging, high discouraging), except if the initial lead was an odd
card. In that case we play Prism signals in that suit (see below). We also play
Prism signals when trumps are lead by either side, or if there is a suitable
obvious running suit in no trumps.

Leads to ruffs and other obvious situations may be McKenny.

\subsection{Discards}

Discards are Italian style, so an odd card is encouraging in the suit discarded
and an even card is McKenny. Thus, low even cards ask for the lower of the
outside suits and a high even card asks for the higher of the two outside
suits.

\subsection{Prism}

Prism is a signalling system based on the shape of the whole hand. An initial
hand of 13 cards will, perforce, have either one suit which is odd (and three
even) or one suit which is even (and three odd). This denotes the parity of the
hand.

The first card in a prism sequence will give the original parity of the hand.
Low for odd, high for even. Second and subsequent cards will indicate the suit
which is unique in parity. Take the spade suit 2 3 6, the following sequences indicate
the following parities and suits:

\begin{tabular}{llll}
\bf order & \bf parity & \bf outside suit & \bf relative to spades \\
\hline
2 3 6 & odd    & higher       & hearts \\
2 6 3 & odd    & middle       & diamonds \\
3 6 2 & odd    & lower        & clubs \\
3 2 6 & even   & higher       & hearts \\
6 2 3 & even   & middle       & diamonds \\
6 3 2 & even   & lower        & clubs \\
\end{tabular}

This information can be combined with the other defender's hand and dummy to
deduce the shape of declarer's hand as follows. Add together the length of your
suits and dummy's, then subtract each from 13. This will give you either four
even numbers, four odd numbers or two of each.

\begin{itemize}
\item If all four are even then declarer's unique suit is the same as partners and the same parity.
\item If all four are odd then declarer's unique suit is the same as partners and the other parity.
\item If partner's suit is in the even pair, then declarer's suit is the other one in that pair and the other parity.
\item If partner's suit is in the odd pair, then declarer's suit is the other one in that pair and the same parity.
\end{itemize}

\end{document}
