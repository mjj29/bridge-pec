\documentclass[a4paper,14pt]{extarticle}

\usepackage{palatino}
\usepackage{fullpage}

\usepackage[
  bookmarks,
  bookmarksopen,
  bookmarksnumbered=false,
  colorlinks,
  pdfpagemode=None,
  urlcolor=black,
  citecolor=black,
  linkcolor=black,
  pagecolor=black,
  plainpages=false,
  pdfpagelabels,
  pdftitle={System notes on Pascal's Encrypted Club},
  pdfauthor={Matthew Johnson and Henry Lockwood}]{hyperref}

\author{Matthew Johnson\\Henry Lockwood}
\title{System notes on Pascal}


\usepackage{ifthen}
\usepackage{color}
\usepackage[usenames,dvipsnames]{xcolor}
\usepackage{txfonts}
 
\newcommand{\newevenside}{
	\ifthenelse{\isodd{\thepage}}{\newpage}{
	\newpage
	\textcolor{white}{placeholder} 
	\thispagestyle{empty}
	\newpage
	}
}


\newcommand{\orange}[1]{[BB #1]}
\newcommand{\xref}[1]{[\ref{#1}]}

\newcommand{\hearts}{\textcolor{Red}{$\varheartsuit$}~}
\newcommand{\diamonds}{\textcolor{Red}{$\vardiamondsuit$}~}
\newcommand{\spades}{$\spadesuit$~}
\newcommand{\clubs}{$\clubsuit$~}




\begin{document}

\maketitle
\tableofcontents

\section*{Notes from system card}
\begin{tabular*}{\textwidth}{ll@{\extracolsep{\fill}}r}
1. & 2NT good raise & \pageref{note:1} \\
2. & Jump fit & \pageref{note:2} \\
3. & 3NT good raise & \pageref{note:3} \\
6. & splinter & \pageref{note:6} \\
7. & 2\clubs 5-card puppet Keri & \pageref{note:7} \\
8. & 3\clubs 5-card puppet Stayman & \pageref{note:8} \\
9. & Responses to a natural 1NT & \pageref{note:9} \\
10. & Responses to a natural 2NT & \pageref{note:10} \\
11. & UCB & \pageref{note:11} \\
12. & Lebensohl & \pageref{note:12} \\
13. & Responses to 1\clubs & \pageref{note:13} \\
14. & Responses to 1\diamonds& \pageref{note:14} \\
15. & Wriggle from 9-15 1N-(x) & \pageref{note:15} \\
18. & Responses to 2x & \pageref{note:18} \\
19. & General Swiss & \pageref{note:19} \\
20. & Suction wriggle & \pageref{note:20} \\
21. & Aardvark with Halmic redouble & \pageref{note:21} \\
22. & Roman Key-Quant Gerber & \pageref{note:22} \\
23. & Viscount & \pageref{note:23} \\
\end{tabular*}

\newpage

\section{Administriva}
\label{sec:admin}

\subsection{Notation}
\label{sec:notation}

References to the EBU Blue Book will be given as \orange{7A1}. References
elsewhere in this document will be given as \xref{sec:resp:2level}.

\subsection{System regulation}

Allowed systems are determined by the Regulating Authority for a given event.
This system is primarily played under the auspices of the English Bridge Union
who have recently split the highest level into Level 4 and Level 5. The
majority of this system is allowed at Level 4 \orange{7}, which is the
standard tournament level in England. The major suit openings are only allowed
at Level 5 \orange{9} (given in {\it italics}) and hence alternatives are
provided for Level 4 events (given in {\color{CadetBlue}grey}).  Events not
directly run by the EBU may have other restrictions.  References to the Blue
Book sections allowing each convention at Level 4 or 5 are given throughout the
document.

\subsubsection{WBF classification}

Under the WBF classification system this is a Red system with Brown Sticker
Conventions. Thus, it is not legal in Category 3 WBF events and in Category 2
events the card with suggested defences and appropriate forms must be submitted
in advance. Suggested defences are given in Appendix \ref{appx:defences}.

\subsubsection{WBF Category 3 variation}

Modifications to the system to be permitted at WBF Category 3 events (as a Red
system) are provided in Section \ref{sec:cat3}.

\newpage

\section{System description}
\label{sec:system}

\subsection{Philosophy}
\label{sec:philosophy}

\begin{itemize}
\item lots of multi-way bids which show very distinct hands
\item establish strength first
\item establish shape (balanced/unbalanced) second
\item then find a contract
\item single-suited hands are all 6+ cards and deny a side 4-card suit, although
this may be bent with 4 small cards in a minor.
\end{itemize}

\subsection{Hand evaluation}

The descriptions below of particular bids will generally be in terms of point
ranges. These are not a hard and fast rule, however. We will try and make the
bid which describes the hand best. This may mean small deviations of strength, 
particularly with extra shape. No-trump ranges are the least likely to vary and
preemptive actions most likely.

Opening bids at the 1 level are 10--15. The lower limit is usually governed by
the rule of 19 as much as the actual point count and some good 15 point hands
may be upgraded to a 1\diamonds opening. Losing trick count is also 
useful for deciding when to upgrade. Upgraded hands will, obviously, meet the
extended rule of 25 in all cases. \orange{5C3}

Weak opening and overcalling hands have nominal point counts associated with
them but are likely to be wider ranging in third and weaker at favourable
vulnerability. Fourth seat openings which are weak in other seats will be 
intermediate.

In order to determine whether a hand is game forcing (for example to open a
distributional hand at the two or three level) we generally
rely on losing trick count. Three or fewer losing tricks is enough to be game
forcing, occasionally 4 if there are also 9 clear-cut tricks. This is just a
guide, however, so might not strictly apply to all hands. The hands will meet
the extended rule of 25 in all cases.

\newpage


\subsection{Openings}
\label{sec:openings}

\subsubsection{1 level}
\label{sec:open:1level}

Responses in Section \ref{sec:resp:1level}.

\paragraph{1\clubs}

One of the following \orange{11C11}:

\begin{itemize}
\item (1\textsuperscript{st}/2\textsuperscript{nd} non-vul) 12--15 balanced (4333, 4432, 5332). May occasionally include some semi-balanced hands esp. 5422.
\item (1\textsuperscript{st}/2\textsuperscript{nd} vul) 11--13 balanced (4333, 4432, 5332). May occasionally include some semi-balanced hands esp. 5422.

\item 10--15, not balanced, no 5-card major and at most one 4-card major
\end{itemize}

\paragraph{1\diamonds}

16+, not game forcing \orange{11C12}. 
\begin{itemize}
\item If balanced or semi-balanced then 16--19 (17--19 1/2 Vul). 
\item If a two suiter (5422 or longer) or three suiter (4441 / 5440) then 16--GF. 
\item If a single suiter (6332 or longer) then 16--GF. (16--GF here is 'less than a 3-loser hand').
\end{itemize}

{\it

\paragraph{1\hearts~(Level 5)}

\begin{itemize}
\item 10--15 points, not balanced 4+ spades, may be longer hearts or a minor (possible canape), but will always have 5 spades or 4 hearts \orange{11R}.
\end{itemize}

}

{\color{CadetBlue}

\paragraph{1\hearts~(Level 4)}

\begin{itemize}
\item 10--15 points, not balanced, 4+ hearts, may contain a longer minor (possible canape) \orange{11C15}.
\end{itemize}
}

{\it

\paragraph{1\spades~(Level 5)}

\begin{itemize}
\item 10--15 points, not balanced. 5+ hearts, 0-3 spades \orange{11R}, may have a longer minor (possible canape).
\end{itemize}

}

{\color{CadetBlue}

\paragraph{1\spades~(Level 4)}

\begin{itemize}
\item 10--15 points, not balanced. 5+ spades \orange{11C15}, may have a longer minor (possible canape).
\end{itemize}
}

\paragraph{1NT}

Varies depending upon position and vulnerability \orange{10A8}

First or second seat non-vulnerable:
\begin{itemize}
\item 9--11 balanced \orange{11F3}.
\end{itemize}

Third seat non-vulnerable:
\begin{itemize}
\item 9--15 balanced \orange{11F3}.
\end{itemize}

First or second seat vulnerable:
\begin{itemize}
\item 14--16 balanced \orange{11F3}.
\end{itemize}

All other positions and vulnerabilities:
\begin{itemize}
\item 12--15 balanced \orange{11F3}.
\end{itemize}

For responses see Section~\ref{sec:resp:1nnat}.

\newpage 

\subsubsection{2 level}
\label{sec:open:2level}

Responses in Section \ref{sec:resp:2level}. Range in HCP for each weak option
is given, but may be wider ranging in 3rd or favourable vulnerability and will
be intermediate in 4th position (12--16).

\paragraph{2\clubs}
One of the following \orange{11G10}:
\begin{itemize}
\item Weak (5--9) with \diamonds (usually 6 cards)
\item Weak (5--9) with \hearts \& \spades, 4--4 or better (usually 5--4)
\item 20--23 balanced
\item Game forcing (3 losers) with \hearts \& \spades, 4--4 or better (usually 5--4)
\end{itemize}

\paragraph{2\diamonds}
One of the following \orange{11G10}:
\begin{itemize}
\item Weak (5--9) with \hearts (usually 6 cards)
\item Weak (5--9) with \spades \& \clubs, 5--4 or better 
\item Game forcing (3 losers) with \spades \& \clubs, 4--4 or better (usually 5--4)
\end{itemize}

\paragraph{2\hearts}
One of the following \orange{11G10}:
\begin{itemize}
\item Weak (5--9) with \spades (usually 6 cards)
\item Weak (5--9) with \clubs \& \diamonds, 4--4 or better (usually 5--4)
\item Game forcing (3 losers) with \clubs \& \diamonds, 4--4 or better (usually 5--4)
\end{itemize}

\paragraph{2\spades}
One of the following \orange{11G10}:
\begin{itemize}
\item Weak (5--9) with \clubs (usually 7 cards)
\item Weak (5--9) with \hearts \& one of the minors, 4--4 or better (usually 5--4)
\item Game forcing (3 losers) with \hearts \& one of the minors, 4--4 or better (usually 5--4)
\end{itemize}

\paragraph{2NT}
One of the following \orange{11H8}:
\begin{itemize}
\item Weak (5--9) with \spades \& \diamonds, 4--4 or better (usually 5--4)
\item Game forcing (3 losers) with \spades \& \diamonds, 4--4 or better (usually 5--4)
\item 24+ (Game forcing) balanced hands
\end{itemize}

\subsubsection{Higher openings}
\label{sec:open:higher}

Responses in Section \ref{sec:resp:higher}.

\paragraph{3 level}

Three level bids are transfers. Either weak preempts or a 3-loser hand.
Typically seven cards, they may be occasionally a good six (particularly at
favourable vulnerability) or a bad eight.

\begin{itemize}
\item 3\clubs - 7 card preempt or GF in diamonds \orange{11J8(a)}
\item 3\diamonds - 7 card preempt or GF in hearts \orange{11J8(a)}
\item 3\hearts - 7 card preempt or GF in spades \orange{11J8(a)}
\end{itemize}

\paragraph{3\spades}

Gambling 3NT~\orange{11J8(d)}. 

Shows a solid seven or eight card minor suit and nothing outside better than a
queen.  Solid in this case is defined as 60\% likely to run given all
distributions of cards in the other hands. The following suits are all
considered to be solid:

\begin{itemize}
\item AKQJxxx
\item AKQxxxx
\item AKxxxxxx
\end{itemize}

\paragraph{3NT}

Shows a 4-level preempt in either minor or a 3 loser hand with
clubs~\orange{11K3(d)}. Usually eight cards. 

\paragraph{4-level suit openings}

We play Namyats, so:

\begin{itemize}
\item 4\clubs - Strong 4-level bid in hearts~\orange{11L3}
\item 4\diamonds - Strong 4-level bid in spades~\orange{11L3}
\item 4\hearts - Purely preemptive 4-level bid in hearts
\item 4\spades - Purely preemptive 4-level bid in spades
\end{itemize}

Strong is defined as an 8--9 playing trick hand.

\paragraph{4NT}

Asks for specific aces~\orange{11L1}.

Will be bid with a hand missing at most two aces, or with a void. It must be happy playing at the six level opposite some responses and the seven level opposite some others.

\paragraph{5 of a major}

Shows an 11 trick hand missing two of the top three trump honours. Raise one level per trump honour.




\subsection{Responses \& rebids}
\label{sec:responses}

\subsubsection{1 level}
\label{sec:resp:1level}

\begin{itemize}
\label{note:13}
\item 1\clubs (3\textsuperscript{rd}/4\textsuperscript{th} Non-Vulnerable):
	\begin{itemize}
   \item 1\diamonds = 0--5, no 6-card suit
		\begin{itemize}
      \item suits = unbalanced, possible canape
      \item 1NT = 10--15, 5--4 minors
		\end{itemize}
   \item 1\hearts = 6--8, no 6-card suit
		\begin{itemize}
      \item 1\spades = any 10-11
			\begin{itemize}
			\item 1NT = balanced, no interest in game. Responses are weak takeout.
			\item suits = natural, NF
			\end{itemize}
      \item suits = unbalanced, possible canape
      \item 1NT = 12--15, 5--4 minors
		\end{itemize}
   \item 1\spades/1N/2x = two-under transfer weak jump shift 6+ card suit two-above
		\begin{itemize}
		\item Complete the transfer with nothing more to say
		\item intervening bid = extras, 4+ support
		\end{itemize}
	\end{itemize}

\newpage

\item 1\clubs (3\textsuperscript{rd}/4\textsuperscript{th} Vulnerable):
	\begin{itemize}
   \item 1\diamonds = 0--8, no 6-card suit
		\begin{itemize}
      \item suits = unbalanced, possible canape
      \item 1NT = 10--15, 5--4 minors
		\end{itemize}
   \item 1\hearts = 9-11
		\begin{itemize}
      \item 1\spades = any 10-11
			\begin{itemize}
			\item 1NT = balanced, no interest in game. Responses are weak takeout.
			\item suits = natural, NF
			\end{itemize}
      \item 1NT = 12--15, 5--4 minors
      \item major suit bids 12-15 unbal and promise exactly 4; either 4-4-4-1
            with singleton other major or has a longer minor (as there's no other
            hand shape that fits the opening and has exactly one 4-card major)
      \item minor suit bids deny a 4-card major and are 12-15 unbal.
		\end{itemize}
   \item 1\spades/1N/2x = two-under transfer weak jump shift 6+ card suit two-above
		\begin{itemize}
		\item Complete the transfer with nothing more to say
		\item intervening bid = extras, 4+ support
		\end{itemize}
	\end{itemize}

\newpage

\item 1\clubs (1\textsuperscript{st}/2\textsuperscript{nd} Any vulnerability):
	\begin{itemize}
   \item 1\diamonds = 0--8, no 6-card suit
		\begin{itemize}
      \item 1NT = 11--13 (vul) 12--15 (non-vul) balanced; responses weak takeout.
      \item Major suit bids 12-15 unbal and promise exactly 4; either 4-4-4-1
            with singleton other major or has a longer minor (as there's no other
            hand shape that fits the opening and has exactly one 4-card major)
      \item Minor suit bids deny a 4-card major.
		\end{itemize}
   \item 1\hearts = 9--15 any
		\begin{itemize}
      \item 1\spades = any 10-11
			\begin{itemize}
			\item 1NT = balanced, no interest in game. Responses are weak takeout.
			\item suits = natural, F1
			\item 2N = balanced, 13-15. Responses see ``Continuations after 2NT" \xref{sec:resp:2n}
			\end{itemize}
      \item 1NT = 12--13 (vul) 12--15 (non-vul) balanced. Responses see ``Continuations after 1NT" \xref{sec:resp:1n}.
      \item Suits = natural, NF, not canape
		\end{itemize}
   \item 1\spades = 15+ any, game-forcing.
		\begin{itemize}
      \item Major suit bids promise exactly 4; either 4-4-4-1 with singleton other
         major or has a longer minor (as there's no other hand shape that fits the
         opening and has exactly one 4-card major)
      \item 1NT = 11--13 (vul) 12--15 (non-vul) balanced. Responses see ``Continuations after 1NT" \xref{sec:resp:1n}.
      \item Major suit bids 12-15 unbal and promise exactly 4; either 4-4-4-1
            with singleton other major or has a longer minor (as there's no other
            hand shape that fits the opening and has exactly one 4-card major)
      \item Minor suit bids deny a 4-card major.
		\end{itemize}
   \item 1N/2x = transfer jump shift 6+ card suit above, 0--8 or slam try
		\begin{itemize}
		\item Complete the transfer with nothing more to say opposite the weak option
		\item 2N = extra values, nothing in the transfer suit
		\item Jump-completion = good support, invitational
		\item new suits = extras, tolerance, side suit
		\end{itemize}
	\end{itemize}

\newpage

\label{note:14}
\item 1\diamonds:
	\begin{itemize}
   \item 1\hearts = 0--7 (not AK suited), no 6-card suit.
		\begin{itemize}
		\item Suit bids are natural
      \item 1NT rebid = 17--19 (vul) 16--19 (non-vul) balanced. Responses see ``Continuations after 1NT" \xref{sec:resp:1n}.
      \item Jump suit rebids are almost-GF, semi-solid single-suited.
		\end{itemize}
   \item 1\spades = 8+, game-forcing.
		\begin{itemize}
      \item 1NT rebid = 17--19 (vul) 16--19 (non-vul) balanced. Responses see ``Continuations after 1NT" \xref{sec:resp:1n}.
      \item Simple rebids are suction-style, showing either the suit above
         (single-suited), the other two or all three, 16--21 HCP.
			\begin{itemize}
			\item \clubs = diamonds or the majors or \diamonds\hearts\spades
			\item \diamonds = hearts or spades and clubs or \hearts\spades\clubs
			\item \hearts = spades or the minors or \spades\clubs\diamonds
			\item \spades = clubs or hearts and diamonds or \clubs\hearts\diamonds
			\item 2NT = non-touching two suiter
			\end{itemize}
		\item Could be 3-suited, short in the suit bid. These are shown: 
			\begin{itemize}
         \item 2\clubs-2\diamonds-3\clubs: short clubs.
			\item 2\diamonds-2\hearts-2\diamonds: short diamonds. 
			\item 2\hearts-2\spades-3\hearts: short hearts.
         \item 2\spades-2NT-3\spades: short spades.
			\end{itemize}
		\item Continuations:
			\begin{itemize}
			\item Responder will bid the next suit (relay). Opener then rebids 2NT with the
				single-suited hand, the longer suit with the two-suited option or as above with
				the three-suiter. After 2\spades - 2NT, 3\clubs asks which option (longer suit or 3NT with longer clubs)
				and other suits are swiss-or-correct below game or pass-or-correct above.
			\item After showing a 3-suiter, responder bids 3NT to play or a suit at the lowest level
				to agree the suit and make a slam try, or a suit at game to play.
			\item After GF+ suit agreement with a 3-suiter if Swiss is still available then 
				4\clubs/\diamonds is Swiss \xref{sec:swiss}, otherwise suit bids are
				first round controls. 4NT is RKCB \xref{sec:rkcb}.
			\item After 1\diamonds-1x and showing suit(s) by opener, bidding 4\clubs/\diamonds
				is Swiss \xref{sec:swiss} agreeing opener's most recently bid suit, bidding one of the other suits below 
				game is slam inviting asking opener to bid swiss for that suit. Other suits below 3NT are an enquiry
				either wanting a stop or to find out openers' longer suit.
			\end{itemize}
      \item Jump suit rebids are almost-GF, semi-solid single-suited.
		\end{itemize}
   \item 1N/2x = transfer weak jump shift 6+ card suit above, 0--7 or slam try
		\begin{itemize}
		\item Complete the transfer with nothing more to say opposite the weak option
		\item 2N = extra values, nothing in the transfer suit
		\item Jump-completion = good support, invitational
		\item new suits = extras, tolerance, side suit
		\item jump new suit = nearly-GF single suiter
		\end{itemize}
	\end{itemize}

{\it 
\item 1\hearts (Level 5):
	\begin{itemize}
\label{note:5}
   \item 1\spades = forcing enquiry.
		\begin{itemize}
      \item 1NT = both majors, minimum. Continuations as after artificial 1NT opening \xref{sec:resp:1nart}.
      \item 2\clubs = 5+ spades and 4+ clubs, some extras
      \item 2\diamonds = 5+ spades and 4+ diamonds, some extras
      \item 2\hearts = both majors, maximum
		\item 2\spades = 5+ spades, minimum
		\item 2NT = 6+ spades, no outside suit, maximum
		\end{itemize}
   \item 1NT = non-forcing, any negative without spades
\label{note:4}
   \item 2\clubs/\diamonds = natural, forcing.
   \item 2\hearts = good raise to 4 spades
		\begin{itemize}
		\item Opener bids long suit trials below the trump suit
		\item Cues or Swiss~\xref{sec:swiss} above the trump suit
		\end{itemize}
   \item 2\spades/3\spades/4\spades = to play/limit raise/to play.
\label{note:1}
   \item 2NT = good raise to 3 or 5 spades
		\begin{itemize}
		\item Opener bids long suit trials below the trump suit
		\item Cues or Swiss~\xref{sec:swiss} above the trump suit
		\end{itemize}
\label{note:2}
   \item 3\clubs/3\diamonds = fit jump.
\label{note:6}
   \item 3\hearts/4\clubs/\diamonds = splinter agreeing spades
	\item 4\hearts = to play
	\end{itemize}
}

{\color{CadetBlue}
\item 1\hearts (Level 4):
	\begin{itemize}
	\item 1\spades = natural, F1
   \item 1NT = 6-9 no better bid.
   \item 2\clubs/\diamonds = natural, F1.
   \item 2\hearts/3\hearts/4\hearts = to play/limit raise/to play.
   \item 2NT = good raise to 3 or 5 hearts
		\begin{itemize}
		\item 3x = long suit trial
		\item 3\hearts = minimum
		\item 3\spades/4\clubs/\diamonds = cue, slam try
		\item 4\hearts = extras, no cue or trial
		\end{itemize}
   \item 2\spades/3\clubs/3\diamonds = fit jump.
   \item 3\spades/4\clubs/\diamonds/ = splinter
\label{note:3}
	\item 3NT = good raise to 4 hearts
	\end{itemize}
}

{\it
\item 1\spades (Level 5):
	\begin{itemize}
   \item 2\hearts/3\hearts/4\hearts = to play/limit raise/to play.
   \item 1NT = 1-round force.
   \item 2\clubs/\diamonds = natural, forcing.
   \item 2\spades = good raise to 4 hearts
		\begin{itemize}
		\item Opener bids long suit trials below the trump suit
		\item Cues or Swiss~\xref{sec:swiss} above the trump suit
		\end{itemize}
   \item 2NT = good raise to 3 or 5 spades
		\begin{itemize}
		\item Opener bids long suit trials below the trump suit
		\item Cues or Swiss~\xref{sec:swiss} above the trump suit
		\end{itemize}
   \item 3\clubs/3\diamonds = fit jump.
   \item 3\spades/4\clubs/\diamonds = splinter agreeing hearts
   \item 4\spades = to play
	\end{itemize}
}
{\color{CadetBlue}
\item 1\spades (Level 4):
	\begin{itemize}
   \item 1NT = 6-9 no better bid.
   \item 2\clubs/\diamonds/\hearts = natural, F1
   \item 2\spades/3\spades/4\spades = to play/limit raise/to play.
   \item 2NT = good raise to 3 or 5 spades
		\begin{itemize}
		\item 3x = long suit trial
		\item 3\spades = minimum
		\item 4\clubs/\diamonds/\diamonds = cue, slam try
		\item 4\spades = extras, no cue or trial
		\end{itemize}
   \item 3\clubs/3\diamonds/\hearts = fit jump.
	\item 3NT = good raise to 4 spades
   \item 4\clubs/\diamonds/\hearts = splinter
	\end{itemize}

}
\subsubsection{Natural 1NT}
\label{sec:resp:1nnat}

If 1NT is natural and not in 3rd/4th non-vulnerable then the responses are
detailed in the section ``Continuations after 1NT" \xref{sec:resp:1n}.

In the case of the third- or fourth-seat non-vulnerable 1NT the responses are
weak takeout and should be announced ``natural, non-forcing" \orange{4E5}.


\newpage

\subsubsection{2 level}
\label{sec:resp:2level}
\label{note:18}

\item 2\clubs:
	\begin{itemize}
   \item 2\diamonds = pass/correct.
			\begin{itemize}
		\item Pass = weak diamonds
      \item 2NT = 20--23 balanced; see ``Continuations after 2NT" \xref{sec:resp:2n}.
      \item 2\hearts = weak majors.
			\begin{itemize}
			\item 2\spades/3\clubs/3\diamonds = to play
			\end{itemize}
		\item 2\spades = GF opening better spades
		\item 3\clubs = GF opening better hearts
		\end{itemize}
   \item 2\hearts/\spades = non-forcing; major-focused.
		\begin{itemize}
		\item Pass = minimum weak with both majors
		\item 2NT = 20-23 balanced
		\item 3\clubs = GF opening
		\item 3\diamonds = weak, diamonds.
		\item 3\hearts = maximum weak with better \hearts
		\item 3\spades = maximum weak with better \spades
		\end{itemize}
	\item 2NT = F1, how good are your diamonds?
		\begin{itemize}
		\item 3\clubs = GF opening
 		\item 3\diamonds = bad diamonds
		\item 3\hearts = 2-suited
		\item 3\spades = good diamonds
		\item 3NT = 20-23 balanced
			\begin{itemize}
         \item 4\clubs/\diamonds = 2 under transfers to \hearts/\spades
         \item 4\hearts = RKQG
			\item 4\spades = either minor
         \item 4N = both minors
			\end{itemize}
		\end{itemize}
	\item 3\clubs = GF enquiry
		\begin{itemize}
		\item 3\diamonds = weak, diamonds
		\item 3\hearts = weak, majors, better hearts
		\item 3\spades = weak, majors, better spades
		\item 3N = weak, majors, balanced-ish
		\end{itemize}

	\end{itemize}

\newpage

\item 2\diamonds:
	\begin{itemize}
   \item 2\hearts = pass/correct.
		\begin{itemize}
		\item Pass = weak hearts
      \item 2\spades = weak blacks.
			\begin{itemize}
			\item 3\clubs/3\diamonds/3\hearts = to play
			\end{itemize}
      \item 2N = GF better spades.
      \item 3\clubs = GF better clubs.
		\end{itemize}
   \item 2\spades/3\clubs = non-forcing, \clubs/\spades-focused.
		\begin{itemize}
		\item Pass = minimum weak \clubs/\spades
      \item 3\diamonds = GF opening.
      \item 3\hearts = weak hearts.
		\item 3\spades = max better spades
		\item 3N = max better clubs
		\end{itemize}
   \item 2NT = F1, how good are your hearts?
		\begin{itemize}
      \item 3\clubs = two-suited.
      \item 3\diamonds = GF two-suited.
      \item 3\hearts = bad hearts.
      \item 3\spades = good hearts.
      \item 3NT = good, single-suited, running suit
		\end{itemize}
	\item 3\diamonds = GF enquiry
		\begin{itemize}
		\item 3\hearts = weak, hearts
		\item 3\spades = weak, blacks, better spades
		\item 3N = weak, blacks, balanced-ish
		\item 3\clubs = weak, blacks, better clubs
		\end{itemize}
	\end{itemize}

\newpage

\item 2\hearts:
	\begin{itemize}
   \item 2\spades = pass/correct.
		\begin{itemize}
	   \item Pass = weak spades
      \item 2NT = GF better clubs
      \item 3\clubs = weak minors.
			\begin{itemize}
			\item 3\diamonds/3\hearts/\spades = to play
			\end{itemize}
      \item 3\diamonds = GF better diamonds
		\end{itemize}
   \item 3\clubs/3\diamonds = non-forcing, \clubs/\diamonds-focused.
		\begin{itemize}
		\item Pass = minimum weak minors
      \item 3\hearts = GF opening.
      \item 3\spades = single-suited.
      \item 3NT = maximum weak minors, both equal.
      \item 4\clubs = maximum weak, better \clubs.
      \item 4\diamonds = maximum weak, better \diamonds.
		\end{itemize}
   \item 2NT = F1, how good are your spades?
		\begin{itemize}
      \item 3\clubs = two-suited.
      \item 3\diamonds = dire, single-suited.
      \item 3\hearts = GF opening.
      \item 3\spades = good, single-suited, points spread.
      \item 3NT = good, single-suited, running suit.
		\end{itemize}
	\item 3\hearts = GF enquiry
		\begin{itemize}
		\item 3\spades = weak, spades
		\item 3N = weak, minors, balanced-ish
		\item 3\clubs = weak, minors, better clubs
		\item 3\diamonds = weak, minors, better diamonds
		\end{itemize}
	\end{itemize}

\newpage

\item 2\spades:
	\begin{itemize}
   \item 2NT = Enquiry (strong or to play in hearts opposite clubs and hearts)
		\begin{itemize}
      \item 3\clubs = weak with \clubs
      \item 3\diamonds = weak with \hearts\&\diamonds
      \item 3\hearts = weak with \hearts\&\clubs
      \item 3\spades = GF, \hearts\&\clubs
			\begin{itemize}
			\item 4\clubs/\diamonds = General Swiss for \clubs ~\xref{sec:swiss}
			\item 4NT = RKCB for \hearts ~\xref{sec:rkcb}
			\end{itemize}
      \item 3NT = GF, \hearts\&\diamonds
			\begin{itemize}
			\item 4\clubs/\diamonds = General Swiss for \diamonds ~\xref{sec:swiss}
			\item 4NT = RKCB for \hearts ~\xref{sec:rkcb}
			\end{itemize}
		\end{itemize}
	\item 3\clubs = Pass if clubs or hearts + clubs weak, or enquiry
		\begin{itemize}
		\item Pass = \clubs or \hearts\&\clubs
		\item 3\diamonds = \hearts\&\diamonds weak
		\item 3\hearts = \hearts\&\clubs GF
		\item 3\spades = \hearts\&\diamonds GF
		\end{itemize}
   \item 3\diamonds/\hearts = Pass or correct
	\end{itemize}

\item 2NT:
	\begin{itemize}
   \item 3\diamonds/3\spades = simple preference
   \item 4+\diamonds/4+\spades = preemptive raise
		\begin{itemize}
      \item After weak preference/raise, any rebid shows GF hand.
		\end{itemize}
   \item 3\clubs/3\hearts = transfer preference; invitational.
		\begin{itemize}
      \item Opener breaks transfer if GF two-suiter; suit has been set so bid controls.
      \item A 3NT rebid shows 24+ balanced
			\begin{itemize}
         \item 4\clubs/\diamonds = 2 under transfers to \hearts/\spades
         \item 4\hearts = RKQG
			\item 4\spades = either minor
         \item 4N = both minors
			\end{itemize}
		\end{itemize}
   \item 4\clubs/4\hearts = transfer preference; mixed raise.
	\end{itemize}

\newpage

\subsubsection{Responses to higher opening bids}
\label{sec:resp:higher}

\item 3\clubs/\diamonds/\hearts:
	\begin{itemize}
	\item complete transfer (at any level) = to play
		\begin{itemize}
		\item 3N (if available) = GF option
		\item Other slam tries (if available) = GF option
		\end{itemize}
	\item step above completing transfer (eg 3\clubs-3\hearts) = shortage ask
		\begin{itemize}
		\item 3N = no shortage, weak option
		\item 4 level-completion (eg 3\clubs-3\hearts-4\diamonds) = GF option
		\item other suits = 0 or 1, weak option
		\end{itemize}
	\item 3N = to play vs weak option
		\begin{itemize}
		\item slam tries = GF option
		\end{itemize}
	\item 4\clubs/\diamonds = General Swiss for weak option~\xref{sec:swiss}
	\end{itemize}

\item 3\spades:
	\begin{itemize}
	\item 3N = stops in 3 suits
	\item clubs at any level = pass or correct
	\item 4\diamonds = asks for singletons
		\begin{itemize}
		\item 4\hearts/\spades = singleton in that suit
		\item 4N = no singletons
		\item 5\clubs/\diamonds = singleton in the other minor
		\end{itemize}
	\end{itemize}

\item 3NT:
	\begin{itemize}
	\item clubs at any level = pass or correct
		\begin{itemize}
		\item Pass = weak clubs
		\item Diamonds = weak diamonds
		\end{itemize}
	\item 4\diamonds = enquiry
		\begin{itemize}
		\item 4\hearts = clubs
		\item 4\spades = diamonds
		\end{itemize}
	\end{itemize}

\newpage

\item 4\clubs/\diamonds:
	\begin{itemize}
	\item next-step = slam try with support
		\begin{itemize}
		\item complete transfer = decline
		\item new suits = cues
		\end{itemize}
	\item completion = to play
	\end{itemize}

\item 4NT:
	\begin{itemize}
	\item 5\clubs = No aces
	\item 5\diamonds = Club ace or Heart, Diamond and Spade aces
	\item 5\hearts = Diamond ace or Club, Heart and Spade aces
	\item 5\spades = Heart ace or Club, Diamond and Spade aces
	\item 5N = Two aces not including Spades
	\item 6\clubs = Spade ace or Club, Diamond and Heart aces
	\item 6\diamonds = Spade and Club aces
	\item 6\hearts = Spade and Diamond aces
	\item 6\spades = Spade and Heart aces
	\end{itemize}

\end{itemize}

\newpage

\subsection{Continuations after 1NT}
\label{sec:resp:1n}


This refers only to a natural 1NT, though it may be 9--11, 11--13, 12--13,
12--15, 14--16, 16--19 or 17--19 depending on position, vulnerability and
auction so far.  Maximum is an 11-count, good 12--13-count, 13-count,
14--15-count, good 15--16-count, good 18--19-count or 18--19-count depending on
range.

After a 9--15 no trump, or a NV 12--15 no trump (NV 3rd and 4th respectively),
or after 1\clubs--1\diamonds--1NT, all bids are natural and to play.

\begin{itemize}

\item 2\clubs is 5-card puppet Keri (see below), which allows the system to be used with weak diamond hands with either tolerance for the majors or playing in 3\diamonds. (e.g. 2=3=6=2 pattern with very limited values). It may also contain a weak hand looking to play in 3\clubs.

\label{note:9}

After 2\clubs is doubled:

	\begin{itemize}
	\item Pass = no 4 or 5 card major, no diamond fit
	\item XX = a 4 card major
	\item 2\diamonds = diamond fit, no 4 or 5 card major
	\item 2\hearts/\spades = 5 card suit
	\end{itemize}

After 2\clubs is overcalled by 2\diamonds:
	
	\begin{itemize}
	\item Pass = no 5 card major
		\begin{itemize}
		\item 2\hearts = 4 spades, possibly hearts
		\item 2\spades = 4 hearts, denies spades
		\end{itemize}
	\item 2\hearts/\spades = 5 card suit
	\end{itemize}

\item 2\diamonds and 2\hearts are transfers; transfer breaks apply with any 4-card fit.  2NT
shows 4/5-card support, no side 4-card suit, and a maximum.  Other suits
show 4 cards, and 4-card support with a maximum.  A simple jump acceptance
shows 4-card support and a minimum.

After the transfer is doubled:

	\begin{itemize}
	\item Pass = 2 cards in support, minimum
	\item Redouble = 2-3 cards in support, maximum
	\item Complete = 3 cards in support, minimum
	\item Normal transfer breaks = 4+ cards in support
	\end{itemize}

\item 2\spades is a range enquiry or invitational+ transfer to
\clubs.  2NT shows a minimum, after which 3\clubs shows invitational
with only clubs and other bids are GF, clubs and another.  3\clubs shows
any maximum, and can be passed if 2\spades showed clubs.

\item 2NT is a transfer to diamonds; 3\clubs is the only available super-accept which 
should be used with any 4 or honour-third.

\item 3\clubs promises a long suit that may need some help to run.  Opener should
pass, or bid 3NT with (e.g.) Kxx in the suit.

\item 3\diamonds is 5--5 in the majors, at least invitational.
	\begin{itemize}
	\item 3M = preference, minimum
	\item 4M = preference, maximum
	\item 4m = General Swiss \xref{sec:swiss}
	\end{itemize}

\item 3\hearts/3\spades are slam tries in clubs and diamonds.
	\begin{itemize}
	\item 3N = small doubleton
		\begin{itemize}
		\item 4N = quant (NT)
		\item 4m = General Swiss \xref{sec:swiss} for the minor
		\end{itemize}
	\item 4m = General Swiss \xref{sec:swiss}
	\end{itemize}

\item 4\clubs is Roman Key-Quant Gerber \xref{sec:rkqg}.

\item 4N is Viscount \xref{sec:viscount}.

\end{itemize}

\paragraph{Responses to 5-card puppet Keri}
\label{note:7}

Because we may have already established a game-force before bidding 1NT and we
use the principle of fast-arrival some of the ranges vary. In the responses
below, min/max and invitational / GF are inverted if a game force has already
been established, so:

\begin{itemize}
\item min $\rightarrow$ max
\item max $\rightarrow$ min
\item GF $\rightarrow$ denies slam interest
\item invitational $\rightarrow$ slam try
\end{itemize}

Accepting slam-tries should generally be done by bidding
4\clubs/4\diamonds which are General Swiss~\xref{sec:swiss} where
available. Either in a suit if one has been agreed or for no-trumps.


	\begin{itemize}
	\item 2\hearts/\spades with a 5-card major
		\begin{itemize}
		\item 3\clubs to play
		\item 3\diamonds to play
		\item other rebids are as after this sequence in 5 card puppet stayman.
		\end{itemize}
	\item 2\diamonds with all other hands. 
		\begin{itemize}
		\item Pass if weak with diamonds
		\item 2\hearts = 4 spades; may have 4 hearts.
			\begin{itemize}
			\item 2\spades = forcing; promises 4 hearts.
				\begin{itemize}
				\item 2NT/3NT = inv/GF, not 4 hearts.
				\item 3\clubs/3\diamonds = 4 hearts, feature in suit bid; GF
				\item 3\hearts/4\hearts = inv/GF, 4 hearts.
				\item 3\spades = GF, 4 hearts, auto-Swiss \xref{sec:swiss}.
				\item 4\clubs/4\diamonds = GF, 4 hearts, Swiss \xref{sec:swiss}.
				\end{itemize}
			\item 2NT/3NT = min/max, denies a 4-card major.
			\item 3\clubs/3\diamonds/3\hearts = 4 spades, feature in suit bid; max.
			\item 3\spades/4\spades = min/max, 4 spades.
			\item 4\clubs/4\diamonds = max, non-serious Swiss \xref{sec:swiss}.
			\end{itemize}

		\item 2\spades = 4 hearts; denies 4 spades.
			\begin{itemize}
			\item 2NT/3NT = min/max, denies 4 \hearts.
			\item 3\clubs/3\diamonds = 4 hearts, feature in suit bid, max.
			\item 3\hearts/4\hearts = min/max, 4 hearts.
			\item 3\spades = 4 hearts, maximum, auto-Swiss \xref{sec:swiss}.
			\item 4\clubs/4\diamonds = GF, 4 hearts, Swiss \xref{sec:swiss}.
			\end{itemize}

		\item 2NT = 3--3 majors; invitational strength.
		\item 3\clubs = to play
		\item 3NT=3--3 majors, GF.
		\end{itemize}
	\end{itemize}

\newpage

\subsubsection{After 1NT is doubled for penalties}
\label{sec:resp:1nx}
\label{note:20}

After a direct seat double we play a modified form of suction as the escape after 1NT-X:

\begin{itemize}
\item Pass forces XX, to play or weak with clubs and another
	\begin{itemize}
	\item XX forced
		\begin{itemize}
		\item Pass with a strong hand, to play
		\item bid 4 card suits up the line
		\end{itemize}
	\end{itemize}
\item XX forces 2\clubs, weak with clubs or the reds
	\begin{itemize}
	\item 2\clubs forced
		\begin{itemize}
		\item Pass with clubs
		\item 2\diamonds with \diamonds\&\hearts
		\end{itemize}
	\end{itemize}
\item 2\clubs puppets 2\diamonds, weak with diamonds or the majors
	\begin{itemize}
	\item Pass with 5 clubs
	\item 2\diamonds with all other hands
		\begin{itemize}
		\item Pass with diamonds
		\item 2\hearts with the majors
		\end{itemize}
	\end{itemize}
\item 2\diamonds is weak with diamonds and spades
\item 2\hearts is weak with hearts
\item 2\spades is weak with spades
\end{itemize}

This defence is played a level higher in the rare case of 2NT being doubled for
penalties and applies in all cases that a natural notrump bid below game is doubled for
penalties.

\label{note:21}

After a protective double of 1NT we play Aardvark with a Halmic redouble where
bids show a two suiter (that suit and a higher) and redouble forces 2\clubs and shows a weak single-suiter.

\label{note:15}

After a 9-15 1NT is doubled in direct seat then it's as with the protective double, but pass is optional, for opener to pass with 13-15 and XX or bid a suit with 9-12.

\subsubsection{After direct overcalls of 1NT}
\label{sec:resp:lebensohl}
\label{note:12}

Lebensohl:

\begin{itemize}
\item suits at the 2 level = to play
\item suits at the 3 level = GF, natural
\item direct cue = staymanic, denies a stop
\item 3N = natural, denies a stop
\item 2N = puppet to 3\clubs
	\begin{itemize}
	\item 3\clubs forced
		\begin{itemize}
		\item suits below the cue = to play
		\item suits above the cue = invitational, natural
		\item cue = staymanic, promises a stop
		\item 3N = natural, promises a stop
		\end{itemize}
	\end{itemize}
\item Double is for penalties
\item Double of a natural 2\clubs overcall is stayman showing a club stop (optional)
\end{itemize}

\newpage

\subsection{Continuations after 2NT}
\label{sec:resp:2n}

\begin{itemize}
\label{note:8}
\item 3\clubs: 5-card puppet Stayman
	\begin{itemize}
   \item 3\hearts/\spades shows 5 cards.
   \item 3\diamonds promises 4 hearts or 4 spades
		\begin{itemize}
      \item 3\hearts = 4 spades, 0--3 hearts
			\begin{itemize}
			\item 3\spades = 4 hearts
         \item 4\spades = to play
			\item 3NT = to play
			\item 4\clubs/\diamonds = Swiss \xref{sec:swiss} for spades
			\end{itemize}
      \item 3\spades = 4 hearts, 0--3 spades
			\begin{itemize}
         \item 4\hearts = to play
			\item 3NT = to play
			\item 4\clubs/\diamonds = Swiss \xref{sec:swiss} for hearts
			\end{itemize}
      \item 3N = to play
		\item 4\clubs = RKQG
		\end{itemize}
   \item 3NT denies 4 hearts or 3 spades
	\end{itemize}

\item 3\diamonds/3\hearts: transfers. Superaccept with any 4-card fit; note that 3NT is a
superaccept that still allows the use of Swiss.

\label{note:10}
\item 3\spades: puppet to 3NT, either to play or for minors.
	\begin{itemize}
   \item 3NT = forced
		\begin{itemize}
		\item Pass = to play
		\item 4\clubs = slam try in diamonds
		\item 4\diamonds = slam try in clubs
		\item 4\hearts = both minors, heart splinter
		\item 4\spades = both minors, spade splinter
		\end{itemize}
	\end{itemize}

\item 3NT: 5 spades + 4 hearts

\item 4\clubs is Roman Key-Quant Gerber \xref{sec:rkqg}.
\item 4N is Viscount \xref{sec:viscount}.
\end{itemize}

\newpage


\section{Slam conventions}
\label{sec:slam}

If suit agreement isn't possible after a multi-suited bid then (if possible)
General Swiss agrees the minor (or lower) suit and Blackwood agrees the major
(or higher) suit.

\subsection{Roman key-card Blackwood (1430)}
\label{sec:rkcb}

4NT with an agreed suit is Roman key-card Blackwood. The key-cards are the four aces
and the king of the agreed suit. Responses are:

\begin{itemize}
\item 5\clubs: 1 or 4 key cards
\item 5\diamonds: 0 or 3 key cards
\item 5\hearts: 2 or 5 key cards without the queen of trumps
\item 5\spades: 2 or 5 key cards with the queen of trumps
\end{itemize}

After 5\clubs or 5\diamonds the next-step (other than trumps) asks for the
queen of trumps. Responses are trumps at the lowest level without, or the
appropriate king response (see Section~\ref{sec:kyhokyd}) with the queen and an
outside king, or 5NT with no outside king.

\newpage

\subsection{Exclusion Blackwood}
\label{sec:exclusion}

Unwarranted bids of an outside suit at the 5 level are Exclusion Blackwood.
Responses are as whichever form of Blackwood would apply, but ignoring the ace
in the suit bid. The next available suit shows the normal 5\clubs response and
bids proceed from there.


\subsection{King you have or king you don't}
\label{sec:kyhokyd}

After RKCB, the next available suit which is not the trump suit asks for the
queen of trumps and outside kings. 5N just asks for outside kings. Without the
queen of trumps the response to the queen-ask is always the trump suit at the
lowest level. Responses to 5N, or if you have the queen of trumps are; with one
outside king, bid that suit. With two outside kings, bid the suit of the king
you are missing. With no kings bid 6 of the trump suit (or 5N if available).

\subsection{First-round-control showing cues}
\label{sec:cues}

Once a suit is agreed and a GF is established, new suits are cues showing an
ace or a void in that suit. Unless otherwise agreed to be something else the
bid of a suit a level above when it would be forcing for a round is a first round
control showing cue agreeing the most recently shown suit.

If a game-forcing auction has been established and if there are the following
ways to agree a suit: bidding Swiss~\xref{sec:swiss}, bidding game and
supporting below game, then the latter is a slam try which would prefer cue
responses rather than swiss.

\newpage

\subsection{General Swiss}
\label{sec:swiss}
\label{note:19}

4\clubs/4\diamonds, once a game-force is established and suit agreed, are slam tries, based
on control points.

Control points are 2 for an ace, 1 for a king/singleton, and 1 for the queen of
trumps.  There are 13 available from honours; it is possible to substitute
outside singletons for kings, but care must be taken to avoid double-counting.
11 CPs are necessary for a small slam; 13 for a grand.

\begin{itemize}
\item 4\clubs: 4 or 6 (exceptionally 8) control points
	\begin{itemize}
   \item trump suit = 0--4 CPs
   \item 4\diamonds = 5/6 CP, relay
		\begin{itemize}
      \item trump suit = 4 CPs
      \item suits = 6 CP, lowest king/singleton
		\end{itemize}
   \item suits = 7 or 9 CP, lowest king/singleton
   \item small slam = 8 CP
   \item grand slam = 10 CP
	\end{itemize}

\item 4\diamonds: 5 or 7 (exceptionally 9) control points
	\begin{itemize}
   \item trump suit = 0--3 CPs
   \item 4\hearts = 4/5 CP, relay
		\begin{itemize}
      \item trump suit = 5 CPs
      \item suits = 7 CP, lowest king/singleton
		\end{itemize}
   \item suits = 6 or 8 CP, lowest king/singleton
   \item small slam = 7 CP
   \item grand slam = 9 CP
	\end{itemize}
\end{itemize}

While cueing kings/singletons, 4NT shows a singleton or king which can't be
shown below 5 of the trump suit (usually diamonds over clubs) and asks partner
to signoff appropriately depending on duplication.

If you find duplication in kings/singletons subtract one CP and sign off at the
appropriate level. If your partner signs off and you have 2 as-yet unshown CPs,
raise one level.

If the hand bidding Swiss has shown a GF (opening a strong 2 level option,
1\diamonds and a jump rebid, or 1\diamonds and a 2NT rebid), add 2 to
all numbers. If it has shown a weak hand (opening a weak 2 or 3 level option or
giving a 0--7 response to 1\diamonds or opening 9-11 1NT), subtract 2 from
all numbers. Direct swiss over a 3-level bid that may be weak or strong assumes
weak for counting control points.

There is a small problem if you have 7 CPs and are agreeing hearts, since in
that case, there is no available next step. In this case you may wish to show
6 CPs and ignore one point of duplication.

\subsubsection{Interference}

After interference direct over the start of General Swiss:

\begin{itemize}
	\item Pass = would sign-off in game (forcing)
	\item X or XX = next-step
	\item Suits = Accepting the slam try and checking for duplication as normal
\end{itemize}

After interference direct after a non-signoff response to Swiss:

\begin{itemize}
	\item Pass = would sign-off in game (forcing)
	\item X or XX = a second-round control in that suit
	\item Suits = a second-round control in that suit
\end{itemize}

\subsection{Roman Key-Quant Gerber}
\label{sec:rkqg}
\label{note:22}

After a NT opening or rebid, 4\clubs is Roman Key-Quant Gerber, asking for
aces and quantatitive. Responses are:

\begin{itemize}
	\item 4\diamonds = 0 or 3 aces
	\item 4\hearts = 1 or 4 aces
	\item 4\spades = 2 aces and denying a quant raise
	\item 4NT = 2 aces and accepting the quant raise
\end{itemize}

After 4\diamonds/\hearts, next step asks for quant, with 4NT denying
and bidding kings or 6NT accepting.

For RKQG sequences starting at higher bids, raise the responses appropriately.

\subsection{Viscount}
\label{sec:viscount}
\label{note:23}

Since all the quantatitive raises and ace asking go through Gerber (above), 4NT
is now free. Bidding 4NT over a NT opening or rebid asks opener to bid 4 card
suits up the line. It's very much like the Baron convention, only higher.

\section{Competitive bidding}
\label{sec:competitive}

We tried an artificial overcall scheme but found it couldn't cope with a number of situations.
Therefore, we have a fairly natural overcall style.

\subsection{Natural suits}
\label{sec:def:1x}

\subsubsection{Simple overcalls}
\label{note:11}
Simple overcalls are constructive and natural 5 cards, starting at
10HCP~\orange{7E2}. A responsive cue is a good raise and
direct raises are preemptive.

\subsubsection{Double}
Double is standard takeout (or a strong hand)~\orange{7E2}. After a double and
a response, a cue by the doubler is a general force, not suit agreeing.

\subsubsection{NT overcall}
Direct 1NT is 15-17 balanced with a stop or semi-balanced and may contain a
singleton ace~\orange{7E3}, protective 1NT is 11-14 with a stop.  After these
``Continuations after 1NT''~\xref{sec:resp:1n} apply, with the exception that
after the auction (1x)-1N-(2y), 2x is stayman without a stop if available.
Other bids are lebensohl \xref{sec:resp:lebensohl}.

\subsubsection{Jump overcalls}
Jump overcalls are weak, showing a 6 card suit and between 4HCP (favourable)
and enough to make a simple overcall. In protective they are intermediate
(11-15, reasonable 6+ card suit). This also applies after the sequence (Pass)-Pass-(1N).

\subsubsection{Jump cue bids}
Jump cuebids are stopper-asking for 3NT \orange{7E2}; they promise a long
running suit.  An overcall of 3NT promises a long running suit and a stopper~\orange{?}.
Responses are the same as a gambling 3NT opening~\xref{sec:resp:higher}.

\subsubsection{2-suited overcalls}

Simple cuebids are Michaels showing at least 5--5 in the majors (over a minor)
or the other major and either minor (over a major) \orange{7E2}. In all cases they are
either weak or strong. Similarly, 2NT over a 1 level bid is unusual, showing
5--5 or better in the minors (over a major) or the other minor and either major
(over a minor) \orange{7E2}. Again, it's weak or strong.

After Michaels or Unusual, bidding any of the suits which could have been shown
is pass or correct at that level. Overcaller then corrects to the next highest
suit (or passes or raises) with a weak hand. With a strong hand he breaks to
another suit. Of the two suits which are not 'pass or correct', re-cueing is
strong and agreeing advancer's preferred suit and the other suit is strong with
the remaining option.

For example:
\begin{itemize}
\item (1\hearts)-2\hearts-(P)
	\begin{itemize}
	\item 3\clubs-3\diamonds is weak with spades and diamonds
	\item 3\clubs-3\hearts is strong with spades and clubs
	\item 3\clubs-3\spades is strong with spades and diamonds
	\item 3\diamonds-3\hearts is strong with spades and diamonds
	\item 3\diamonds-3\spades is weak with spades and clubs
	\item 3\diamonds-4\clubs is strong with spades and clubs
	\item 2\spades-3\clubs is strong with spades and clubs
	\item 2\spades-3\diamonds is strong with spades and diamonds
	\end{itemize}
\end{itemize}

If advancer has a strong hand then the following response structures apply:

Any suit which overcaller could have at any level is to play opposite a weak
hand and pass or correct. If responder has passed and overcalled has two known
suits, then other two suits at the 3 level are strong raises in one of
overcaller's suits; the cheaper outside suit for the cheaper known suit. These
are invitational-plus if it is possible to stop below game in that suit and
game forcing otherwise. 2NT and 3NT are both natural. The special case of
(1\clubs)-2\clubs-(P)-2\diamonds asks for the better of
overcall's majors.

After an invitational raise, overcaller bids the agreed suit at the 3 level
with a minimum weak hand and bids game with a good weak hand. With a strong
hand he bids swiss in that suit or does something else.

For example:

\begin{itemize}
\item (1\clubs)-2\clubs-(P)
	\begin{itemize}
	\item 2\diamonds - which is your better major
	\item 2\hearts - to play
	\item 2\spades - to play
	\item 2NT - natural, invitational
	\item 3\clubs - invitational with hearts
	\item 3\diamonds - invitational with spades
	\item 3\hearts - to play
	\item 3\spades - to play
	\item 3NT - to play
	\end{itemize}
\item (1\hearts)-2NT-(P)
	\begin{itemize}
	\item 3\clubs - to play
	\item 3\diamonds - to play
	\item 3\hearts - invitational with clubs
	\item 3\spades - invitational with diamonds
	\item 3NT - to play
	\end{itemize}
\end{itemize}

In the case where the two suits are not specified (responder still having
		passed), then any suit which overcaller could have is pass or correct.
2NT, if available (i.e. after Michaels), is an invitational-plus enquiry as to
the other suit and re-cueing is a strong raise in the suit which has been
promised. A direct 4\clubs/4\diamonds if the promised suit is a major
is Swiss~\xref{sec:swiss} for that suit and 3NT is to play. If 2NT is not
available (i.e. after Unusual) then the re-cue is any stronger hand and
responses are which other outside suit overcaller has.

\newpage

For example:

\begin{itemize}
\item (1\hearts)-2\hearts-(P)
	\begin{itemize}
	\item 2\spades - to play
	\item 2NT - enquiry, invitational
		\begin{itemize}
		\item 3\clubs - weak clubs and spades (responses = Swiss for clubs)
		\item 3\diamonds - weak diamonds and spades (responses = Swiss for diamonds)
		\item 3\hearts - strong clubs and spades
		\item 3\spades - strong diamonds and spades
		\end{itemize}
	\item 3\clubs - pass or correct
	\item 3\diamonds - pass or correct
	\item 3\hearts - invitational in spades
	\item 3\spades - to play
	\item 4\clubs - Swiss for Spades
	\item 4\diamonds - Swiss for Spades
	\item 3NT - to play
	\end{itemize}
\item (1\clubs)-2NT-(P)
	\begin{itemize}
	\item 3\clubs - any invitational
		\begin{itemize}
		\item 3\hearts - hearts and diamonds
		\item 3\spades - spades and diamonds
		\end{itemize}
	\item 3\diamonds - to play
	\item 3\hearts - pass or correct
	\item 3\spades - pass or correct
	\item 3NT - to play
	\end{itemize}
\end{itemize}

If responder makes a raise of opener's suit (so, the sequences (1\clubs)-2\clubs-(3\clubs)
or (1\hearts)-2NT-(3\clubs)) the structure is as follows: Double shows a strong raise of
overcaller's known suit (if only one) or cheaper known suit (if two). If an outside suit other than the re-cue
is available then that is a strong raise in the remaining suit(s) and the re-cue is a strong raise with no 
preference, otherwise the re-cue is a strong raise in the remaining suit(s).

\newpage

For example:

\begin{itemize}
\item (1\clubs)-2\clubs-(3\clubs)
	\begin{itemize}
	\item X - invitational hearts
	\item 3\diamonds - invitational spades
	\item 3NT - to play
	\item 4\clubs - GF, no preference
	\end{itemize}
\item (1\clubs)-2NT-(3\clubs)
	\begin{itemize}
	\item X - invitational diamonds
	\item 3\diamonds - to play
	\item 3\hearts - pass or correct
	\item 3\spades - pass or correct
	\item 3NT - to play
	\item 4\clubs - GF in either major
	\end{itemize}
\end{itemize}

\subsection{Defences}
\label{sec:defences}

Aside from defences to natural suit openings, these are mostly all based on suction:

\begin{itemize}
\item clubs: diamonds or the majors
\item diamonds: hearts or spades and clubs (or diamonds)
\item hearts: spades or the minors
\item spades: clubs or hearts and diamonds (or clubs)
\end{itemize}

The responses to these are not the same as the responses to a 2-level opening,
but are more standard `pass or correct' responses. For example, after
(1N)-2\diamonds-(P), 2\spades says pass with the blacks, invitational
in hearts.

\subsubsection{Natural 1NT}
\label{sec:def:1n}

Suction, constructive values. Double is for penalties, which may be based on 7
top tricks, and 2N shows a non-touching two suiter. \orange{7E1}

After a direct penalty double, passes are forcing until we've bid or doubled
them.  Doubles are for penalties. In passout seat after they have escaped to
the two level and partner has not bid then Lebensohl \xref{sec:resp:lebensohl}
is in effect, eg: (1N)-X-(2x)-P-(P)-2N

\subsubsection{Artificial strong bids}
\label{sec:def:strong}

Suction, weak. Double shows the suction bid in their suit and 2N shows a
non-touching two suiter. \orange{7E1}

\subsubsection{Short suits}
\label{sec:def:short}

If it `could be as short as 2', then we treat it as a natural opening and the
defences in Section~\ref{sec:def:1x} apply.

\subsubsection{Phoney/Prepared suits}
\label{sec:def:phoney}

Suction, constructive values. Double shows the suction bid in their suit and 2N
is natural. \orange{7E1}

\subsubsection{Multi 2\diamonds}
\label{sec:def:multi}

\begin{itemize}
\item X: 16+ or 13-15 bal
\item 2N: 16-18 bal, stops in both majors.
	\begin{itemize}
	\item Responses as ``Continuations after 2NT" \xref{sec:resp:2n}
	\end{itemize}
\item 3x: intermediate, natural
\end{itemize}

\subsubsection{Natural weak openings}
\label{sec:def:weak}

Double is for takeout up to 4\hearts. After a weak two and a takeout double, there
are Lebensohl responses \xref{sec:resp:lebensohl}. 4N is takeout over 4\spades and a big two suiter otherwise.

Over a weak two,  2NT is natural showing 16-18 and ``Continuations after 2NT'' \xref{sec:resp:2n}
apply.

\subsubsection{Natural stronger jump openings}
\label{sec:def:inter-strong}

For intermediate natural 2-level openings the overcalls are as natural 1-level
openings~\xref{sec:def:1x}, with 2N as modified unusual. All actions need to be
sounder than over a 1-level opening.

Strong (forcing or not) natural 2-level openings all overcalls are weak and natural.

\subsubsection{Transfer openings}
\label{sec:def:transfer}

Including 2\clubs which is a weak 2 in diamonds or a strong hand.

\begin{itemize}
\item Double shows the suit bid
\item Bidding the transferred to suit is takeout of that suit
\item Jump-bidding the transferred to suit is Michaels
\item Other bids and continuations are as normal for that level and strength of bid
\end{itemize}

\subsubsection{Shortage preempts}

If the suit bid is short then double shows an overcall in that suit or any
large hand.  Other suits are natural but reasonably robust and 2NT shows stops
in all the other suits.

If the suit bid is one of the three promised then double is takeout or any large hand.
Suit bids are natural but reasonably robust. 2NT shows a balanced hand with at 
least a stop in the suit bid and some cards in the others.

\subsubsection{Other multi-way preempts}

Multi-way preempts will deny the suit bid and so double shows an overcall in
that suit. Big hands either pass then act or start with a double.

Other overcalls are natural.

\subsubsection{Definite two-suited preempts}

Such as 2NT showing both minors. Double shows the other two suits and a direct
cue bid asks for a stop in that suit, promises stops for any lower suits, but
does not show or deny stops in any higher suits. With a stop partner bids 3N if
he has a stop in all remaining higher suits, or the next suit he does not have
a stop in.

\subsubsection{Doubles}
\label{sec:def:doubles}

Doubles of freely bid slams are Lightner, asking for the lead of dummy's first shown suit.

Doubles of other artificial bids are lead directing. If the doubler has already shown that
suit then it asks for the lead of a different suit.

\newpage

\subsection{Dealing with interference}
\label{sec:interference}

\subsubsection{Direct interference over 1\clubs}
\label{sec:intf:1c}

\begin{itemize}
\item 1\clubs-(X) (penalty interest (showing 16+):
	\begin{itemize}
   \item Redouble shows either diamonds or the majors
   \item Pass forces redouble; either to play or with diamonds and a major. Can be pulled rather than XX with an unsuitable hand for clubs.
   \item 1\diamonds/1\hearts are transfers to 1\hearts/1\spades.  Note that opener may pass these if holding a 5-card 
      suit.
	\end{itemize}
\item 1\clubs-(X) (showing \clubs):
	\begin{itemize}
   \item Pass shows a weak hand, with nowhere in particular to play.
   \item Redouble creates a game-force; passes are now forcing and doubles are for penalty.
	\item 1D shows some values (6-9), any shape
   \item Suit bids are natural, 10--15, and promise 4 cards.
   \item 1NT shows a balanced hand, 10--15, with a club stop.
	\end{itemize}
\item 1\clubs-(X) (takeout):
	\begin{itemize}
   \item Redouble still creates a forcing pass situation.
	\item 1D shows some values (6-9), any shape
   \item Ask opponents ``takeout of what, exactly?".
   \item 1NT cannot show a stopper in any particular suit.
	\end{itemize}
\item 1\clubs-(natural bid):
	\begin{itemize}
   \item Responder's double creates a game-force.
   \item 1NT promises a stop; 10--15 balanced. See ``Responses after 1NT'' \xref{sec:resp:1n}
   \item Suit bids are natural, 10--15. After a NT rebid by opener, 3\clubs is checkback and normal system is off. 
   \item Exception: cheapest bid (or 2\spades over 1\spades) shows a 12-15 balanced hand with no stop.  System is on, including Keri, transfers etc.  After 1\clubs-(1\diamonds)-1\hearts, for example, 1\spades is a range enquiry or transfer to \clubs. See \xref{sec:resp:1n} or \xref{sec:resp:2n}.
   \item Pass is a weak hand with nowhere to go (potentially including an 8--11 balanced hand with no stop).
	\end{itemize}
\item 1\clubs-(1N)-P-P:
	\begin{itemize}
	\item 2N = minors
	\end{itemize}
\end{itemize}

\subsubsection{Direct interference over 1\diamonds}
\label{sec:intf:1d}

\begin{itemize}
\item 1\diamonds-(X):
	\begin{itemize}
   \item Redouble with good 8+ hands.
   \item Pass with weak hands.
   \item Suit bids are natural, weak, 6-card suit.
	\end{itemize}
\item 1\diamonds-(bid):
	\begin{itemize}
   \item Pass with weak hands.
   \item Double to establish a game-force (and forcing passes, etc.).
		\begin{itemize}
      \item Suction rebids are on; immediate NT bid denies a stop.
      \item To show a stop, Suction-force into opponents' suit then bid NT.
		\end{itemize}
   \item Suit bids are natural, weak, 6-card suit.
	\end{itemize}
\item 1\diamonds-(spade for a laugh):
	\begin{itemize}
   \item Double is for penalties
   \item Passes are forcing
	\end{itemize}
\end{itemize}

\subsubsection{Sandwich-seat interference (e.g. 1\diamonds-(P)-1\spades-(2\clubs))}
\label{sec:intf:bal}

\begin{itemize}
\item 1\clubs-(P)-1\spades-(anything):
\item 1\diamonds-(P)-1\spades-(anything):
\begin{itemize}
   \item Passes are forcing
   \item NT bids promise a stopper.
   \item Suction-style rebids are on if 3\clubs is a sufficient bid (1\diamonds opener)
   \item Pass indicates a balanced hand without a stopper in opponents' suit.
	\begin{itemize}
      \item Responder's NT bids here are Lebensohl style \xref{sec:resp:lebensohl}
      \item There are no weak hands to show, so the only difference is in showing/denying a stopper.
	\end{itemize}
   \item Double is penalties.
\end{itemize}

\item 1\clubs-(P)-1 any-X: XX shows a maximum and 4 cards in that suit.

\item 1\clubs-(P)-1\diamonds-(anything): it's a good idea to subside quietly, however, double is takeout below 2NT, penalties above.

\item 1\clubs/\diamonds-(P)-1\hearts-(anything): passes are non-forcing, X is takeout / suction

\item 1\clubs-(P)-1\hearts-(1\spades)-P: 12-15 balanced without a spade stop.

Responses:
	\begin{itemize}
	\item 1NT = Lebensohl
		\begin{itemize}
		\item 2x = invitational
		\item 2\spades = stayman with a stop
		\end{itemize}
	\item 2x = weak
	\item 2\spades = stayman without a stop
	\item 2NT = invitational with a stop
	\item 3x = forcing
	\item 3NT = to play, shows a stop
	\end{itemize}

\end{itemize}

\subsubsection{Interference over 1-major}
\label{sec:intf:1M}

{\it
\begin{itemize}
\item 1\hearts-(X) (Level 5):
	\begin{itemize}
	\item Pass = dross
	\item XX = 4 hearts and starts a penalty (but not forcing pass) auction
		\begin{itemize}
		\item Pass = 4 hearts
		\item 1\spades = dross
		\item 1N = 5\spades + 4 minor
		\item 2\clubs = 5\spades + 5\clubs
		\item 2\diamonds = 5\spades + 5\diamonds
		\item 2\spades = 6+\spades
		\end{itemize}
	\item 1\spades = to play
	\item Other bids system on.
	\end{itemize}
\item 1\spades-(X) (Level 5):
	\begin{itemize}
	\item Pass = dross
	\item XX = 5 spades and starts a penalty (but not forcing pass) auction
		\begin{itemize}
		\item 1N = dross
		\item 2\clubs = 5\hearts + 5\clubs
		\item 2\diamonds = 5\hearts + 5\diamonds
		\item 2\hearts = 6+\hearts
		\end{itemize}
	\end{itemize}
\end{itemize}
}

{\color{CadetBlue}
\begin{itemize}
\item 1\hearts/\spades-(X) (Level 4):
	\begin{itemize}
	\item Pass = 0-2 card support, bad hand, NF
	\item XX = 0-2 card support, good hand, no specific suit to bid or strong enough to force, F1
	\item 1N = 3 card support, good values, F1
	\item 2M (raise) = 3 card support, poor values, NF
	\item 3M (raise) = 4 card support, poor values, NF
	\item Other raises as without the double, F1 or GF
	\item Other new suits = 0-2 card support, focussed suit, NF
	\end{itemize}
\end{itemize}
}

\begin{itemize}
\item 1\hearts/\spades-(2x):
	\begin{itemize}
	\item 2N = balanced, 10-12 with a stop
	\item 3x = good raise
	\end{itemize}
	 
\end{itemize}


\subsection{Interference over a 2-level opening}
\label{sec:intf:2level}

A few general principles here:

\begin{itemize}
\item 2x-(X):
   \begin{itemize}
      \item XX = forcing enquiry (even over further interference, starts a penalty auction)
      \item Pass = no desire to play opposite a weak hand
      \item Relay = some tolerence for the weak options
      \item Enquiries = forcing unless oppo bid
   \end{itemize}
\newpage
\item 2x-(2y):
   \begin{itemize}
      \item X = optional (pass if it's your suit, pull otherwise, starts a penalty auction)
      \item \ldots
   \end{itemize}
\item Opener's rebids facing passed responder:
   \begin{itemize}
   \item Suit from a weak option at the lowest level = decent weak option
   \item X/denied suit/NT = strong option (X is optional)
   \end{itemize}
\item 2x-(p)-relay-(X):
   \begin{itemize}
	\item Pass = weak single suiter
	\item Cheaper of two suits = weak 2 suiter, prefer cheaper suit
	\item XX = weak 2 suiter, prefer other suit
	\item 2N ($x = \clubsuit$) = 20-23 bal
	\item 2N ($x \neq \clubsuit$) = GF, prefer cheaper suit
	\item Other of two suits = GF, prefer other suit
	\item Jump cheaper suit ($x = \clubsuit$) = GF, prefer cheaper suit
   \end{itemize}
\item 2N-(x):
   \begin{itemize}
	\item Pass = no preference, weak
	\item XX = no preference, strong
	\item others = as normal
   \end{itemize}
\end{itemize}

\subsection{Forcing passes}
\label{sec:forcepass}

There are a number of situations where a pass can be forcing in the sense of
``either we bid at least once more or defend doubled". The following situations
are definitely forcing-pass situations:

\begin{itemize}
\item After a freely bid game
\item After 1NT-X and they escape at the two level
\item If we have shown values for game in the auction so far
\item ...
\end{itemize}

\newpage

\section{Carding}
\label{sec:carding}

\subsection{Leads}
\label{sec:card:leads}

Leads are standard (2nd from bad suits, 4th from an honour, top of sequences),
except that they may vary based on signal requested (see below).

The exception is that we lead top of bad suits against no trumps.

After the initial lead we lead high from bad suits and low from good suits.

\subsection{Signals}
\label{sec:card:signals}

Signals are reverse (low encourages) attitude on partner's lead of the A, Q or
even pip cards and standard count (high even) on K or J leads or odd pip cards
and on declarer's lead.

Carding will frequently be false on declarer's lead.

Leads to ruffs and other obvious situations may be lead directing Lavinthal-style.

\subsection{Discards}
\label{sec:card:discards}

Discards are a modified Italian style whereby Fibonacci (A2358K) discards are
encouraging and other discards (4679TJQ) are suit-preference
Lavinthal-style. Thus, low non-Fibonacci cards ask for the lower of the outside
suits and high non-Fibonacci cards ask for the higher of the two outside suits.
\newpage

\section{Conventions for lower-level events}
\label{sec:cat3}

Sometimes it is neccessary to play at WBF category 3 events or other venues
more restrictive than EBU Level 4. In those cases, it is not permitted to play
the system as described above. The treatments restricted at lower levels are
the two and three level openings, although this also has minor differences to
other bids.

\subsection{1 diamond}

The strong diamond opening must include GF single suiters. After opening
1\diamonds, which is now forcing, jump rebids show GF single suited hands.

\subsection{Two level openings}

\begin{itemize}
\item 2\clubs = 5-10HCP, 4+ hearts, 4+ spades
	\begin{itemize}
	\item 2\diamonds = enquiry as to strength and better major
	\item either major at any level = to play
	\end{itemize}
\item 2\diamonds = One of: 6+ weak major OR 21-24 bal OR 2-suited GF not \spades+\diamonds
	\begin{itemize}
	\item 2\hearts = pass or correct
		\begin{itemize}
		\item pass = weak hearts
		\item 2\spades = weak spades
		\item 2NT = balanced
		\item 3\clubs = GF minors
		\item 3\diamonds = GF reds
		\item 3\hearts = GF majors
		\item 3\spades = GF blacks
		\item 3NT = GF rounds
		\end{itemize}
	\item 2NT = enquiry
		\begin{itemize}
		\item 3\clubs = bad weak hearts
		\item 3\diamonds = bad weak spades
		\item 3\hearts = good weak hearts
		\item 3\spades = good weak spades
		\item 3NT = balanced
		\item 4\clubs = GF minors
		\item 4\diamonds = GF reds
		\item 4\hearts = GF majors
		\item 4\spades = GF blacks
		\item 4NT = GF rounds
		\end{itemize}
	\end{itemize}
\item 2\hearts = 5-10HCP, 5+ hearts, 4+ minor
	\begin{itemize}
	\item 2NT = enquiry for minor and strength
	\item either minor at any level = paradox 
	\end{itemize}
\item 2\spades = 5-10HCP, 5+ hearts, 4+ minor
	\begin{itemize}
	\item 2NT = enquiry for minor and strength
	\item either minor at any level = paradox 
	\end{itemize}
\item 2NT = GF pointeds or 25+ balanced
	\begin{itemize}
	\item transfer preference
	\end{itemize}
\end{itemize}

\subsection{Three level openings}

Three level bids must be natural, not transfers, so all the three level suit
openings are natural and weak. 3NT is gambling.

\newevenside

% appendices below here
\def\thesection{\Alph{section}}
\setcounter{section}{0}

\small
\section{Prepared Defences}
\label{appx:defences}

Please note that the EBU does not recommend providing prepared defences to
systems.  Nonetheless, if you would find these useful here are some {\em simple}
defences to our system, but there are probably several better ones.

\subsection{1 level bids}

\begin{itemize}
\item 1\clubs
	One of the following:
	\begin{itemize}
	\item Your defence to a possibly short club or diamond
	\item Your defence to a weak no-trump
	\item Your defence to a natural club opening 
	\item Your defence to a polish club
	\end{itemize}
\item 1\diamonds
	\begin{itemize}
	\item Defend the same as you would a strong (e.g. precision) club
	\end{itemize}
\item 1\hearts (Level 5 only)
	\begin{itemize}
	\item X = a heart overcall
	\item 1\spades = takeout of spades
	\item 1N = normal 1N overcall range showing a spade stop
	\item suits = natural overcalls
	\item 2\spades = whatever a cue of a natural 1 spade would mean 
	\item Jump bids = normal jump-overcall range
	\end{itemize}
\item 1\spades (Level 5 only)
	\begin{itemize}
	\item X = takeout of hearts
	\item 1N = normal 1N overcall range showing a heart stop
	\item suits = natural overcalls
	\item 2\hearts = whatever a cue of a natural 1 heart would mean 
	\item Jump bids = normal jump-overcall range
	\end{itemize}
\end{itemize}

\subsection{2 level bids}

\begin{itemize}
\item 2\clubs/\diamonds/\hearts
	\begin{itemize}
	\item X = an overcall in the suit bid
	\item suits = natural overcalls
	\item 2N = normal 2N over a weak-two range with some values in the other three suits
	\end{itemize}
\item 2\spades
	\begin{itemize}
	\item X = takeout of hearts
	\item 2N = normal 2N over a weak-two range with a decent heart stop
	\item suits = natural overcalls
	\end{itemize}
\item 2NT
	\begin{itemize}
	\item X = takeout (showing hearts and clubs)
	\item 3\clubs/\hearts = natural overcalls
	\item 3\diamonds/\spades = asking for a stop for 3NT
	\end{itemize}
\end{itemize}

\subsection{3/4 level bids}

\begin{itemize}
\item 3\clubs/\diamonds/\hearts
	\begin{itemize}
	\item X = an overcall in the suit bid
	\item next suit = takeout of that suit
	\item 3N = normal 3N over a 3-level preempt with a stop in the suit shown (not bid)
	\end{itemize}
\item 3\spades
	\begin{itemize}
	\item X = spade overcall (as of a 3-minor preempt or gambling 3N opening)
	\item suits = natural overcalls (as of a 4-minor preempt or gambling 3N opening)
	\end{itemize}
\item 3NT
	\begin{itemize}
	\item X = both majors (takeoutish)
	\item suits = natural overcalls (as of a 4-minor preempt)
	\end{itemize}
\item 4\clubs/\diamonds
	\begin{itemize}
	\item X = an overcall in the suit bid (as of a 3-major preempt)
	\item suits = natural overcalls (as of a 3-major preempt)
	\item our major = takeout
	\end{itemize}
\end{itemize}




\end{document}
