
\subsection{Openings}
\label{sec:openings}

\subsubsection{1 level}
\label{sec:open:1level}

Responses in Section \ref{sec:resp:1level}.

\paragraph{1\clubs}

One of the following \orange{7B1(i)}:

\begin{itemize}
\item (1\textsuperscript{st}/2\textsuperscript{nd} non-vul) 12--15 balanced (4333, 4432, 5332). May occasionally include some semi-balanced hands esp. 5422.
\item (1\textsuperscript{st}/2\textsuperscript{nd} vul) 11--13 balanced (4333, 4432, 5332). May occasionally include some semi-balanced hands esp. 5422.

\item 10--15, not balanced, no 5-card major and at most one 4-card major
\end{itemize}

\paragraph{1\diamonds}

16+, not game forcing \orange{7B1(ii)}. 
\begin{itemize}
\item If balanced or semi-balanced then 16--19 (17--19 1/2 Vul). 
\item If a two suiter (5422 or longer) or three suiter (4441 / 5440) then 16--GF. 
\item If a single suiter (6332 or longer) then 16--GF. (16--GF here is 'less than a 3-loser hand').
\end{itemize}

{\it

\paragraph{1\hearts~(Level 5)}

\begin{itemize}
\item 10--15 points, not balanced 4+ spades, may be longer hearts or a minor (possible canape), but will always have 5 spades or 4 hearts \orange{9A}.
\end{itemize}

}

{\color{CadetBlue}

\paragraph{1\hearts~(Level 4)}

\begin{itemize}
\item 10--15 points, not balanced, 4+ hearts, may contain a longer minor (possible canape) \orange{7B2}.
\end{itemize}
}

{\it

\paragraph{1\spades~(Level 5)}

\begin{itemize}
\item 10--15 points, not balanced. 5+ hearts, 0-3 spades \orange{9A}, may have a longer minor (possible canape).
\end{itemize}

}

{\color{CadetBlue}

\paragraph{1\spades~(Level 4)}

\begin{itemize}
\item 10--15 points, not balanced. 5+ spades \orange{7B2}, may have a longer minor (possible canape).
\end{itemize}
}

\paragraph{1NT}

Varies depending upon position and vulnerability \orange{5A5}

First or second seat non-vulnerable:
\begin{itemize}
\item 9--11 balanced \orange{7B3(i)}.
\end{itemize}

Third seat non-vulnerable:
\begin{itemize}
\item 9--15 balanced \orange{7B3(i)}.
\end{itemize}

First or second seat vulnerable:
\begin{itemize}
\item 14--16 balanced \orange{7B3(i)}.
\end{itemize}

All other positions and vulnerabilities:
\begin{itemize}
\item 12--15 balanced \orange{7B3(i)}.
\end{itemize}

For responses see Section~\ref{sec:resp:1nnat}.

\newpage 

\subsubsection{2 level}
\label{sec:open:2level}

Responses in Section \ref{sec:resp:2level}. Range in HCP for each weak option
is given, but may be wider ranging in 3rd or favourable vulnerability and will
be intermediate in 4th position (12--16).

\paragraph{2\clubs}
One of the following \orange{7C1(a) \& (b)(iv)}:
\begin{itemize}
\item Weak (5--9) with \diamonds (usually 6 cards)
\item Weak (5--9) with \hearts \& \spades, 5--4 or better
\item 20--23 balanced
\item Game forcing (3 losers) with \hearts \& \spades, 5--4 or better
\end{itemize}

\paragraph{2\diamonds}
One of the following \orange{7C1(a) \& (b)(iv)}:
\begin{itemize}
\item Weak (5--9) with \hearts (usually 6 cards)
\item Weak (5--9) with \spades \& \clubs, 5--4 or better 
\item Game forcing (3 losers) with \spades \& \clubs, 5--4 or better
\end{itemize}

\paragraph{2\hearts}
One of the following \orange{7C1(a) \& (b)(iv)}:
\begin{itemize}
\item Weak (5--9) with \spades (usually 6 cards)
\item Weak (5--9) with \clubs \& \diamonds, 5--4 or better
\item Game forcing (3 losers) with \clubs \& \diamonds, 5--4 or better
\end{itemize}

\paragraph{2\spades}
One of the following \orange{7C1(a) \& (b)(iv)}:
\begin{itemize}
\item Weak (5--9) with \clubs (usually 7 cards)
\item Weak (5--9) with \hearts \& one of the minors, 5--4 or better
\item Game forcing (3 losers) with \hearts \& one of the minors, 5--4 or better 
\end{itemize}

\paragraph{2NT}
One of the following \orange{7C1(a) \& (b)(iv)}:
\begin{itemize}
\item Weak (5--9) with \spades \& \diamonds, 5--4 or better
\item Game forcing (3 losers) with \spades \& \diamonds, 5--4 or better
\item 24+ (Game forcing) balanced hands
\end{itemize}

\subsubsection{Higher openings}
\label{sec:open:higher}

Responses in Section \ref{sec:resp:higher}.

\paragraph{3 level}

Three level bids are transfers. Either weak preempts or a 3-loser hand.
Typically seven cards, they may be occasionally a good six (particularly at
favourable vulnerability) or a bad eight.

\begin{itemize}
\item 3\clubs - GF clubs or 7 card preempt in diamonds \orange{7D}
\item 3\diamonds - GF clubs or 7 card preempt in hearts \orange{7D}
\item 3\hearts - GF clubs or 7 card preempt in spades \orange{7D}
\end{itemize}

\paragraph{3\spades}

GF spades or Gambling 3NT~\orange{7D}. 

In the Gambling case it shows a solid seven or eight card minor suit and nothing outside better than a
queen.  Solid in this case is defined as 60\% likely to run given all
distributions of cards in the other hands. The following suits are all
considered to be solid:

\begin{itemize}
\item AKQJxxx
\item AKQxxxx
\item AKxxxxxx
\end{itemize}

\paragraph{3NT}

Shows a 4-level preempt in either minor~\orange{7D}. Usually eight cards. 

\paragraph{4-level suit openings}

We play Namyats, so:

\begin{itemize}
\item 4\clubs - Strong 4-level bid in hearts~\orange{7D}
\item 4\diamonds - Strong 4-level bid in spades~\orange{7D}
\item 4\hearts - Purely preemptive 4-level bid in hearts
\item 4\spades - Purely preemptive 4-level bid in spades
\end{itemize}

Strong is defined as an 8--9 playing trick hand.

\paragraph{4NT}

Asks for specific aces~\orange{11L1}.

Will be bid with a hand missing at most two aces, or with a void. It must be happy playing at the six level opposite some responses and the seven level opposite some others.

\paragraph{5 of a major}

Shows an 11 trick hand missing two of the top three trump honours. Raise one level per trump honour.


